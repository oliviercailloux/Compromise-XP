\RequirePackage[l2tabu, orthodox]{nag}
\documentclass[pagesize, twoside=off, bibliography=totoc, DIV=calc, fontsize=12pt, a4paper]{scrartcl}
%Permits to copy eg x ⪰ y ⇔ v(x) ≥ v(y) from PDF to unicode data, and to search. From pdfTeX users manual. See https://tex.stackexchange.com/posts/comments/1203887.
	\input glyphtounicode
	\pdfgentounicode=1
%Latin Modern has more glyphs than Computer Modern, such as diacritical characters. fntguide commands to load the font before fontenc, to prevent default loading of cmr.
	\usepackage{lmodern}
%Encode resulting accented characters correctly in resulting PDF, permits copy from PDF.
	\usepackage[T1]{fontenc}
%UTF8 seems to be the default in recent TeX installations, but not all, see https://tex.stackexchange.com/a/370280.
	\usepackage[utf8]{inputenc}
%Provides \newunicodechar for easy definition of supplementary UTF8 characters such as → or ≤ for use in source code.
	\usepackage{newunicodechar}
%Text Companion fonts, much used together with CM-like fonts. Provides \texteuro and commands for text mode characters such as \textminus, \textrightarrow, \textlbrackdbl.
	\usepackage{textcomp}
%St Mary’s Road symbol font, used for ⟦ = \llbracket. The \SetSymbolFont command avoids spurious warnings, but also some valid ones, see https://tex.stackexchange.com/a/106719/.
	\usepackage{stmaryrd}\SetSymbolFont{stmry}{bold}{U}{stmry}{m}{n}
%Solves bug in lmodern, https://tex.stackexchange.com/a/261188; probably useful only for unusually big font sizes; and probably better to use exscale instead. Note that the authors of exscale write against this trick.
	%\DeclareFontShape{OMX}{cmex}{m}{n}{
		%<-7.5> cmex7
		%<7.5-8.5> cmex8
		%<8.5-9.5> cmex9
		%<9.5-> cmex10
	%}{}
	%\SetSymbolFont{largesymbols}{normal}{OMX}{cmex}{m}{n}
%More symbols (such as \sum) available in bold version, see https://github.com/latex3/latex2e/issues/71. In article mode (but not in presentation mode), also hides some potentially useful warnings such as when using $\bm{\llbracket}$, see stmaryrd in this document (not sure why).
	\DeclareFontShape{OMX}{cmex}{bx}{n}{%
	   <->sfixed*cmexb10%
	   }{}
	\SetSymbolFont{largesymbols}{bold}{OMX}{cmex}{bx}{n}
%https://english.stackexchange.com/questions/93008
	\usepackage[super]{nth}
%For small caps also in italics, see https://tex.stackexchange.com/questions/32942/italic-shape-needed-in-small-caps-fonts, https://tex.stackexchange.com/questions/284338/italic-small-caps-not-working.
	\usepackage{slantsc}
	\AtBeginDocument{%
		%“Since nearly no font family will contain real italic small caps variants, the best approach is to substitute them by slanted variants.” -- slantsc doc
		%\DeclareFontShape{T1}{lmr}{m}{scit}{<->ssub*lmr/m/scsl}{}%
		%There’s no bold small caps in Latin Modern, we switch to Computer Modern for bold small caps, see https://tex.stackexchange.com/a/22241
		%\DeclareFontShape{T1}{lmr}{bx}{sc}{<->ssub*cmr/bx/sc}{}%
		%\DeclareFontShape{T1}{lmr}{bx}{scit}{<->ssub*cmr/bx/scsl}{}%
	}
%Warn about missing characters.
	\tracinglostchars=2
%Nicer tables: provides \toprule, \midrule, \bottomrule.
	%\usepackage{booktabs}
%For new column type X which stretches; can be used together with booktabs, see https://tex.stackexchange.com/a/97137. “tabularx modifies the widths of the columns, whereas tabular* modifies the widths of the inter-column spaces.” Loads array.
	%\usepackage{tabularx}
%math-mode version of "l" column type. Requires \usepackage{array}.
	%\usepackage{array}
	%\newcolumntype{L}{>{$}l<{$}}
%Provides \xpretocmd and loads etoolbox which provides \apptocmd, \patchcmd, \newtoggle… Also loads xparse, which provides \NewDocumentCommand and similar commands intended as replacement of \newcommand in LaTeX3 for defining commands (see https://tex.stackexchange.com/q/98152 and https://github.com/latex3/latex2e/issues/89).
	\usepackage{xpatch}
%ntheorem doc says: “empheq provides an enhanced vertical placement of the endmarks”; must be loaded before ntheorem. Loads the mathtools package, which loads and fixes some bugs in amsmath and provides \DeclarePairedDelimiter. amsmath is considered a basic, mandatory package nowadays (Grätzer, More Math Into LaTeX).
	\usepackage[ntheorem]{empheq}
%Package frenchb asks to load natbib before babel-french. Package hyperref asks to load natbib before hyperref.
	\usepackage{natbib}

\newtoggle{LCpres}
	\newtoggle{LCart}
	\newtoggle{LCposter}
	\makeatletter
	\@ifclassloaded{beamer}{
		\toggletrue{LCpres}
		\togglefalse{LCart}
		\togglefalse{LCposter}
		\wlog{Presentation mode}
	}{
		\@ifclassloaded{tikzposter}{
			\toggletrue{LCposter}
			\togglefalse{LCpres}
			\togglefalse{LCart}
			\wlog{Poster mode}
		}{
			\toggletrue{LCart}
			\togglefalse{LCpres}
			\togglefalse{LCposter}
			\wlog{Article mode}
		}
	}
	\makeatother%

%Language options ([french, english]) should be on the document level (last is main); except with tikzposter: put [french, english] options next to \usepackage{babel} to avoid warning. beamer uses the \translate command for the appendix: omitting babel results in a warning, see https://github.com/josephwright/beamer/issues/449. Babel also seems required for \refname.
	\iftoggle{LCpres}{
		\usepackage{babel}
	}{
	}
	%\frenchbsetup{AutoSpacePunctuation=false}
%listings (1.7) does not allow multi-byte encodings. listingsutf8 works around this only for characters that can be represented in a known one-byte encoding and only for \lstinputlisting. Other workarounds: use literate mechanism; or escape to LaTeX (but breaks alignment).
	%\usepackage{listings}
	%\lstset{tabsize=2, basicstyle=\ttfamily, escapechar=§, literate={é}{{\'e}}1}
%I favor acro over acronym because the former is more recently updated (2018 VS 2015 at time of writing); has a longer user manual (about 40 pages VS 6 pages if not counting the example and implementation parts); has a command for capitalization; and acronym suffers a nasty bug when ac used in section, see https://tex.stackexchange.com/q/103483 (though this might be the fault of the silence package and might be solved in more recent versions, I do not know) and from a bug when used with cleveref, see https://tex.stackexchange.com/q/71364. However, loading it makes compilation time (one pass on this template) go from 0.6 to 1.4 seconds, see https://bitbucket.org/cgnieder/acro/issues/115.
	\usepackage[single]{acro}
	\DeclareAcronym{AMCD}{short=AMCD, long={Aide Multicritère à la Décision}}
\DeclareAcronym{AHP}{short=AHP, long={Analytic Hierarchy Process}}
\DeclareAcronym{AR}{short=AR, long={Argumentative Recommender}}
\DeclareAcronym{DA}{short=DA, long={Decision Analysis}}
\DeclareAcronym{DJ}{short=DJ, long={Deliberated Judgment}}
\DeclareAcronym{DM}{short=DM, long={Decision Maker}}
\DeclareAcronym{DP}{short=DP, long={Deliberated Preference}}
\DeclareAcronym{MAVT}{short=MAVT, long={Multiple Attribute Value Theory}}
\DeclareAcronym{MCDA}{short=MCDA, long={Multicriteria Decision Aid}}
\DeclareAcronym{MIP}{short=MIP, long={Mixed Integer Program}}
\DeclareAcronym{SEU}{short=SEU, long={Subjective Expected Utility}}


\iftoggle{LCpres}{
	%I favor fmtcount over nth because it is loaded by datetime anyway; and fmtcount warns about possible conflicts when loaded after nth.
	\usepackage{fmtcount}
	%For nice input of date of presentation. Must be loaded after the babel package. Has possible problems with srcletter: https://golatex.de/verwendung-von-babel-und-datetime-in-scrlttr2-schlaegt-fehlt-t14779.html.
	\usepackage[nodayofweek]{datetime}
}{
}
%For presentations, Beamer implicitely uses the pdfusetitle option. ntheorem doc says to load hyperref “before the first use of \newtheorem”. autonum doc mandates option hypertexnames=false. I want to highlight links only if necessary for the reader to recognize it as a link, to reduce distraction. In presentations, this is already taken care of by beamer (https://tex.stackexchange.com/a/262014). If using colorlinks=true in a presentation, see https://tex.stackexchange.com/q/203056. Crashes the first compilation with tikzposter, just compile again and the problem disappears, see https://tex.stackexchange.com/q/254257.
\makeatletter
\iftoggle{LCpres}{
	\usepackage{hyperref}
}{
	\usepackage[hypertexnames=false, pdfusetitle, linkbordercolor={1 1 1}, citebordercolor={1 1 1}, urlbordercolor={1 1 1}]{hyperref}
	%https://tex.stackexchange.com/a/466235
	\pdfstringdefDisableCommands{%
		\let\thanks\@gobble
	}
}
\makeatother
%urlbordercolor is used both for \url and \doi, which I think shouldn’t be colored, and for \href, thus might want to color manually when required. Requires xcolor.
	\NewDocumentCommand{\hrefblue}{mm}{\textcolor{blue}{\href{#1}{#2}}}
%hyperref doc says: “Package bookmark replaces hyperref’s bookmark organization by a new algorithm (...) Therefore I recommend using this package”.
	\usepackage{bookmark}
%Need to invoke hyperref explicitly to link to line numbers: \hyperlink{lintarget:mylinelabel}{\ref*{lin:mylinelabel}}, with \ref* to disable automatic link. Also see https://tex.stackexchange.com/q/428656 for referencing lines from another document.
	%\usepackage{lineno}
	%\NewDocumentCommand{\llabel}{m}{\hypertarget{lintarget:#1}{}\linelabel{lin:#1}}
	%\setlength\linenumbersep{9mm}
%For complex authors blocks. Seems like authblk wants to be later than hyperref, but sooner than silence. See https://tex.stackexchange.com/q/475513 for the patch to hyperref pdfauthor.
	\ExplSyntaxOn
	\seq_new:N \g_oc_hrauthor_seq
	\NewDocumentCommand{\addhrauthor}{m}{
		\seq_gput_right:Nn \g_oc_hrauthor_seq { #1 }
	}
	%Should be \NewExpandableDocumentCommand, but this is not yet provided by my version of xparse
	\DeclareExpandableDocumentCommand{\hrauthor}{}{
		\seq_use:Nn \g_oc_hrauthor_seq {,~}
	}
	\ExplSyntaxOff
	{
		\catcode`#=11\relax
		\gdef\fixauthor{\xpretocmd{\author}{\addhrauthor{#2}}{}{}}%
	}
	\iftoggle{LCart}{
		\usepackage{authblk}
		\renewcommand\Affilfont{\small}
		\fixauthor
		\AtBeginDocument{
		    \hypersetup{pdfauthor={\hrauthor}}
		}
	}{
	}
%I do not use floatrow, because it requires an ugly hack for proper functioning with KOMA script (see scrhack doc). Instead, the following command centers all floats (using \centering, as the center environment adds space, http://texblog.net/latex-archive/layout/center-centering/), and I manually place my table captions above and figure captions below their contents (https://tex.stackexchange.com/a/3253).
	\makeatletter
	\g@addto@macro\@floatboxreset\centering
	\makeatother
%Permits to customize enumeration display and references
	%\nottoggle{LCpres}{
		%\usepackage{enumitem} %follow list environments by a string to customize enumeration, example: \begin{description}[itemindent=8em, labelwidth=!] or \begin{enumerate}[label=({\roman*}), ref={\roman*}].
	%}{
	%}
%Provides \Centering, \RaggedLeft, and \RaggedRight and environments Center, FlushLeft, and FlushRight, which allow hyphenation. With tikzposter, seems to cause 1=1 to be printed in the middle of the poster.
	%\usepackage{ragged2e}
%To typeset units by closely following the “official” rules.
	%\usepackage[strict]{siunitx}
%Turns the doi provided by some bibliography styles into URLs. However, uses old-style dx.doi url (see 3.8 DOI system Proxy Server technical details, “Users may resolve DOI names that are structured to use the DOI system Proxy Server (https://doi.org (current, preferred) or earlier syntax http://dx.doi.org).”, https://www.doi.org/doi_handbook/3_Resolution.html). The patch solves this.
	\usepackage{doi}
	\makeatletter
	\patchcmd{\@doi}{http://dx.doi.org}{https://doi.org}{}{}
	\makeatother
%Makes sure upper case greek letters are italic as well.
	\usepackage{fixmath}
%Provides \mathbb; obsoletes latexsym (see http://tug.ctan.org/macros/latex/base/latexsym.dtx). Relatedly, \usepackage{eucal} to change the mathcal font and \usepackage[mathscr]{eucal} (apparently equivalent to \usepackage[mathscr]{euscript}) to supplement \mathcal with \mathscr. This last option is not very useful as both fonts are similar, and the intent of the authors of eucal was to provide a replacement to mathcal (see doc euscript). Also provides \mathfrak for supplementary letters.
	\usepackage{amsfonts}
%Provides a beautiful (IMHO) \mathscr and really different than \mathcal, for supplementary uppercase letters. But there is no bold version. Alternative: mathrsfs (more slanted), but when used with tikzposter, it warns about size substitution, see https://tex.stackexchange.com/q/495167.
	\usepackage[scr]{rsfso}
%Multiple means to produce bold math: \mathbf, \boldmath (defined to be \mathversion{bold}, see fntguide), \pmb, \boldsymbol (all legacy, from LaTeX base and AMS), \bm (the most recommended one), \mathbold from package fixmath (I don’t see its advantage over \boldsymbol).
%“The \boldsymbol command is obtained preferably by using the bm package, which provides a newer, more powerful version than the one provided by the amsmath package. Generally speaking, it is ill-advised to apply \boldsymbol to more than one symbol at a time.” — AMS Short math guide. “If no bold font appears to be available for a particular symbol, \bm will use ‘poor man’s bold’” — bm. It is “best to load the package after any packages that define new symbol fonts” – bm. bm defines \boldsymbol as synonym to \bm. \boldmath accesses the correct font if it exists; it is used by \bm when appropriate. See https://tex.stackexchange.com/a/10643 and https://github.com/latex3/latex2e/issues/71 for some difficulties with \bm.
	\usepackage{bm}
	\nottoggle{LCpres}{
	%https://ctan.org/pkg/amsmath recommends ntheorem, which supersedes amsthm, which corrects the spacing of proclamations and allows for theoremstyle. Option standard loads amssymb and latexsym. Must be loaded after amsmath (from ntheorem doc). From cleveref doc, “ntheorem is fully supported and even recommended”; says to load cleveref after ntheorem. When used with tikzposter, warns about size substitution for the lasy (latexsym) font when using \url, because ntheorem loads latexsym; relatedly (but not directly related to ntheorem), size substitution warning with the cmex font happens when loading amsmath and using \url. According to https://tex.stackexchange.com/q/535950, ntheorem “seems essentially unmaintaned and has severe problems”, but I use it anyway because it is very handy. Yields “! LaTeX Error: Something's wrong--perhaps a missing \item.” if some theorem follows thebibliography.
		\usepackage[thmmarks, amsmath, standard, hyperref]{ntheorem}
		%empheq doc says to do this after loading ntheorem
		\usetagform{default}
	%Provides \cref. Unfortunately, cref fails when the language is French and referring to a label whose name contains a colon (https://tex.stackexchange.com/q/83798). Use \cref{sec\string:intro} to work around this. cleveref should go “laster” than hyperref.
		\usepackage{cleveref}
	}{
	}
	\nottoggle{LCposter}{
	%Equations get numbers iff they are referenced. Loading order should be “amsmath → hyperref → cleveref → autonum”, according to autonum doc. Use this in preference to the showonlyrefs option from mathtools, see https://tex.stackexchange.com/q/459918 and autonum doc. See https://tex.stackexchange.com/a/285953 for the etex line. Incompatible with my version of tikzposter (produces “! Improper \prevdepth”).
		\expandafter\def\csname ver@etex.sty\endcsname{3000/12/31}\let\globcount\newcount
		\usepackage{autonum}
	}{
	}
%Also loaded by tikz.
	\usepackage{xcolor}
\iftoggle{LCpres}{
	\usepackage{tikz}
	%\usetikzlibrary{babel, matrix, fit, plotmarks, calc, trees, shapes.geometric, positioning, plothandlers, arrows, shapes.multipart}
}{
}
%Vizualization, on top of TikZ
	%\usepackage{pgfplots}
	%\pgfplotsset{compat=1.14}
\usepackage{graphicx}
	\graphicspath{{graphics/}}

%Provides \printlength{length}, useful for debugging.
	%\usepackage{printlen}
	%\uselengthunit{mm}

\iftoggle{LCpres}{
	\usepackage{appendixnumberbeamer}
	%I have yet to see anyone actually use these navigation symbols; let’s disable them
	\setbeamertemplate{navigation symbols}{} 
	\usepackage{preamble/beamerthemeParisFrance}
	\setcounter{tocdepth}{10}
}{
}

%Do not use the displaymath environment: use equation. Do not use the eqnarray or eqnarray* environments: use align(*). This improves spacing. (See l2tabu or amsldoc.)


%Requires package xcolor.
%\definecolor{ao(english)}{rgb}{0.0, 0.5, 0.0}
\NewDocumentCommand{\commentOC}{m}{\textcolor{blue}{\small$\big[$OC: #1$\big]$}}
\NewDocumentCommand{\commentRS}{m}{\textcolor{red}{\small$\big[$YM: #1$\big]$}}

\bibliographystyle{abbrvnat}
\NewDocumentCommand{\possessivecite}{mO{}}{\citeauthor{#1}’s \citeyearpar[#2]{#1}}
\NewDocumentCommand{\Possessivecite}{mO{}}{\Citeauthor{#1}’s \citeyearpar[#2]{#1}}%TODO test

%https://tex.stackexchange.com/a/467188, https://tex.stackexchange.com/a/36088 - uncomment if one of those symbols is used.
%\DeclareFontFamily{U} {MnSymbolD}{}
%\DeclareFontShape{U}{MnSymbolD}{m}{n}{
%  <-6> MnSymbolD5
%  <6-7> MnSymbolD6
%  <7-8> MnSymbolD7
%  <8-9> MnSymbolD8
%  <9-10> MnSymbolD9
%  <10-12> MnSymbolD10
%  <12-> MnSymbolD12}{}
%\DeclareFontShape{U}{MnSymbolD}{b}{n}{
%  <-6> MnSymbolD-Bold5
%  <6-7> MnSymbolD-Bold6
%  <7-8> MnSymbolD-Bold7
%  <8-9> MnSymbolD-Bold8
%  <9-10> MnSymbolD-Bold9
%  <10-12> MnSymbolD-Bold10
%  <12-> MnSymbolD-Bold12}{}
%\DeclareSymbolFont{MnSyD} {U} {MnSymbolD}{m}{n}
%\DeclareMathSymbol{\ntriplesim}{\mathrel}{MnSyD}{126}
%\DeclareMathSymbol{\nlessgtr}{\mathrel}{MnSyD}{192}
%\DeclareMathSymbol{\ngtrless}{\mathrel}{MnSyD}{193}
%\DeclareMathSymbol{\nlesseqgtr}{\mathrel}{MnSyD}{194}
%\DeclareMathSymbol{\ngtreqless}{\mathrel}{MnSyD}{195}
%\DeclareMathSymbol{\nlesseqgtrslant}{\mathrel}{MnSyD}{198}
%\DeclareMathSymbol{\ngtreqlessslant}{\mathrel}{MnSyD}{199}
%\DeclareMathSymbol{\npreccurlyeq}{\mathrel}{MnSyD}{228}
%\DeclareMathSymbol{\nsucccurlyeq}{\mathrel}{MnSyD}{229}
%\DeclareFontFamily{U} {MnSymbolA}{}
%\DeclareFontShape{U}{MnSymbolA}{m}{n}{
%  <-6> MnSymbolA5
%  <6-7> MnSymbolA6
%  <7-8> MnSymbolA7
%  <8-9> MnSymbolA8
%  <9-10> MnSymbolA9
%  <10-12> MnSymbolA10
%  <12-> MnSymbolA12}{}
%\DeclareFontShape{U}{MnSymbolA}{b}{n}{
%  <-6> MnSymbolA-Bold5
%  <6-7> MnSymbolA-Bold6
%  <7-8> MnSymbolA-Bold7
%  <8-9> MnSymbolA-Bold8
%  <9-10> MnSymbolA-Bold9
%  <10-12> MnSymbolA-Bold10
%  <12-> MnSymbolA-Bold12}{}
%\DeclareSymbolFont{MnSyA} {U} {MnSymbolA}{m}{n}
%%Rightwards wave arrow: ↝. Alternative: \rightsquigarrow from amssymb, but it’s uglier
%\DeclareMathSymbol{\rightlsquigarrow}{\mathrel}{MnSyA}{160}

%0394 
\newunicodechar{Δ}{\Delta}
%03B3 Greek Small Letter Gamma
\newunicodechar{γ}{\gamma}
%03B4 Greek Small Letter Delta
\newunicodechar{δ}{\delta}
%2115 Double-Struck Capital N
\newunicodechar{ℕ}{\mathbb{N}}
%211D Double-Struck Capital R
\newunicodechar{ℝ}{\mathbb{R}}
%21CF Rightwards Double Arrow with Stroke
\newunicodechar{⇏}{\nRightarrow}
%21D0 Leftwards Double Arrow
\newunicodechar{⇐}{\ensuremath{\Leftarrow}}
%21D2 Rightwards Double Arrow
\newunicodechar{⇒}{\ensuremath{\Rightarrow}}
%21D4 Left Right Double Arrow
\newunicodechar{⇔}{\Leftrightarrow}
%21DD Rightwards Squiggle Arrow
\newunicodechar{⇝}{\rightsquigarrow}
%2205 Empty Set
\newunicodechar{∅}{\emptyset}
%2212 Minus Sign
\newunicodechar{−}{\ifmmode{-}\else\textminus\fi}
%2227 Logical And
\newunicodechar{∧}{\land}
%2228 Logical Or
\newunicodechar{∨}{\lor}
%2229 Intersection
\newunicodechar{∩}{\cap}
%222A Union
\newunicodechar{∪}{\cup}
%2260 Not Equal To (handy also as text in informal writing)
\newunicodechar{≠}{\ensuremath{\neq}}
%2264 Less-Than or Equal To
\newunicodechar{≤}{\leq}
%2265 Greater-Than or Equal To
\newunicodechar{≥}{\geq}
%2270 Neither Less-Than nor Equal To
\newunicodechar{≰}{\nleq}
%2271 Neither Greater-Than nor Equal To
\newunicodechar{≱}{\ngeq}
%2272 Less-Than or Equivalent To
\newunicodechar{≲}{\lesssim}
%2273 Greater-Than or Equivalent To
\newunicodechar{≳}{\gtrsim}
%2274 Neither Less-Than nor Equivalent To – also, from MnSymbol: \nprecsim, a more exact match to the Unicode symbol; and \npreccurlyeq, too small
\newunicodechar{≴}{\not\preccurlyeq}
%2275 Neither Greater-Than nor Equivalent To
\newunicodechar{≵}{\not\succcurlyeq}
%2279 Neither Greater-Than nor Less-Than – requires MnSymbol; also \nlessgtr from txfonts/pxfonts, \ngtreqless from MnSymbol (but much higher), \ngtrless from MnSymbol (a more exact match to the Unicode symbol); for incomparability (not matching this Unicode symbol), may also consider \ntriplesim from MnSymbol,\nparallelslant from fourier, \between from mathabx, or ⋈
\newunicodechar{≹}{\ngtreqlessslant}
%227A Precedes
\newunicodechar{≺}{\prec}
%227B Succeeds
\newunicodechar{≻}{\succ}
%227C Precedes or Equal To
\newunicodechar{≼}{\preccurlyeq}
%227D Succeeds or Equal To
\newunicodechar{≽}{\succcurlyeq}
%227E Precedes or Equivalent To
\newunicodechar{≾}{\precsim}
%227F Succeeds or Equivalent To
\newunicodechar{≿}{\succsim}
%2280 Does Not Precede
\newunicodechar{⊀}{\nprec}
%2281 Does Not Succeed
\newunicodechar{⊁}{\nsucc}
%2286
\newunicodechar{⊆}{\subseteq}
%22B2 Normal Subgroup Of – using \vartriangleleft from amsfonts, which goes well with \trianglelefteq, \ntriangleright, and so on, also from amsfonts; another possibility is \lhd from latexsym, which seems visually equivalent to \vartriangleleft from amsfonts; latexsym also has ⊴=\unlhd, but doesn’t have a symbol for ⊴. Other related symbols: \triangleleft from latesym package is too small; fdsymbol provides \triangleleft=\medtriangleleft and \vartriangleleft=\smalltriangleleft; MnSymbol provides \medtriangleleft and \vartriangleleft=\lessclosed=\lhd which are smaller than \vartriangleleft from amsfont; \vartriangleleft from mathabx (p. 67), looks different (wider); also \vartriangleleft from boisik (p. 69) looks still different; \vartriangleleft=\lhd from stix are smaller. Oddly enough, \triangleright appears as the LMMathItalic12-Regular font whereas \rhd appears as LASY10 and \vartriangleright appears as MSAM10.
\newunicodechar{⊲}{\vartriangleleft}
%22B3 Contains as Normal Subgroup (also: 25B7 White right-pointing triangle or 25B9 White right-pointing small triangle)
\newunicodechar{⊳}{\vartriangleright}
%22B4 Normal Subgroup of or Equal To
\newunicodechar{⊴}{\trianglelefteq}
%22B5 Contains as Normal Subgroup or Equal To
\newunicodechar{⊵}{\trianglerighteq}
%22C8 Bowtie
\newunicodechar{⋈}{\bowtie}
%22EA Not Normal Subgroup Of
\newunicodechar{⋪}{\ntriangleleft}
%22EB Does Not Contain As Normal Subgroup
\newunicodechar{⋫}{\ntriangleright}
%22EC Not Normal Subgroup of or Equal To
\newunicodechar{⋬}{\ntrianglelefteq}
%22ED Does Not Contain as Normal Subgroup or Equal
\newunicodechar{⋭}{\ntrianglerighteq}
%25A1 White Square
\newunicodechar{□}{\Box}
%27E6 Mathematical Left White Square Bracket – requires stmaryrd (alternative: \text{\textlbrackdbl}, but ugly if used in an italicized text such as a theorem)
\newunicodechar{⟦}{\llbracket}
%27E7 Mathematical Right White Square Bracket
\newunicodechar{⟧}{\rrbracket}
%27FC Long Rightwards Arrow from Bar
\newunicodechar{⟼}{\longmapsto}
%2AB0 Succeeds Above Single-Line Equals Sign
\newunicodechar{⪰}{\succeq}
%301A Left White Square Bracket
\newunicodechar{〚}{\textlbrackdbl}
%301B Right White Square Bracket
\newunicodechar{〛}{\textrbrackdbl}
%→ is defined by default as \textrightarrow, which is invalid in math mode. Same thing for the three other commands. Using \DeclareUnicodeCharacter instead of \newunicodechar because the latter warns about the previous definition.
%← Leftwards Arrow
\DeclareUnicodeCharacter{2190}{\ifmmode\leftarrow\else\textleftarrow\fi}
%→ Rightwards Arrow
\DeclareUnicodeCharacter{2192}{\ifmmode\rightarrow\else\textrightarrow\fi}
%¬ Not Sign
\DeclareUnicodeCharacter{00AC}{\ifmmode\lnot\else\textlnot\fi}
%… Horizontal Ellipsis
\DeclareUnicodeCharacter{2026}{\ifmmode\dots\else\textellipsis\fi}
%× Multiplication Sign
\DeclareUnicodeCharacter{00D7}{\ifmmode\times\else\texttimes\fi}
%Permits to really obtain a straight quote when typing a straight quote; potentially dangerous, see https://tex.stackexchange.com/a/521999
\catcode`\'=\active
\DeclareUnicodeCharacter{0027}{\ifmmode^\prime\else\textquotesingle\fi}


\NewDocumentCommand{\R}{}{ℝ}
\NewDocumentCommand{\N}{}{ℕ}
%\mathscr is rounder than \mathcal.
\NewDocumentCommand{\powerset}{m}{\mathscr{P}(#1)}
%Powerset without zero.
\NewDocumentCommand{\powersetz}{m}{\mathscr{P}^*(#1)}
%https://tex.stackexchange.com/a/45732, works within both \set and \set*, same spacing than \mid (https://tex.stackexchange.com/a/52905).
\NewDocumentCommand{\suchthat}{}{\;\ifnum\currentgrouptype=16 \middle\fi|\;}
%Integer interval.
\NewDocumentCommand{\intvl}{m}{⟦#1⟧}
%Allows for \abs and \abs*, which resizes the delimiters.
\DeclarePairedDelimiter\abs{\lvert}{\rvert}
\DeclarePairedDelimiter\card{\lvert}{\rvert}
\DeclarePairedDelimiter\floor{\lfloor}{\rfloor}
\DeclarePairedDelimiter\ceil{\lceil}{\rceil}
%Perhaps should use U+2016 ‖ DOUBLE VERTICAL LINE here?
\DeclarePairedDelimiter\norm{\lVert}{\rVert}
%From mathtools. Better than using the package braket because braket introduces possibly undesirable space. Then: \begin{equation}\set*{x \in \R^2 \suchthat \norm{x}<5}\end{equation}.
\DeclarePairedDelimiter\set{\{}{\}}
\DeclareMathOperator*{\argmax}{arg\,max}
\DeclareMathOperator*{\argmin}{arg\,min}

%UTR #25: Unicode support for mathematics recommend to use the straight form of phi (by default, given by \phi) rather than the curly one (by default, given by \varphi), and thus use \phi for the mathematical symbol and not \varphi. I however prefer the curly form because the straight form is too easy to mix up with the symbol for empty set.
\let\phi\varphi

%The amssymb solution.
%\NewDocumentCommand{\restr}{mm}{{#1}_{\restriction #2}}
%Another acceptable solution.
%\NewDocumentCommand{\restr}{mm}{{#1|}_{#2}}
%https://tex.stackexchange.com/a/278631; drawback being that sometimes the text collides with the line below.
\NewDocumentCommand\restr{mm}{#1\raisebox{-.5ex}{$|$}_{#2}}


%Decision Theory (MCDA and SC)
\NewDocumentCommand{\allalts}{}{\mathscr{A}}
\NewDocumentCommand{\allcrits}{}{\mathscr{C}}
\NewDocumentCommand{\alts}{}{A}
\NewDocumentCommand{\dm}{}{i}
\NewDocumentCommand{\allF}{}{\mathscr{F}}
\NewDocumentCommand{\allvoters}{}{\mathscr{N}}
\NewDocumentCommand{\voters}{}{N}
\NewDocumentCommand{\allprofs}{}{\linors^{\set{1, 2}}}
\NewDocumentCommand{\fprofs}{}{\mathscr{G}}
\NewDocumentCommand{\prof}{}{P}
\NewDocumentCommand{\ibar}{}{\overline{i}}
\NewDocumentCommand{\lprof}{}{\lambda_P}
\NewDocumentCommand{\lprofi}{O{x}}{\lambda_P(#1)_i}
\NewDocumentCommand{\lprofibar}{O{x}}{\lambda_P(#1)_{\overline{i}}}
\NewDocumentCommand{\ineq}{}{(d \circ \lambda_P)}

\NewDocumentCommand{\linors}{}{\mathcal{L}(\allalts)}
%Thanks to https://tex.stackexchange.com/q/154549
	%\makeatletter
	%\def\@myRgood@#1#2{\mathrel{R^X_{#2}}}
	%\def\myRgood{\@ifnextchar_{\@myRgood@}{\mathrel{R^X}}}
	%\makeatother
\NewDocumentCommand{\pref}{}{\succ}
\NewDocumentCommand{\prefto}{}{\succ^{\mkern-8mu -1}}
\NewDocumentCommand{\preftoeq}{}{\succeq^{\mkern-8mu -1}}
\NewDocumentCommand{\prefi}{O{i}}{\succ_{#1}}
\NewDocumentCommand{\prefiinv}{O{i}}{\prec_{#1}}
\NewDocumentCommand{\PO}{}{\mathit{PO}}
\NewDocumentCommand{\paretopt}{}{\mathit{PO}}
\NewDocumentCommand{\SPPd}{}{\Sigma^\text{PPd}}
\NewDocumentCommand{\SAll}{}{\Sigma^\text{All}}
\NewDocumentCommand{\SThreshold}{}{\Sigma_\text{threshold}}
\NewDocumentCommand{\vpr}{}{\boldsymbol{v}}

\NewDocumentCommand{\musigma}{O{\sigma}O{P}}{\min_{{#1}\circ\lambda_{{#2}}}(A)}
\NewDocumentCommand{\mustar}{O{\sigma}O{P}}{\min_{{#1} \circ \lambda_{#2}} (\paretopt({#2}))}
\NewDocumentCommand{\minineq}{O{\allalts}}{\min_{#1}(d \circ \lambda_P)}
\NewDocumentCommand{\MS}{}{\mathit{MS}}
\NewDocumentCommand{\MSP}{}{\mathit{MS(P)}}
\NewDocumentCommand{\FB}{}{\mathit{FB}}
\NewDocumentCommand{\FBP}{}{\mathit{FB}(P)}
\NewDocumentCommand{\POP}{}{\mathit{PO}(P)}

\NewDocumentCommand{\alllosses}{}{\intvl{0, m-1}^{\set{1, 2}}}

\NewDocumentCommand{\Ptop}{}{\bar{P}}
\NewDocumentCommand{\sigmatop}{}{\bar{\sigma}}

\NewDocumentCommand{\fltwo}{}{\floor{\bar{l_2}}}
\NewDocumentCommand{\bltwo}{}{\bar{l_2}}

\newtheorem{conjecture}{Conjecture}

\usepackage{csvsimple}

%I find these settings useful in draft mode. Should be removed for final versions.
	%Which line breaks are chosen: accept worse lines, therefore reducing risk of overfull lines. Default = 200.
		\tolerance=2000
	%Accept overfull hbox up to...
		\hfuzz=2cm
	%Reduces verbosity about the bad line breaks.
		\hbadness 5000
	%Reduces verbosity about the underful vboxes.
		\vbadness=1300

\title{Compromise XPs}
\author[1]{Olivier Cailloux}
\author[2]{Ayça Ebru Giritligil}
\author[2]{Ipek Ozkal Sanver}
\author[1]{Remzi Sanver}
\affil[1]{Université Paris-Dauphine, PSL Research University, CNRS, LAMSADE, 75016 PARIS, FRANCE}
\affil[2]{Bilgi, …}
\hypersetup{
	pdfsubject={Social choice},
	pdfkeywords={axiomatic analysis},
}

\begin{document}
\maketitle

%\begin{abstract}
%	
%\end{abstract}

\section{Introduction}
\label{sec:introduction}
Goal: investigate empirically the conditions in which people adopt a compromise notion based on minimal inequality of losses, as opposed to more classical compromise notions, in two-persons situations.

\section{Basic definitions and notation}
\label{sec:notation}
We have a non empty set of alternatives $\allalts$ with $\card{\allalts} = m$ and a set $N = \set{1, 2}$ of two individuals. The possible profiles are $\linors^{\set{1, 2}}$. A \ac{SCR} is a function $f: \linors^{\set{1, 2}} → \powersetz{\allalts}$. Given $\prof \in \allprofs$, the set of Pareto-optimal alternatives is $\POP = \set{x \in \allalts \suchthat \forall y \in \allalts: x \prefi[1] y \lor x \prefi[2] y}$.

Given $\prof \in \allprofs$ and $x \in \allalts$, let $\lprof(x): \set{1, 2} → \intvl{0, m - 1}$ associate to each individual her loss at $x$, defined as the number of alternatives that are strictly preferred to $x$ in her preference $\prefi$, namely, $\lprof(x)_i = \card{{\prefiinv}(x)}$, where ${\prefiinv}(x)$ designate the upper contour set of $x$.
Let $\min\lprof(x) = \min_{i \in N}{\lprof(x)_i}$, $\max\lprof(x) = \max_{i \in N}{\lprof(x)_i}$ and $\sum \lprof(x) = \sum_{i \in N} \lprof(x)_i$ designate the minimal value, maximal value and sum of the losses of $x$.
Given a loss vector $l \in \alllosses$, define $d(l) = \abs{l_1 - l_2} = \max l - \min l$.

Given $\prof$, note that $\argmin_{\POP} (d \circ \lprof) = \set{x \in \POP \suchthat (d \circ \lprof)(x) = \min_{y \in \POP} (d \circ \lprof)(y)}$ is the least unequal alternatives among the Pareto-optimal ones, according to the measure $d \circ \lprof$.
Given $\prof$, note that $\min_\allalts \max \lprof$ is the threshold that is reached when going from the first (best) rank downwards until some alternative is unanimously considered worth that rank or better. 

We study the “min spread” SCR $\MSP = \argmin_{\POP} (d \circ \lprof)$, which picks the least unequal alternatives among the Pareto-optimal ones, according to the differences of losses; the Fallback-Bargaining SCR \citep{Brams2001} $\FBP = \argmin_\allalts \max \lprof$; and the Borda SCR $B(P) = \argmin \sum \lprof$ which selects the alternatives that minimize the sum of losses.

\commentOC{We should have a look at \doi{10.1287/mnsc.2021.4025}.}

\section{Filtering and classifying profiles}
\subsection{Filtering profiles}
We are interested in selecting profiles on which $\FB$ and $\MS$ disagree. 
As a minimal requirement, we thus request $\MSP \nsubseteq \FBP$. 
This implies $\card{\FBP} = 1$; $\card{\MSP} = 1$; $\min_\allalts \max \lprof < \frac{m}{2}$ (see \cref{sec:proofs} for the proofs).

We also require $\prof$ to contain no Pareto-dominated alternative with a smaller spread than $\minineq[\POP]$.
This constraint stems from a desire to focus on the subject of our study, and not perturb the subject by letting alternatives that do not distinguish our rules of interest appear possibly attractive. Studying the attractivity of Pareto-dominated alternatives is an interesting goal but requires its own design.

Let $\fprofs = \set{\prof \in \allprofs \suchthat \MSP \nsubseteq \FBP \land \forall a \in \allalts: (d \circ \lprof)(a) ≥ \minineq[\POP]}$ designate the remaining profiles.

During a run with a given subject, we want the spread of the Fallback-Bargaining winner, $d(\lprof(x))$, to reach high enough values compared to $d(\lprof(y))$, in order to ensure a sufficient contrast with min spread. We also want $m$ to be small enough. \Cref{th:conds} indicates that $\delta = d(\lprof(x)) - d(\lprof(y)) < \frac{m}{2} - 2$. As a compromise, we opt for $m = 13$, which permits to reach $\delta = 4$.

\subsection{Classifying profiles}
Given $\prof \in \fprofs$ (thus with $\card{\FBP} = \card{\MSP} = 1$), we write $\set{x} = \FBP$ and $\set{y} = \MSP$.

To each profile $\prof \in \fprofs$ can be associated these properties.
\footnote{About \cref{it:easy}, it can be shown that by our requirements on $\prof$, $\min \lprof(z) > \max \lprof(y)$ implies the other conditions for being easily excluded, and, as the alternatives $\set{w ≠ x \suchthat \lprof(w)_1 ≤ \rho}$ have to satisfy $\lprof(w)_2 > \rho$, we obtain $e = 12 - 2 \rho$.}
\begin{enumerate}
	\item The losses of $x$ and $y$, $\lprof(x)$ and $\lprof(y)$
	\item Whether $x$ is the unique best average Rank.
\end{enumerate}
This in turn determines the following properties.
\begin{enumerate}
	\item The spread of the Fallback-Bargaining winner and of the min spread winner, $d(\lprof(x))$ and $d(\lprof(y))$.
	\item \label{it:sumRank} The sum of the losses of $x$ and $y$, $\sum \lprof(x)$ and $\sum \lprof(y)$.
	\item \label{it:avgRank} The average loss of $x$ and $y$, $\frac{\sum \lprof(x)}{2}$ and $\frac{\sum \lprof(y)}{2}$.
	\item \label{it:easy} Keeping the indicators above fixed, we prefer “easier” profiles, as captured by the number $e = \card{\set{z \in \allalts \suchthat \min \lprof(z) > \max \lprof(x) \land \min \lprof(z) > \max \lprof(y) \land z \notin \POP \land d(\lprof(z)) ≥ d(\lprof(y))}}$ of alternatives in $\prof$ that we consider as “easily excluded” from the set of promising candidates (because of their readily visible relatively bad quality in terms of absolute ranks and spread).
\end{enumerate} 

\subsection{Enumerating classes}
\begin{corollary}
	$\forall \prof \in \allprofs \suchthat \argmin_{\POP} (d \circ \lprof) \nsubseteq \FBP$, with $m = 13$:
	\begin{enumerate}
		\item $4 ≤ \max \lprof(y) ≤ 8$;
		\item $\max \set{2, 2 \max \lprof(y) - 12} ≤ \min \lprof(y) ≤ \min \set{12 - \max \lprof(y), \max \lprof(y) - 2}$;
		\item $\max \set{\min \lprof(y) + 1, \max \lprof(y) - \min \lprof(y) + 1} ≤ \max \lprof(x) ≤ \min \set{\max \lprof(y) - 1, 13 - \max \lprof(y)}$;
		\item $\min \lprof(x) ≤ \max \lprof(x) - (\max \lprof(y) - \min \lprof(y) + 1)$.
	\end{enumerate}
\end{corollary}

\begin{corollary}
	$\forall \prof \in \allprofs \suchthat \argmin_{\POP} (d \circ \lprof) \nsubseteq \FBP \land \max \lprof(y) ≤ 7$, with $m = 13$:
	\begin{enumerate}
		\item $4 ≤ \max \lprof(y) ≤ 7$;
		\item $2 ≤ \min \lprof(y) ≤ \max \lprof(y) - 2$;
		\item $\max \set{\min \lprof(y) + 1, \max \lprof(y) - \min \lprof(y) + 1} ≤ \max \lprof(x) ≤ \max \lprof(y) - 1$;%
		\footnote{The bound that triggers in $\max \set{\min \lprof(y) + 1, \max \lprof(y) - \min \lprof(y) + 1}$ depends whether $\min \lprof(y) ≥ \frac{\max \lprof(y)}{2}$;  which varies for all $5 ≤ \max \lprof(y) ≤ 7$ (they are equal for $\max \lprof(y) = 4$).}
		\item $\min \lprof(x) ≤ \max \lprof(x) - (\max \lprof(y) - \min \lprof(y) + 1)$.
	\end{enumerate}
\end{corollary}

\subsection{For m 13}
For $\max \lprof(y) = 8$, we get $4 ≤ \min \lprof(y) ≤ 4$.
For $\max \lprof(y) = 8$ and $\min \lprof(y) = 4$, we get $5 ≤ \max \lprof(x) ≤ 5$ and $\min \lprof(x) ≤ \max \lprof(x) - 5$ thus only $\max \lprof(x) = 5$ and $\min \lprof(x) = 0$ is possible.
For $\max \lprof(y) = 8$: 1 class.

For $\max \lprof(y) = 7$, we get $2 ≤ \min \lprof(y) ≤ 5$.
For $\max \lprof(y) = 7$ and $\min \lprof(y) = 5$, we get $6 ≤ \max \lprof(x) ≤ 6$ and $\min \lprof(x) ≤ \max \lprof(x) - 3$ thus only $\max \lprof(x) = 6$ and $0 ≤ \min \lprof(x) ≤ 3$ are possible.
For $\max \lprof(y) = 7$ and $\min \lprof(y) = 4$, we get $5 ≤ \max \lprof(x) ≤ 6$ and $\min \lprof(x) ≤ \max \lprof(x) - 4$ thus only $\max \lprof(x) = 6$ and $0 ≤ \min \lprof(x) ≤ 2$; and $\max \lprof(x) = 5$ and $0 ≤ \min \lprof(x) = 1$ are possible.
For $\max \lprof(y) = 7$ and $\min \lprof(y) = 3$, we get $5 ≤ \max \lprof(x) ≤ 6$ and $\min \lprof(x) ≤ \max \lprof(x) - 5$ thus only $\max \lprof(x) = 6$ and $0 ≤ \min \lprof(x) ≤ 1$; and $\max \lprof(x) = 5$ and $\min \lprof(x) = 0$ are possible.
For $\max \lprof(y) = 7$ and $\min \lprof(y) = 2$, we get $6 ≤ \max \lprof(x) ≤ 6$ and $\min \lprof(x) ≤ \max \lprof(x) - 6$ thus only $\max \lprof(x) = 6$ and $\min \lprof(x) = 0$ is possible.
For $\max \lprof(y) = 7$: 13 classes.

For $\max \lprof(y) = 6$, we get $2 ≤ \min \lprof(y) ≤ 4$.
For $\max \lprof(y) = 6$ and $\min \lprof(y) = 4$, we get $5 ≤ \max \lprof(x) ≤ 5$ and $\min \lprof(x) ≤ \max \lprof(x) - 3$ thus only $\max \lprof(x) = 5$ and $0 ≤ \min \lprof(x) ≤ 2$ are possible.
For $\max \lprof(y) = 6$ and $\min \lprof(y) = 3$, we get $4 ≤ \max \lprof(x) ≤ 5$ and $\min \lprof(x) ≤ \max \lprof(x) - 4$ thus only $\max \lprof(x) = 5$ and $0 ≤ \min \lprof(x) ≤ 1$; and $\max \lprof(x) = 4$ and $\min \lprof(x) = 0$ are possible.
For $\max \lprof(y) = 6$ and $\min \lprof(y) = 2$, we get $5 ≤ \max \lprof(x) ≤ 5$ and $\min \lprof(x) ≤ \max \lprof(x) - 5$ thus only $\max \lprof(x) = 5$ and $\min \lprof(x) = 0$ is possible.
For $\max \lprof(y) = 6$: 7 classes.

For $\max \lprof(y) = 5$, we get $2 ≤ \min \lprof(y) ≤ 3$.
For $\max \lprof(y) = 5$ and $\min \lprof(y) = 3$, we get $4 ≤ \max \lprof(x) ≤ 4$ and $\min \lprof(x) ≤ \max \lprof(x) - 3$ thus only $\max \lprof(x) = 4$ and $0 ≤ \min \lprof(x) ≤ 1$ are possible.
For $\max \lprof(y) = 5$ and $\min \lprof(y) = 2$, we get $4 ≤ \max \lprof(x) ≤ 4$ and $\min \lprof(x) ≤ \max \lprof(x) - 4$ thus only $\max \lprof(x) = 4$ and $\min \lprof(x) = 0$ is possible.
For $\max \lprof(y) = 5$: 3 classes.

For $\max \lprof(y) = 4$, we get $2 ≤ \min \lprof(y) ≤ 2$.
For $\max \lprof(y) = 4$ and $\min \lprof(y) = 2$, we get $3 ≤ \max \lprof(x) ≤ 3$ and $\min \lprof(x) ≤ \max \lprof(x) - 3$ thus only $\max \lprof(x) = 3$ and $\min \lprof(x) = 0$ is possible.
For $\max \lprof(y) = 4$: 1 class.

\begin{table}
	\begin{tabular}{CCCllCCCCCC}
		\toprule
		\text{id} & \scriptscriptstyle d(\lprof(x)) & \scriptscriptstyle d(\lprof(y)) & v(x) & v(y) & e & \scriptscriptstyle \text{top} & \scriptscriptstyle \Delta^\text{min} & \scriptscriptstyle \Delta^\text{avg} & \scriptscriptstyle \Delta^\text{ineq}&\text{cl}\\
		\midrule 
		\csvreader[late after line = \\]{Profiles.csv}%
		{dlPx = \dlpx, dlPy = \dlpy, max lPx = \maxlpx, max lPy = \maxlpy, min lPx = \minlpx, avg lPx = \avglPx, min lPy = \minlpy, avg lPy = \avglpy, e = \cole, top = \coltop, dmin = \dmin, davg = \davg, delta = \coldelta, cl coarse = \clcoarse, subclass = \subclass}{%
			\thecsvrow & \dlpx & \dlpy & (\minlpx, \maxlpx, \num[round-precision = 1]{\avglPx}) & (\minlpy, \maxlpy, \avglpy) & \cole & \coltop & \dmin & \davg & \coldelta & \clcoarse.\subclass
		}% 
		\bottomrule
	\end{tabular}
	\caption{Possible values for $m = 13$; $v(z) = (\min \lprof(z), \max \lprof(z), \frac{\sum \lprof(z)}{2})$; top indicates whether $\min \lprof(x) = 0$; $\Delta^\text{ineq} = d(\lprof(x)) - d(\lprof(y))$}
	\label{fig:m13}
\end{table}

\Cref{fig:m13} lists the coherent values for these variables. 
To identify narrow classes (rows in the table), consider $\Delta^\text{min}$ or $\Delta^\text{avg}$, $\delta$, $\min \lprof(x)$, $\max \lprof(x)$.
To identify wide classes (indicated in “cl”), consider top, $\Delta^\text{min}$ or $\Delta^\text{avg}$, and $\Delta^\text{ineq}$.

\subsection{Some profile classes (automatic)}
%Generated – please do not edit.

\begin{example}[$\lprof(x) = \{0, 6\}$; $\lprof(y) = \{5, 7\}$]
  \label{ex:0657}
  \begin{equation}
    \begin{array}{*{13}c}
      \bm{a}&b&c&d&e&f&g&\boxed{h}&i&j&k&l&m\\
      i&j&k&l&m&\boxed{h}&\bm{a}&b&c&d&e&f&g
    \end{array}
  \end{equation}
\end{example}

\begin{example}[$\lprof(x) = \{1, 6\}$; $\lprof(y) = \{5, 7\}$]
  \label{ex:1657}
  \begin{equation}
    \begin{array}{*{13}c}
      a&\bm{b}&c&d&e&f&g&\boxed{h}&i&j&k&l&m\\
      i&j&k&l&m&\boxed{h}&\bm{b}&a&c&d&e&f&g
    \end{array}
  \end{equation}
\end{example}

\begin{example}[$\lprof(x) = \{0, 5\}$; $\lprof(y) = \{4, 6\}$]
  \label{ex:0546}
  \begin{equation}
    \begin{array}{*{13}c}
      \bm{a}&b&c&d&e&f&\boxed{g}&h&i&j&k&l&m\\
      h&i&j&k&\boxed{g}&\bm{a}&l&m&b&c&d&e&f
    \end{array}
  \end{equation}
\end{example}

\begin{example}[$\lprof(x) = \{0, 6\}$; $\lprof(y) = \{4, 7\}$]
  \label{ex:0647}
  \begin{equation}
    \begin{array}{*{13}c}
      \bm{a}&b&c&d&e&f&g&\boxed{h}&i&j&k&l&m\\
      i&j&k&l&\boxed{h}&m&\bm{a}&b&c&d&e&f&g
    \end{array}
  \end{equation}
\end{example}

\begin{example}[$\lprof(x) = \{2, 6\}$; $\lprof(y) = \{5, 7\}$]
  \label{ex:2657}
  \begin{equation}
    \begin{array}{*{13}c}
      a&b&\bm{c}&d&e&f&g&\boxed{h}&i&j&k&l&m\\
      i&j&k&l&m&\boxed{h}&\bm{c}&a&b&d&e&f&g
    \end{array}
  \end{equation}
\end{example}

\begin{example}[$\lprof(x) = \{1, 5\}$; $\lprof(y) = \{4, 6\}$]
  \label{ex:1546}
  \begin{equation}
    \begin{array}{*{13}c}
      a&\bm{b}&c&d&e&f&\boxed{g}&h&i&j&k&l&m\\
      h&i&j&k&\boxed{g}&\bm{b}&l&m&a&c&d&e&f
    \end{array}
  \end{equation}
\end{example}

\begin{example}[$\lprof(x) = \{0, 4\}$; $\lprof(y) = \{3, 5\}$]
  \label{ex:0435}
  \begin{equation}
    \begin{array}{*{13}c}
      \bm{a}&b&c&d&e&\boxed{f}&g&h&i&j&k&l&m\\
      g&h&i&\boxed{f}&\bm{a}&j&k&l&m&b&c&d&e
    \end{array}
  \end{equation}
\end{example}

\begin{example}[$\lprof(x) = \{1, 6\}$; $\lprof(y) = \{4, 7\}$]
  \label{ex:1647}
  \begin{equation}
    \begin{array}{*{13}c}
      a&\bm{b}&c&d&e&f&g&\boxed{h}&i&j&k&l&m\\
      i&j&k&l&\boxed{h}&m&\bm{b}&a&c&d&e&f&g
    \end{array}
  \end{equation}
\end{example}

\begin{example}[$\lprof(x) = \{0, 5\}$; $\lprof(y) = \{3, 6\}$]
  \label{ex:0536}
  \begin{equation}
    \begin{array}{*{13}c}
      \bm{a}&b&c&d&e&f&\boxed{g}&h&i&j&k&l&m\\
      h&i&j&\boxed{g}&k&\bm{a}&l&m&b&c&d&e&f
    \end{array}
  \end{equation}
\end{example}

\begin{example}[$\lprof(x) = \{0, 6\}$; $\lprof(y) = \{3, 7\}$]
  \label{ex:0637}
  \begin{equation}
    \begin{array}{*{13}c}
      \bm{a}&b&c&d&e&f&g&\boxed{h}&i&j&k&l&m\\
      i&j&k&\boxed{h}&l&m&\bm{a}&b&c&d&e&f&g
    \end{array}
  \end{equation}
\end{example}

\begin{example}[$\lprof(x) = \{0, 5\}$; $\lprof(y) = \{4, 7\}$]
  \label{ex:0547}
  \begin{equation}
    \begin{array}{*{13}c}
      \bm{a}&b&c&d&e&f&g&\boxed{h}&i&j&k&l&m\\
      i&j&k&l&\boxed{h}&\bm{a}&m&b&c&d&e&f&g
    \end{array}
  \end{equation}
\end{example}

\begin{example}[$\lprof(x) = \{3, 6\}$; $\lprof(y) = \{5, 7\}$]
  \label{ex:3657}
  \begin{equation}
    \begin{array}{*{13}c}
      a&b&c&\bm{d}&e&f&g&\boxed{h}&i&j&k&l&m\\
      i&j&k&l&m&\boxed{h}&\bm{d}&a&b&c&e&f&g
    \end{array}
  \end{equation}
\end{example}

\begin{example}[$\lprof(x) = \{2, 5\}$; $\lprof(y) = \{4, 6\}$]
  \label{ex:2546}
  \begin{equation}
    \begin{array}{*{13}c}
      a&b&\bm{c}&d&e&f&\boxed{g}&h&i&j&k&l&m\\
      h&i&j&k&\boxed{g}&\bm{c}&l&m&a&b&d&e&f
    \end{array}
  \end{equation}
\end{example}

\begin{example}[$\lprof(x) = \{1, 4\}$; $\lprof(y) = \{3, 5\}$]
  \label{ex:1435}
  \begin{equation}
    \begin{array}{*{13}c}
      a&\bm{b}&c&d&e&\boxed{f}&g&h&i&j&k&l&m\\
      g&h&i&\boxed{f}&\bm{b}&j&k&l&m&a&c&d&e
    \end{array}
  \end{equation}
\end{example}

\begin{example}[$\lprof(x) = \{0, 3\}$; $\lprof(y) = \{2, 4\}$]
  \label{ex:0324}
  \begin{equation}
    \begin{array}{*{13}c}
      \bm{a}&b&c&d&\boxed{e}&f&g&h&i&j&k&l&m\\
      f&g&\boxed{e}&\bm{a}&h&i&j&k&l&m&b&c&d
    \end{array}
  \end{equation}
\end{example}

\begin{example}[$\lprof(x) = \{2, 6\}$; $\lprof(y) = \{4, 7\}$]
  \label{ex:2647}
  \begin{equation}
    \begin{array}{*{13}c}
      a&b&\bm{c}&d&e&f&g&\boxed{h}&i&j&k&l&m\\
      i&j&k&l&\boxed{h}&m&\bm{c}&a&b&d&e&f&g
    \end{array}
  \end{equation}
\end{example}

\begin{example}[$\lprof(x) = \{1, 5\}$; $\lprof(y) = \{3, 6\}$]
  \label{ex:1536}
  \begin{equation}
    \begin{array}{*{13}c}
      a&\bm{b}&c&d&e&f&\boxed{g}&h&i&j&k&l&m\\
      h&i&j&\boxed{g}&k&\bm{b}&l&m&a&c&d&e&f
    \end{array}
  \end{equation}
\end{example}

\begin{example}[$\lprof(x) = \{0, 4\}$; $\lprof(y) = \{2, 5\}$]
  \label{ex:0425}
  \begin{equation}
    \begin{array}{*{13}c}
      \bm{a}&b&c&d&e&\boxed{f}&g&h&i&j&k&l&m\\
      g&h&\boxed{f}&i&\bm{a}&j&k&l&m&b&c&d&e
    \end{array}
  \end{equation}
\end{example}

\begin{example}[$\lprof(x) = \{1, 6\}$; $\lprof(y) = \{3, 7\}$]
  \label{ex:1637}
  \begin{equation}
    \begin{array}{*{13}c}
      a&\bm{b}&c&d&e&f&g&\boxed{h}&i&j&k&l&m\\
      i&j&k&\boxed{h}&l&m&\bm{b}&a&c&d&e&f&g
    \end{array}
  \end{equation}
\end{example}

\begin{example}[$\lprof(x) = \{0, 5\}$; $\lprof(y) = \{2, 6\}$]
  \label{ex:0526}
  \begin{equation}
    \begin{array}{*{13}c}
      \bm{a}&b&c&d&e&f&\boxed{g}&h&i&j&k&l&m\\
      h&i&\boxed{g}&j&k&\bm{a}&l&m&b&c&d&e&f
    \end{array}
  \end{equation}
\end{example}

\begin{example}[$\lprof(x) = \{0, 6\}$; $\lprof(y) = \{2, 7\}$]
  \label{ex:0627}
  \begin{equation}
    \begin{array}{*{13}c}
      \bm{a}&b&c&d&e&f&g&\boxed{h}&i&j&k&l&m\\
      i&j&\boxed{h}&k&l&m&\bm{a}&b&c&d&e&f&g
    \end{array}
  \end{equation}
\end{example}

\begin{example}[$\lprof(x) = \{1, 5\}$; $\lprof(y) = \{4, 7\}$]
  \label{ex:1547}
  \begin{equation}
    \begin{array}{*{13}c}
      a&\bm{b}&c&d&e&f&g&\boxed{h}&i&j&k&l&m\\
      i&j&k&l&\boxed{h}&\bm{b}&m&a&c&d&e&f&g
    \end{array}
  \end{equation}
\end{example}

\begin{example}[$\lprof(x) = \{0, 4\}$; $\lprof(y) = \{3, 6\}$]
  \label{ex:0436}
  \begin{equation}
    \begin{array}{*{13}c}
      \bm{a}&b&c&d&e&f&\boxed{g}&h&i&j&k&l&m\\
      h&i&j&\boxed{g}&\bm{a}&k&l&m&b&c&d&e&f
    \end{array}
  \end{equation}
\end{example}

\begin{example}[$\lprof(x) = \{0, 5\}$; $\lprof(y) = \{3, 7\}$]
  \label{ex:0537}
  \begin{equation}
    \begin{array}{*{13}c}
      \bm{a}&b&c&d&e&f&g&\boxed{h}&i&j&k&l&m\\
      i&j&k&\boxed{h}&l&\bm{a}&m&b&c&d&e&f&g
    \end{array}
  \end{equation}
\end{example}

\begin{example}[$\lprof(x) = \{0, 5\}$; $\lprof(y) = \{4, 8\}$]
  \label{ex:0548}
  \begin{equation}
    \begin{array}{*{13}c}
      \bm{a}&b&c&d&e&f&g&h&\boxed{i}&j&k&l&m\\
      j&k&l&m&\boxed{i}&\bm{a}&b&c&d&e&f&g&h
    \end{array}
  \end{equation}
\end{example}


\subsection{Some profile classes (manual)}
\begin{example}[$\Delta^\text{min} = 2$; $\delta = 1$; $\lprof(x) = (0, 6)$ thus cl $1$]
	\label{ex1.1}
	\begin{equation}
		\begin{array}{*{13}l}
			\bm{a}	& b	& c	& d	& e	& f	& g	& h	& i & j & k & l & m\\
			& & h & & & & \bm{a}
		\end{array} 
	\end{equation}
\end{example}

\begin{example}[$\Delta^\text{min} = 2$; $\delta = 1$; $\lprof(x) = (0, 3)$ thus cl $1$]
	\label{ex:06}
	\begin{equation}
		\begin{array}{lllllllllllll}
			\bm{a}	& b	& c	& d	& e	& f	& g	& h	& i & j & k & l & m\\
			& & e & \bm{a}
		\end{array}
	\end{equation}
\end{example}

\begin{example}[$\Delta^\text{min} = 2$; $\delta = 1$; $\lprof(x) = (1, 6)$ thus cl $11$]
	\begin{equation}
		\begin{array}{lllllllllllll}
			a	& \bm{b}	& c	& d	& e	& f	& g	& h	& i & j & k & l & m\\
			& & & h & & & \bm{b}
		\end{array}
	\end{equation}
\end{example}

\begin{example}[$\Delta^\text{min} = 2$; $\delta = 1$; $\lprof(x) = (3, 6)$ thus cl $11$]
	\begin{equation}
		\begin{array}{lllllllllllll}
			a	& b	& c	& \bm{d}	& e	& f	& g	& h	& i & j & k & l & m\\
			& & & & & h & \bm{d}
		\end{array}
	\end{equation}
\end{example}

\begin{example}[spread $x$: 6, spread $y$: 3, $\max \lprof(y) = 7$]
	\begin{equation}
		\begin{array}{lllllllllllll}
			\bm{a}	& b	& c	& d	& e	& f	& g	& h	& i & j & k & l & m\\
			& & & & h & & \bm{a}
		\end{array}
	\end{equation}
	Put $\set{b, c, d, e, f, g}$ after $a$ (some constraints omitted).
\end{example}

\begin{example}[spread $x$: 6, spread $y$: 3, $\max \lprof(y) = 8$]
	\begin{equation}
		\begin{array}{lllllllllllll}
			\bm{a}	& b	& c	& d	& e	& f	& g	& h	& i & j & k & l & m\\
			& & & & & i & \bm{a}
		\end{array}
	\end{equation}
	Put $\set{b, c, d, e, f, g}$ after $a$ (some constraints omitted).
\end{example}

\begin{example}[spread $x$: 4, spread $y$: 2, $\rho = 6$]
	\begin{equation}
		\begin{array}{lllllllllllll}
			a	& b	& \bm{c}	& d	& e	& f	& g	& h	& i & j & k & l & m\\
			& & & & & h & \bm{c}
		\end{array}
	\end{equation}
\end{example}

\begin{example}[spread $x$: 4, spread $y$: 2, $\rho = 4$]
	\begin{equation}
		\begin{array}{lllllllllllll}
			\bm{a}	& b	& c	& d	& e	& f	& g	& h	& i & j & k & l & m\\
			& & & f & \bm{a} & & & & & l & m & j & k
		\end{array}
	\end{equation}
	The end is set to ease choice.
\end{example}

\begin{example}[spread $x$: 6, spread $y$: 2]
	\begin{equation}
		\begin{array}{lllllllllllll}
			\bm{a}	& b	& c	& d	& e	& f	& g	& h	& i & j & k & l & m\\
			& & & & & h & \bm{a}
		\end{array}
	\end{equation}
	Put $\set{b, c, d, e, f, g}$ after $a$, with $g$ not just after $a$.
\end{example}

\subsection{Classes}
Given a profile $P$ with a single FB winner and a single MS winner, let $\set{x} = FB(\prof)$ and $\set{y} = MS(\prof)$ denote the corresponding alternatives. One can associate to such a profile $\prof$ the four numbers $(\min \lprof(x), \max \lprof(x), \min \lprof(y), \max \lprof(y))$.
This determines the losses of the FB and the MS winner.
Assume that $\max \lprof(x) ≤ 9$ and $\max \lprof(y) ≤ 9$. In such a case, the tuple can be given simply as a four digits number, such as 0324 for $(\min \lprof(x) = 0, \max \lprof(x) = 3, \min \lprof(y) = 2, \max \lprof(y) = 4)$. 
Furthermore, once the losses of the FB and the MS winner are specified, there exists some systematic way of completing the profile to “ease choice” (described later), up to renaming the alternatives and voters. As a consequence, 

\section{Protocol}
\subsection{Questions}
Ceteris Paribus…
\begin{enumerate}
	\item Does high $\Delta$ favor MS?
	\begin{itemize}
%		\item $\delta = 4 ⇔ \Delta^\text{min} = 5$ and  $\delta = 4 ⇒ \Delta^\text{avg} = 3$.
		\item {}[0324 (15) or 0526 (20)] VS 0657 (1)
		\item or any other $\Delta^\text{ineq} = 1$ VS 0657
%		\item $\delta = 3 ⇔ 
	\end{itemize}
	\item Does it matter for the FB winner to be top-ranked by a voter? 
	\begin{itemize}
		\item 0324 (15) VS 1435 (14)
		\item 0526 (20) VS 1637 (19, not taken)
	\end{itemize}
	\item Does the best position of the FB winner matter?
	\begin{itemize}
		\item 0324 (15) VS 3657 (12)
	\end{itemize}
	\item Does $d(\lprof(y))$ matter, fixing $\Delta^\text{ineq}$?
	\begin{itemize}
		\item{} [0546 (3) or 1657 (2)] VS 0647 (4)
	\end{itemize}
	\item Does showing profiles in favor of MS first favor MS choices in other profiles?
	\item Does using the word “compromise” matter or using scenario  versus abstract presentation?
	\item Does $FB(P) = B(P)$ increase the propension to select the $FB$ winner?
	\item Do people select Borda rather than FB or MS?
\end{enumerate}
As a consequence, we pick the following set of seven profiles: $\set{0324 (15), 0526 (20), 0657 (1), \allowbreak 1435 (14), \allowbreak 3657 (12), \allowbreak 0546 (3), 0647 (4)}$.
Note that 3657 (12) has $B ≠ FB$ so it might point out a disagreement with our notion of compromise.
\commentOC{Is it possible to first check that subjects indeed generally pick one of our expected alternatives (i.e., the FB or MS winner), and ask subjects for a reason when their choice differs from this expectation?}

\subsection{Protocol}
We could run an abstract setting. We use the word compromise. (If we have enough money we can also investigate the abstract-non-compromise wording, though Olivier is a bit skeptical about the interest of this version.)
We also might compare two or three scenarios, one of which that describes a situation where envy matters, but we might still have to convince Ayça about this.
We do not pinpoint $x$ or $y$: subjects can choose any alternative.

Towards the end of the sequence, we use a seventh prime profile, to check for consistency.

We use two sequences. $S_1$ orders profiles from the most to the least MS-favoring profiles, and $S_2$ is $S_1$ reversed. How much a profile is MS-favoring is determined using these criteria, lexicographically.
\begin{itemize}
	\item high $\Delta^\text{ineq}$
	\item small $\Delta^\text{avg}$
	\item small $d(\lprof(y))$
	\item high $\min \lprof(x)$
\end{itemize}

We pick a random permutation to rename alternatives, for each choice we present.

We ask for how they choose, their rationale, for each choice we present. Subjects answer using free text and must provide some text before continuing.

\subsection{An example run}
%Generated – please do not edit.

\begin{equation}
  \begin{array}{*{13}c}
    b&d&j&f&k&g&h&i&m&e&a&c&l\\
    g&h&k&b&i&m&e&a&c&l&d&j&f
  \end{array}
\end{equation}
\begin{equation}
  \begin{array}{*{13}c}
    e&f&k&m&h&g&d&i&j&c&l&a&b\\
    i&j&d&c&l&e&a&b&f&k&m&h&g
  \end{array}
\end{equation}
\begin{equation}
  \begin{array}{*{13}c}
    f&b&i&k&c&g&a&d&h&l&m&e&j\\
    h&l&m&e&j&d&f&b&i&k&c&g&a
  \end{array}
\end{equation}
\begin{equation}
  \begin{array}{*{13}c}
    m&j&h&a&d&l&e&k&b&g&c&i&f\\
    e&k&b&l&j&g&c&i&f&m&h&a&d
  \end{array}
\end{equation}
\begin{equation}
  \begin{array}{*{13}c}
    i&f&l&j&e&a&c&k&g&h&b&m&d\\
    g&h&b&m&d&k&j&i&f&l&e&a&c
  \end{array}
\end{equation}
\begin{equation}
  \begin{array}{*{13}c}
    b&e&m&k&i&d&l&j&c&g&f&a&h\\
    j&c&g&f&l&b&a&h&e&m&k&i&d
  \end{array}
\end{equation}
\begin{equation}
  \begin{array}{*{13}c}
    a&i&e&f&d&h&l&j&g&b&m&k&c\\
    g&b&m&k&j&c&a&i&e&f&d&h&l
  \end{array}
\end{equation}
\begin{equation}
  \begin{array}{*{13}c}
    h&d&g&f&m&b&l&j&c&a&i&e&k\\
    j&c&l&a&i&h&e&k&d&g&f&m&b
  \end{array}
\end{equation}

\bibliography{biblio}

\appendix
\section{Formal statements}
\label{sec:proofs}
Define $\rho_\prof = \min_\allalts \max \lprof$.
(We often use simply $\rho$ when the profile considered is clear.)
Let ${\prefto_i}(z)$ designate the alternatives better than $z$ for voter $i$.

\subsection{Restraining profiles}
\begin{lemma}
	\label{th:maxNope}
	$[\max \set{y_1, y_2} - \min \set{y_1, y_2} ≤ x_2 - x_1]$ implies $[y_2 - y_1 ≤ x_2 - x_1]$.
\end{lemma}
\begin{proof}
	If $\max \set{y_1, y_2} = y_2$, this is immediate, and otherwise, $y_2 - y_1 ≤ y_1 - y_2$ and $y_1 - y_2 ≤ x_2 - x_1$ hence $y_2 - y_1 ≤ x_2 - x_1$.
\end{proof}

\begin{lemma}
	\label{th:eqToEq}
	$[y_2 - y_1 ≤ x_2 - x_1 \land x_1 ≤ y_1 ⇔ y_2 ≤ x_2]$ is equivalent to $[y_2 - y_1 ≤ x_2 - x_1 \land x_1 ≤ y_1 \land y_2 ≤ x_2]$ and equivalent to $[x_1 ≤ y_1 \land y_2 ≤ x_2]$.
\end{lemma}
\begin{proof}
	From the left hand side, $y_2 > x_2$ implies $x_1 > y_1$ hence $y_2 - y_1 > x_2 - x_1$, excluded; thus, $y_2 ≤ x_2$, hence, $x_1 ≤ y_1$. From the right hand side, $y_2 + x_1 ≤ x_2 + y_1$.
\end{proof}

\begin{lemma}
	\label{th:simpleImpl}
	$[y_2 - y_1 ≤ x_2 - x_1 \land x_1 < y_1 ⇒ y_2 < x_2]$ implies that $x_2 ≥ y_2$.
\end{lemma}
\begin{proof}
	If $x_1 < y_1$, done; otherwise, $x_1 ≥ y_1$, equivalently, $x_1 - y_1 ≥ 0$, and we know $x_2 - y_2 ≥ x_1 - y_1$ thus $x_2 - y_2 ≥ 0$ whence $x_2 ≥ y_2$.
\end{proof}

\begin{theorem}
	\label{th:equiv}
	The following propositions are equivalent:
	\begin{enumerate}
		\item \label{it:bigY} $\exists y \in \argmin_{\POP} (d \circ \lprof) \suchthat \max \lprof(y) > \rho$;
		\item \label{it:allBigY} $\forall y \in \argmin_{\POP} (d \circ \lprof): \max \lprof(y) > \rho$;
		\item \label{it:noInters} $\FBP \cap \argmin_{\POP} (d \circ \lprof) = \emptyset$;
		\item \label{it:notSubs} $\argmin_{\POP} (d \circ \lprof) \nsubseteq \FBP$.
	\end{enumerate}
	Furthermore, they imply:
	\begin{enumerate}[label=({\roman*}), ref={\roman*}]
		\item \label{it:card1} $\card{\FBP} = 1$;
		\item \label{it:dispMin} Letting $\FBP = \set{x}$, we have that $\forall \ibar \in \argmax \lprof(x): \argmin_{\POP} (d \circ \lprof) = \max \restr{{\pref_i}}{{\prefto_{\ibar}}(x)}$;
		\item \label{it:singMS} $\card{\argmin_{\POP} (d \circ \lprof) = 1}$;
		\item \label{it:pareto} Letting furthermore $\argmin_{\POP} (d \circ \lprof) = \set{y}$, we have that $\forall i \in \argmax \lprof(y): [x \pref_i y \land y \pref_{\ibar} x]$;
%		\item $\exists w \suchthat \min \lprof(w) = \rho < \max \lprof(w)$;
		\item \label{it:ordMinY} $\min \lprof(x) < \min \lprof(y)$;
		\item \label{it:singV} $\argmax \lprof(x) \cap \argmax \lprof(y) = \emptyset$ (in words: the voter that maximizes $\lprof(x)$ differs from the one that maximizes $\lprof(y)$);
		\item \label{it:singVcontrast} $\forall i \in N: [i \in \argmax \lprof(y) ⇔ \ibar \in \argmax \lprof(x)]$.
	\end{enumerate}
\end{theorem}
\begin{proof}
Pick any $y \in \argmin_{\POP} (d \circ \lprof)$ and $\ibar, x \in \FBP \suchthat \lprof(x)_{\ibar} = \rho$.

$x, y \in \POP$ thus $\lprof(y)_i < \lprof(x)_i ⇔ \lprof(y)_{\ibar} > \lprof(x)_{\ibar}$. 
Also, $y$ has smallest spread; formally, $\max \lprof(y) - \min \lprof(y) ≤ \lprof(x)_{\ibar} - \lprof(x)_i$.

Using \cref{th:maxNope,th:eqToEq}, these two claims imply that $\lprof(y)_{\ibar} ≤ \lprof(x)_{\ibar} = \rho \land \lprof(x)_i ≤ \lprof(y)_i$.

By definition of $\rho$, $\max \lprof(y) ≥ \rho$. We obtain $\lprof(y)_{\ibar} ≤ \lprof(x)_{\ibar} = \rho ≤ \max \lprof(y) \land \lprof(x)_i ≤ \lprof(y)_i$.

Pick any $y' \in \argmin_{\POP} (d \circ \lprof) \suchthat \max \lprof(y') > \rho$, and let us show that $y' \in \argmin_{\set{w \suchthat \lprof(w)_{\ibar} < \rho}} \lprof(w)_i$. 

Note that our conclusions so far concerning $y$ apply to $y'$.
By our assumption that $\rho < \max \lprof(y')$, $\max \lprof(y') = \lprof(y')_i$.
Pick any $w$ such that $\lprof(w)_{\ibar} < \rho$.
Because $y'$ is not Pareto-dominated by $w$, $\lprof(w)_{\ibar} < \lprof(y')_{\ibar} ⇒ \lprof(w)_i > \lprof(y')_i$. 
By definition of $\rho$, $\max \lprof(w) ≥ \rho$. Thus, $\lprof(w)_i ≥ \rho > \lprof(w)_{\ibar}$.
Also, $y'$ has smallest spread; formally, $\max \lprof(y') - \min \lprof(y') ≤ \lprof(w)_i - \lprof(w)_{\ibar}$.
\Cref{th:maxNope,th:simpleImpl} yield that $\lprof(w)_i ≥ \lprof(y')_i$, which proves that $\argmin_{\POP} (d \circ \lprof) \subseteq  \argmin_{\set{w \suchthat \lprof(w)_{\ibar} < \rho}} \lprof(w)_i$.

As $1 ≤ \card{\argmin_{\POP} (d \circ \lprof)} ≤ \card{\argmin_{\set{w \suchthat \lprof(w)_{\ibar} < \rho}} \lprof(w)_i} = 1$, this suffices to establish that $\ref{it:bigY} ⇒ \ref{it:card1}$ and $\ref{it:bigY} ⇒ \ref{it:dispMin}$.

We have shown that $\card{\argmin_{\POP} (d \circ \lprof)} = 1$, thus, \ref{it:bigY} ⇒ \ref{it:allBigY} and \ref{it:allBigY} ⇒ \ref{it:bigY}.

$\card{\argmin_{\POP} (d \circ \lprof)} = 1$ and we see from $\max \lprof(y') > \rho$ that $y' \notin \FBP$, therefore obtaining $\ref{it:bigY} ⇒ \ref{it:noInters}$. 

The implication $\ref{it:noInters} ⇒ \ref{it:notSubs}$ is immediate as $\argmin_{\POP} (d \circ \lprof) ≠ \emptyset$.

Starting now from $\ref{it:notSubs}$, and picking any $y \in \argmin_{\POP} (d \circ \lprof)$, we know $y \notin \FBP$; but this requires $\rho < \max \lprof(y)$, by definition of $\FB$. This establishes $\ref{it:notSubs} ⇒ \ref{it:bigY}$.

\Cref{it:singMS} follows from \cref{it:dispMin}.

Considering \cref{it:pareto},	we know that $\max \lprof(x) < \max \lprof(y)$ from \cref{it:bigY}. 
As $\min \lprof(x) ≤ \max \lprof(x)$, it follows that $x \pref_i y$.
As $y \in POP$, we cannot have $x \pref_{\ibar} y$, thus, $y \pref_{\ibar} x$.

Considering \cref{it:ordMinY}, by \cref{it:bigY}, $\max \lprof(x) < \max \lprof(y)$.
Note that $\min \lprof(y) ≤ \min \lprof(x)$ would lead to the ordering $\min \lprof(y) ≤ \min \lprof(x) ≤ \max \lprof(x) < \max \lprof(y)$, contradicting $(d \circ \lprof)(y) < (d \circ \lprof)(x)$, thus, $\min \lprof(x) < \min \lprof(y)$.

Considering \cref{it:singV}, assume that $\exists i \in \argmax \lprof(x) \cap \argmax \lprof(y)$. Thus, $\lprof(x)_i = \max \lprof(x)$ and $\max \lprof(y) = \lprof(y)_i$, whence $\lprof(y)_{\ibar} = \min \lprof(y)$ and $\min \lprof(x) = \lprof(x)_{\ibar}$.
	From \cref{it:pareto}, $y \pref_{\ibar} x$.
	Thus $\lprof(y)_{\ibar} = \min \lprof(y) < \min \lprof(x) = \lprof(x)_{\ibar}$, which contradicts $\min \lprof(y) > \min \lprof(x)$, and thus establishes that $\argmax \lprof(x) \cap \argmax \lprof(y) = \emptyset$.
	
	It follows from $\argmax \lprof(x) \cap \argmax \lprof(y) = \emptyset$ that $\forall i \in N: [i \in \argmax \lprof(y) ⇔ \ibar \in \argmax \lprof(x)]$.
\end{proof}

\begin{corollary}
	\label{th:rhoMin}
	If $\argmin_{\POP} (d \circ \lprof) \nsubseteq \FBP$:
	\begin{enumerate}
		\item $\card{\FBP} = 1$;
		\item \label{it:singVC} $\card{\argmax \lprof(x)} = \card{\argmax \lprof(y)} = 1$;
		\item \label{it:singVbis} $\argmax \lprof(x) \cap \argmax \lprof(y) = \emptyset$;
		\item \label{it:rhoMin} Letting $\FBP = \set{x}$, $\argmin_{\POP} (d \circ \lprof) = \set{y}$, $\argmax \lprof(x) = \set{\ibar}$, we have that $\forall w \succ_i y: x \succeq_{\ibar} w$;
		\item \label{it:order} Letting furthermore $\argmax \lprof(y) = \set{i}$ (thanks to \cref{it:singVbis}), we have that $\lprof(x)_i = \min \lprof(x) < \lprof(y)_{\ibar} = \min \lprof(y) < \lprof(x)_{\ibar} = \max \lprof(x) < \lprof(y)_i = \max \lprof(y)$.
	\end{enumerate}
\end{corollary}
\begin{proof}
	\Cref{it:singVC} follows from (and \cref{it:singVbis} repeats) \cref{th:equiv} \cref{it:singV}.
	\Cref{it:rhoMin} also follows from \cref{th:equiv} \cref{it:dispMin} which states that $y$ is the best-for-$i$ alternative among the better-than-$x$ for $\ibar$: contrapositively, any alternative better than $y$ for $i$ cannot be among the better than $x$ for $\ibar$.
	Turning to \cref{it:order}, by definition of $i$ and $\ibar$, $\lprof(x)_{\ibar} = \max \lprof(x)$ and $\lprof(y)_i = \max \lprof(y)$, thus, $\lprof(x)_i = \min \lprof(x)$ and $\lprof(y)_{\ibar} = \min \lprof(y)$; by \cref{th:equiv} \cref{it:bigY}, $\max \lprof(x) < \max \lprof(y)$; by \cref{th:equiv} \cref{it:pareto}, $\lprof(y)_{\ibar} < \lprof(x)_{\ibar}$; and by \cref{th:equiv} \cref{it:ordMinY}, $\min \lprof(x) < \min \lprof(y)$.
\end{proof}

\subsection{Bounds on losses}
\begin{theorem}
	$\forall \prof \in \allprofs \suchthat \argmin_{\POP} (d \circ \lprof) \nsubseteq \FBP$:
	\begin{enumerate}
		\item \label{it:minYUSum} $\min \lprof(y) ≤ m - 1 - \max \lprof(y)$;
		\item \label{it:maxXUMaxC} $\max \lprof(x) ≤ m - \max \lprof(y)$;
		\item \label{it:minXU} $\min \lprof(x) ≤ \max \lprof(x) - (\max \lprof(y) - \min \lprof(y) + 1)$;
		\item \label{it:maxYU} $\max \lprof(y) ≤ \frac{2 m - 2}{3}$;
		\item \label{it:maxXUM} $\max \lprof(x) ≤ \khalf$;
		\item \label{it:maxXL} $\max \lprof(y) - \min \lprof(y) + 1 ≤ \max \lprof(x)$;
		\item \label{it:minYL} $\max \lprof(y) - \khalf + 1 ≤ \min \lprof(y)$;
		\item \label{it:minYUM} $\min \lprof(y) ≤ \khalf - 1$;
		\item \label{it:maxXUMax} $\max \lprof(x) ≤ \max \lprof(y) - 1$;
		\item \label{it:maxXLMin} $\min \lprof(y) + 1 ≤ \max \lprof(x)$;
		\item \label{it:minYUMax} $\min \lprof(y) ≤ \max \lprof(y) - 2$;
		\item \label{it:minYL2} $2 ≤ \min \lprof(y)$;
		\item \label{it:maxYL} $4 ≤ \max \lprof(y)$;
		\item \label{it:minYLC} $2 \max \lprof(y) - m + 1 ≤ \min \lprof(y)$.
	\end{enumerate}
\end{theorem}
\begin{proof}
From \cref{th:rhoMin} \cref{it:singVC}, $\card{\argmax \lprof(x)} = \card{\argmax \lprof(y)} = 1$, thus we can define $\argmax \lprof(x) = \set{\ibar}$. Furthermore, $\argmax \lprof(x) \cap \argmax \lprof(y) = \emptyset$ as per \cref{th:rhoMin} \cref{it:singVbis}, thus, we can define $\argmax \lprof(y) = \set{i}$. 

Note that $\forall \prof \in \allprofs, z \in \POP: \sum \lprof(z) ≤ m - 1$.
In particular, $y \in PE$ requires $\sum \lprof(y) ≤ m - 1$, equivalently, \cref{it:minYUSum}.

From \cref{th:rhoMin} \cref{it:rhoMin}, $\forall w \succ_i y: x \succeq_{\ibar} w$.
Also, by definition of $\lprof$, $\card{{\prefto_i}(y)} = \lprof(y)_i$.
We obtain $\lprof(x)_{\ibar} ≤ m - \lprof(y)_i$, or equivalently by definition of $i$ and $\ibar$, \cref{it:maxXUMaxC}.

As $(d \circ \lprof)(y) < (d \circ \lprof)(x)$ (equivalently, $(d \circ \lprof)(y) + 1 ≤ (d \circ \lprof)(x)$), we obtain \cref{it:minXU}.

From \cref{it:minXU}, $\max \lprof(y) ≤ \max \lprof(x) - \min \lprof(x) + \min \lprof(y) - 1$, whence, using \cref{it:minYUSum,it:maxXUMaxC} and $- \min \lprof(x) ≤ 0$, $\max \lprof(y) ≤ m - \max \lprof(y) + m - 1 - \max \lprof(y) - 1$, equivalently, \cref{it:maxYU}.

Observe that $\exists z \in \allalts \suchthat \lprof(z)_i ≤ \khalf \land \lprof(z)_{\ibar} ≤ \khalf$. That’s because $\card \set{z \in \allalts \suchthat \lprof(z)_i ≤ \khalf} = \khalf + 1$ and there are only $\card \set{z \in \allalts \suchthat \lprof(z)_{\ibar} > \khalf} = m - (\khalf + 1) = \floor{\frac{m - 1}{2}} < \khalf + 1$ positions that do not satisfy $\lprof(z)_{\ibar} ≤ \khalf$.
It follows that $\argmin \max \lprof ≤ \khalf$, which establishes \cref{it:maxXUM}.

Combining \cref{it:minXU} with $0 ≤ \min \lprof(x)$ establishes \cref{it:maxXL}. 

Combining \cref{it:maxXUM,it:maxXL},
we get \cref{it:minYL}.
%Using $\min \lprof(x) ≤ \max \lprof(x) - (\max \lprof(y) - \min \lprof(y)) - 1$, equivalently, $\min \lprof(x) + \max \lprof(y) - \min \lprof(y) + 1 ≤ \max \lprof(x)$ together with $0 ≤ \min \lprof(x)$ yields $\max \lprof(y) - \min \lprof(y) + 1 ≤ \max \lprof(x)$.

Combining \cref{it:minYL} with $0 ≤ \max \lprof(y)$ establishes \cref{it:minYUM}. 
%Also, $\min \lprof(y) < \max \lprof(x)$ by ordering, equivalently $\min \lprof(y) ≤ \max \lprof(x) - 1$, and $\max \lprof(x) ≤ \khalf$, thus $\min \lprof(y) ≤ \khalf - 1$.

By \cref{th:rhoMin} \cref{it:order}, $\max \lprof(x) < \max \lprof(y)$, equivalently, \cref{it:maxXUMax}.

Similarly, by \cref{th:rhoMin} \cref{it:order}, $\min \lprof(y) < \max \lprof(x)$, equivalently, \cref{it:maxXLMin}.

Combining \cref{it:maxXUMax,it:maxXLMin} yields \cref{it:minYUMax}.

From \cref{it:maxXL,it:maxXUMax},
%Also, from $\max \lprof(y) - \min \lprof(y) + 1 ≤ \max \lprof(x) ≤ \max \lprof(y) - 1$, thus 
$\max \lprof(y) - \min \lprof(y) + 1 ≤ \max \lprof(y) - 1$, whence \cref{it:minYL2}.

Also, \cref{it:minYUMax,it:minYL2} lead to \cref{it:maxYL}.

Finally, \cref{it:maxXUMaxC,it:maxXL} lead to \cref{it:minYLC}.
\end{proof}

\begin{corollary}
	\label{th:boundsSummary}
	$\forall \prof \in \allprofs \suchthat \argmin_{\POP} (d \circ \lprof) \nsubseteq \FBP$:
	\begin{enumerate}
		\item $4 ≤ \max \lprof(y) ≤ \frac{2 m - 2}{3}$;
		\item $\max \set{2, \max \lprof(y) - \khalf + 1, 2 \max \lprof(y) - m + 1} ≤ \min \lprof(y) ≤ \min \set{m - 1 - \max \lprof(y), \max \lprof(y) - 2, \khalf - 1}$;%
		\footnote{Seems like $\max \lprof(y) - \khalf + 1 < 2 \max \lprof(y) - m + 1$ so it has been omitted. 
		Indeed, $ 4 ≤ \frac{2m - 2}{3}$ thus $7 ≤ m$ thus $m - \khalf ≤ \frac{2m - 2}{3}$ thus $1 - \khalf ≤ 1 + \frac{2m - 2}{3} - m$, and then what?}
		\item $\max \set{\min \lprof(y) + 1, \max \lprof(y) - \min \lprof(y) + 1} ≤ \max \lprof(x) ≤ \min \set{\khalf, \max \lprof(y) - 1, m - \max \lprof(y)}$;
		\item $\min \lprof(x) ≤ \max \lprof(x) - (\max \lprof(y) - \min \lprof(y) + 1)$.
	\end{enumerate}
\end{corollary}

\subsection{Classifying profiles}
We identify each class of profiles relevant to our experiment using four numbers (representing the losses of $x$ and $y$) that satisfy some feasibility constraints. 
Considering a set $t$ of $4$ values, given $k \in \intvl{1, 4}$, let $t_k$ denote the $k^\text{th}$ smallest value in $t$ (thus with $t_1 = \min t$ and $t_4 = \max t$).
Let $T \subseteq \set{t \subseteq \intvl{0, \khalf} \suchthat \card{t} = 4}$ denote the set of tuples of loss values that satisfy the bounds of \cref{th:boundsSummary}, considering that $t_1 = \min \lprof(x)$, $t_3 = \max \lprof(x)$, $t_2 = \min \lprof(y)$ and $t_4 = \max \lprof(y)$.

Given $t \in T$ and $i \in N$, define the class $C_{t, i} = \set{\prof \in \fprofs \suchthat \lprof(\FBP)(i, \ibar) = (t_1, t_3) \land \lprof(\MSP)(i, \ibar) = (t_4, t_2)}$ as the set of profiles where $\FBP$ has losses $((i, t_1), (\ibar, t_3))$ and $\MSP$ has losses $((i, t_4), (\ibar, t_2))$ (recall that $\FBP$ and $\MSP$ are singletons by definition of $\fprofs$). 
Define the class $C_t = \bigcup_{i \in N} C_{t, i}$ as the set of profiles where $x$ has losses $(t_1, t_3)$ and $y$ has losses $(t_2, t_4)$, displaying a loss $l$ as $(l(1), l(2))$ or $(l(2), l(1))$. 
The set of classes is $\mathscr{C} = \set{C_t \suchthat t \in T}$.

Recall that ${\prefto_i}(z)$ designate the alternatives better than $z$ for voter $i$.
\begin{theorem}
	\label{th:B}
	$\forall t \in T, i \in N$,
	$\exists \prof \in C_{t, i} \suchthat B(\prof) = \FBP$ (equivalently, $B(\prof) \subseteq \FBP$).
\end{theorem}
\begin{proof}[to be checked]
	Let us list here the constraints from \cref{th:boundsSummary} that $t$ satisfy (as $t \in T$), for convenience.
	\begin{enumerate}
		\item \label{it:t4} $4 ≤ t_4 ≤ \frac{2 m - 2}{3}$;
		\item \label{it:t2} $\max \set{2, t_4 - \khalf + 1, 2 t_4 - m + 1} ≤ t_2 ≤ \min \set{m - 1 - t_4, t_4 - 2, \khalf - 1}$;
		\item \label{it:t3} $\max \set{t_2 + 1, t_4 - t_2 + 1} ≤ t_3 ≤ \min \set{\khalf, t_4 - 1, m - t_4}$;
		\item \label{it:t1} $t_1 ≤ t_3 - (t_4 - t_2 + 1)$.
	\end{enumerate}
	
	Name the alternatives $a_1, …, a_m$.
	Define $x = a_{t_1 + 1}$ and $y = a_{t_4 + 1}$.
	Define sequences of alternatives 
	\begin{description}
		\item[$A_1$] $(a_{t_4 + 2}, …, a_{t_2 + t_4 + 1})$;
		\item[$A_2$] $(a_{t_2 + t_4 + 2}, …, a_{t_3 + t_4})$ (with $A_2 = \emptyset$ iff $t_3 + t_4 < t_2 + t_4 + 2$);
		\item[$A_3$] $(a_{t_3 + t_4 + 1}, …, a_m)$;
		\item[$A_4$] $(a_1, …, a_{t_1})$;
		\item[$A_5$] $(a_{t_1 + 2}, …, a_{t_4})$.
	\end{description}
	Given a sequence $A$, define $A^{-1}$ as the inverse of $A$.
	Define $\prof$ as
	\begin{description}
		\item[$i$] $(a_1, …, a_m) = (A_4, x, A_5, y, A_1, A_2, A_3)$;
		\item[$\ibar$] $(A_1^{-1}, y, A_2, x, A_3, A_5, A_4^{-1})$.
	\end{description}
	Note that $\prof(\ibar)$ is indeed a linear order on $\allalts$, equivalently, lists each alternative of $m$ exactly once, as it is a permutation of the sequence $(A_4, x, A_5, y, A_1, A_2, A_3)$ which lists each alternative of $m$ exactly once. The latter fact holds because $t_2 + t_4 + 1 ≤ t_3 + t_4$, equivalently, $t_2 + 1 ≤ t_3$, from \cref{it:t3}.
	
	Observe that $t_1 = \card{{\prefto_i}(x)} = \lprof(x)_i$ and $t_3 = \card{(A_1 \cup \set{y} \cup A_2)} = \card{{\prefto_{\ibar}}(x)} = \lprof(x)_{\ibar}$. It follows that $\min \lprof(x) = t_1$ and $\max \lprof(x) = t_3$.
	Similarly, $t_4 = \card{{\prefto_i}(y)} = \lprof(y)_i$ and $t_2 = \card{A_1} = \card{{\prefto_{\ibar}}(y)} = \lprof(y)_{\ibar}$. It follows that $\min \lprof(y) = t_2$ and $\max \lprof(y) = t_4$.

	To see that $\FBP = \argmin \max \lprof = \set{x}$, observe that $\max \lprof(x) = t_3$ and let us show that $\forall z \in \allalts \setminus \set{x}: \max \lprof(z) > t_3$. 
	Indeed,
	$\forall z \in \set{y} \cup A_1 \cup A_2 \cup A_3, t_3 < t_4 = \lprof(y)_i < \lprof(z)_i ≤ \max \lprof(z)$, and
	$\forall z \in \set{A_4 \cup A_5}, t_3 = \lprof(x)_{\ibar} < \lprof(z)_{\ibar} ≤ \max \lprof(z)$.
	
	To show that $\MSP = \argmin_{\PO} (d \circ \lprof) = \set{y}$, observe that $y \in \PO$, $(d \circ \lprof)(y) = t_4 - t_2$, and let us show that $\forall z \in \allalts \setminus \set{y}: (d \circ \lprof)(z) > t_4 - t_2$.
	Indeed,
	$\forall z \in A_3$, $d(\lprof(z)) = \max \lprof(z) - \min \lprof(z) = \card{(A_4 \cup \set{x} \cup A_5 \cup \set{y} \cup A_1 \cup A_2)} - \card{(A_1 \cup \set{y} \cup A_2 \cup \set{x})} = \card{(A_4 \cup A_5)} = t_4 - 1 > t_4 - t_2$ (using $t_2 ≥ 2$ from \cref{it:t2}).
	Similarly, $\forall z \in A_2$, $d(\lprof(z)) > \card{(A_4 \cup A_5)} > t_4 - t_2$.
	$\forall z \in A_1$, $\lprof(z)_{\ibar} < \lprof(y)_{\ibar} < \lprof(y)_i < \lprof(z)_i$ thus $d(\lprof(z)) > d(\lprof(y)) = t_4 - t_2$.
	$\forall z \in A_5$, $d(\lprof(z)) = \max \lprof(z) - \min \lprof(z) = \abs{\card{(A_1 \cup \set{y} \cup A_2 \cup \set{x} \cup A_3)} - \card{(A_4 \cup \set{x})}} = \abs{\card{(A_1 \cup \set{y} \cup A_2 \cup A_3)} - \card{A_4}} = \abs{m - t_4 - t_1}$.
	From \cref{it:t1,it:t3}, $t_1 ≤ t_3 - t_4 + t_2 - 1$ and $t_3 ≤ m - t_4$, thus $t_1 ≤ m - 2 t_4 + t_2 - 1 < m - 2 t_4 + t_2$. Thus, $m - t_4 - t_1 > t_4 - t_2$ (and $m - t_4 - t_1 > 0$).
	Finally, $\forall z \in A_4$, $d(\lprof(z)) = \card{(A_1 \cup \set{y} \cup A_2 \cup \set{x} \cup A_3 \cup A_5)} - \card{A_4} > m - t_4 - t_1 > t_4 - t_2$.
	
	To see that $B(\prof) = \argmin \sum \lprof = \set{x}$, observe that $\sum \lprof(x) = t_1 + t_3$ and let us show that $\forall z \in \allalts \setminus \set{x}: \sum \lprof(z) > t_1 + t_3$. 
	Indeed,
	$\forall z \in A_1$, $\sum \lprof(z) = \card{(A_4 \cup \set{x} \cup A_5 \cup \set{y} \cup A_1 \setminus \set{z})} = t_4 + 1 + t_2 - 1 = t_2 + t_4 > t_1 + t_3$.
	Similarly, $\forall z \in \set{y} \cup A_2 \cup A_3$, $\sum \lprof(z) ≥ \sum \lprof(y) = \card{(A_4 \cup \set{x} \cup A_5 \cup A_1)} = t_2 + t_4 > t_1 + t_3$.
	$\forall z \in A_4$, $\sum \lprof(z) ≥ \card{(A_1 \cup \set{y} \cup A_2 \cup \set{x} \cup A_3 \cup A_5 \cup A_4 \setminus \set{z})} = \card{\allalts \setminus \set{z}} = m - 1 > t_1 + t_3$ (using $\khalf ≥ t_3 > t_1$ from \cref{it:t3}).

	Finally, we see that $\prof \in \fprofs$ as it has already been established that $\forall z \in \allalts \setminus \set{y}: (d \circ \lprof)(z) > (d \circ \lprof)(y)$.
\end{proof}

\begin{example}
Here is an example construction of a profile as in the proof of \cref{th:B}, with $m = 13$ and $t_1 = \lprof(x)_i = 1$; $t_2 = \lprof(y)_{\ibar} = 4$; $t_3 = \lprof(x)_{\ibar} = 5$; $t_4 = \lprof(y)_i = 6$; thus $\sum \lprof(y) = 10$.
	\begin{description}
		\item[$A_1$] $(a_8, …, a_{11}) = (h, i, j, k)$;
		\item[$A_2$] $(a_{12}, …, a_{11}) = \emptyset$;
		\item[$A_3$] $(a_{12}, …, a_{13}) = (l, m)$;
		\item[$A_4$] $(a_1, …, a_1) = (a)$;
		\item[$A_5$] $(a_3, …, a_6) = (c, d, e, f)$.
	\end{description}
  \begin{equation}
    \begin{array}{*{13}c}
      a&\bm{b}&c&d&e&f&\boxed{g}&h&i&j&k&l&m\\
      k&j&i&h&\boxed{g}&\bm{b}&l&m&a&c&d&e&f
    \end{array}
  \end{equation}
\end{example}

\subsection{Borda}
\begin{theorem}
	$\forall \prof \in \allprofs: [\exists x \in FB(P) \suchthat \min \lprof(x) = 0] ⇒ B(P) \subseteq FB(P)$.
\end{theorem}
\begin{proof}
	Consider any $z \in B(P)$. 
	Observe that $\sum \lprof(z) ≤ \sum \lprof(x)$, thus, $\max \lprof(z) ≤ \max \lprof(z) + \min \lprof(z) = \sum \lprof(z) ≤ \sum \lprof(x) = \max \lprof(x)$.
	It follows that $z \in FB(P)$.
\end{proof}
\begin{theorem}
	$\forall \prof \in \allprofs: \MSP \nsubseteq \FBP ⇒ \MSP \cap B(P) = \emptyset$.
\end{theorem}
\begin{proof}[to be checked]
	Consider any $y \in \MSP$ and let us show that $y \notin \argmin \sum \lprof = B(\prof)$.
	Pick any $x \in \FBP$ (thus $\max \lprof(x) = \min \max \lprof$).
	
	As $\MSP \nsubseteq \FBP$, $y \notin \FBP$, thus $\max \lprof(x) = \min \max \lprof < \max \lprof(y)$.
	
	We furthermore deduce that $\min \lprof(y) ≥ \min \lprof(x)$, otherwise, $\min \lprof(y) < \min \lprof(x) ≤ \max \lprof(x) = \min \max \lprof < \max \lprof(y)$ hence $y \notin \MSP = \argmin_{\PO}(d \circ \lprof)$ (using $x \in \PO$ from $x \in \FBP$).
	
	It follows that $\sum \lprof(x) < \sum \lprof(y)$.
\end{proof}

\begin{conjecture}
	\label{th:notB}
	$\forall t \in T, i \in N$:
	$t_1 ≥ 1 ⇒ \exists \prof \in C_{t, i} \suchthat B(\prof) \cap \FBP = \emptyset$.
\end{conjecture}
\begin{proof}[to be checked]
	The proof goes as for \cref{th:B} with the following differences.
	
	Define $\prof'$ as
	\begin{description}
		\item[$i$] $(a_1, …, a_m) = (A_4, x, A_5, y, A_1, A_2, A_3)$;
		\item[$\ibar$] $(A_1^{-1}, y, A_2, x, A_4, A_3, A_5)$.
	\end{description}

	Indeed,
	$\forall z \in A_3$, $d(\lprof'(z)) = \max \lprof(z) - \min \lprof(z) = \card{(A_4 \cup \set{x} \cup A_5 \cup \set{y} \cup A_1 \cup A_2)} - \card{(A_1 \cup \set{y} \cup A_2 \cup \set{x} \cup A_4)} = \card{A_5} = t_4 - t_1 - 1 > t_4 - t_2$ (using $t_1 ≤ t_2 + t_3 - t_4 - 1$ from \cref{it:t1} and $t_3 ≤ t_4 - 1$ from \cref{it:t3} hence $t_1 + 1 ≤ t_2 - 1 < t_2$).
	$\forall z \in A_5$, $d(\lprof'(z)) = d(\lprof(z)) + \card{A_4} > t_4 - t_2$.
	Finally, $\forall z \in A_4$, $d(\lprof'(z)) = \card{(A_1 \cup \set{y} \cup A_2 \cup \set{x})} = t_3 + 1 ≥ t_4 - t_2 + 2 > t_4 - t_2$ (using \cref{it:t3}).
	
	About Borda.
	That $A_4$ and $x$ are the only candidates is established as in the previous proof. Thus, $a_1$ and $x$ are the only candidates. Note that $a_1 ≠ x ⇔ t_1 ≥ 1$.
	We see that $\sum \lprof'(a_1) - \sum \lprof'(x) = (\lprof'(a_1)_i - \lprof'(x)_i) + (\lprof'(a_1)_{\ibar} - \lprof'(x)_{\ibar}) = - \card{A_4} + 1$.
	Thus, if $t_1 = 1$, $B(\prof') = \set{a_1, x}$ and if $t_1 ≥ 2$, $B(\prof') = \set{a_1}$.
\end{proof}

\subsection{Draft}
Consider any profile $\prof \in \allprofs$. Pick any $x \in \FBP$. Pick any $i \in N$ such that $\lprofi[x] = \min\lprof(x)$ and define $\ibar$ as the other individual, thus with $\lprofibar[x] = \max\lprof(x) = \rho$. Define $z$ as the alternative such that $\lprofi[z] = \rho$.

\begin{theorem}
	\label{th:fourfacts}
	The following four facts hold for any value of $m$.
	\begin{enumerate}
		\item $\rho ≤ \frac{m}{2}$.
		\item If $\rho = \frac{m}{2}$, then $\FBP = \set{x, z}$. 
		\item If $\FBP = \set{x, z}$, then $\minineq[\FBP] = \minineq[\POP]$. 
		\item If $\minineq[\FBP] ≤ 2$, then $\minineq[\FBP] = \minineq[\POP]$.
	\end{enumerate}
\end{theorem}
\begin{proof}
 	\emph{First fact}
 	
	Define $S_1 = \set{y ≠ x \suchthat 0 ≤ \lprofi[y] < \rho}$ and $S_2 = \set{y \suchthat \rho < \lprofibar[y] ≤ m - 1}$.
 	By definition of $\rho$, $\nexists y \suchthat \max\lprof(y) < \rho$,
	thus, $\forall y: [\lprofi[y] < \rho \land \lprofibar[y] ≠ \rho] ⇒ \rho < \lprofibar[y]$.
	Because $x ≠ y ⇒ \lprofibar[y] ≠ \rho$ (as $\lprofibar[x] = \rho$), we obtain that $S_1 \subseteq S_2$.
	Because $\card{\set{y \suchthat 0 ≤ \lprofi[y] < \rho}} = \rho$, $\card{S_1} ≥ \rho - 1$.
	It follows that $\card{S_2} ≥ \rho - 1$.
	Also, $\card{S_2} = m - 1 - \rho$. We obtain $m - 1 - \rho ≥ \rho - 1$, thus $\rho ≤ \frac{m}{2}$.
	
	\emph{Second fact}
	
	Pursuing with $S_1$ and $S_2$ as defined above, and assuming further that $\rho = \frac{m}{2}$, or equivalently $m - \rho = \rho$, we see that $S_1$ and $S_2$ have the same cardinalities, hence, are equal. By definition of $z$, $z \notin S_1$. Thus, $z \notin S_2$. Therefore, $\lprofibar[z] ≤ \rho$, and as $\lprofi[z] = \rho$, $z \in \FBP$. As $\FBP = \set{y \suchthat \max\lprof(y) = \rho}$, $y \in \FBP$ requires that $\lprofi[y] = \rho$ or $\lprofibar[y] = \rho$, thus, no alternative but $x$ and $z$ may be in $\FBP$.
	
	\emph{Third fact}
	
	Assuming now that $\FBP = \set{x, z}$, and picking any $y \in \POP$, let us show that $\ineq(y) ≥ \minineq[\FBP]$.
	By hypothesis, $\lprofibar[z] ≤ \rho$ and $\lprofi[x] ≤ \rho$.
	Now if $\rho < \lprofi[y]$, then $\lprofibar[y] < \lprofibar[z]$ (otherwise $z$ Pareto dominates $y$, as $\lprofi[z] = \rho$), whence $\ineq(y) > \ineq(z)$ as $\lprofibar[y] < \lprofibar[z] ≤ \lprofi[z] = \rho < \lprofi[y]$.
	Similarly, assuming that $\rho < \lprofibar[y]$ yields that $\ineq(y) > \ineq(x)$.
	The only remaining possibility is that $\lprofi[y] ≤ \rho$ and $\lprofibar[y] ≤ \rho$, in which case $y \in \FBP$.
	
	\emph{Fourth fact}
	
	Assuming that $\minineq[\FBP] ≤ 2$, and picking any $y \in \POP$, let us show that $\ineq(y) ≥ \minineq[\FBP]$.
	If $\rho < \lprofi[y]$, then $\lprofi[x] < \lprofi[y]$ and $\lprofibar[x] < \lprofi[y]$, thus $\lprofibar[y] < \lprofibar[x]$ (otherwise $x$ Pareto dominates $y$), thus $\lprofibar[y] < \lprofibar[x] < \lprofi[y]$, thus $\ineq(y) ≥ 2$.
	If $\rho < \lprofibar[y]$, an identical reasoning, exchanging $i$ and $\ibar$, concludes identically.
	Otherwise, $y \in \FBP$.
 \end{proof}

\begin{corollary}
	\label{th:morethan6}
 	For $m ≤ 6$, FB satisfies PCC for the inequality relation $d$.
\end{corollary}
\begin{proof}
	Using the first fact of \cref{th:fourfacts}, $\rho ≤ 3$, and suffices then to use either, if $\rho = 3$, the second and the third facts, or otherwise, the fourth fact (because $\minineq[\FBP] ≤ \rho$) to prove as required that some alternative in $\FBP$ reaches the required minimal inequality over the Pareto optimal alternatives.
\end{proof}

\begin{corollary}
	\label{th:conds}
	The profiles such that $\minineq[\FBP] ≠ \minineq[\POP]$ must satisfy all the following conditions. Let $x \in \FBP$.
	\begin{itemize}
		\item $\card{\FBP} = 1$.
		\item $3 ≤ \ineq(x) ≤ \rho < \frac{m}{2}$ (whence $m ≥ 7$).
		\item $2 ≤ \minineq[\POP] < \ineq(x)$.
	\end{itemize}
\end{corollary}
\begin{proof}
	We see that these conditions are required using \cref{th:fourfacts} and \cref{th:morethan6}: $\FBP$ must be a singleton because of the third fact; $3 ≤ \ineq(x)$ because of the fourth fact; and $\rho < \frac{m}{2}$ because of the first and second facts.
\end{proof}

\section{Examples with m = 13}
Define $\epsilon = \max \lprof(y) - \max \lprof(x)$. We have $\epsilon ≤ \minineq[\POP] - 1$.

When $\rho = 6$, to ensure $\FBP = \set{x}$, must put $\set{a, b, c, d, e, f, g} \setminus \set{x}$ after $x$, hence, every alternative among $\set{h, i, j, k, l, m}$ must be better than $x$ for $2$, thus, $\argmin_{\POP} (d \circ \lprof) = h$, whatever $\minineq[\POP]$: any alternative among $\set{i, j, k, l, m}$ in between $h$ and $x$ for $2$ is Pareto-dominated by $h$ and any alternative among $\set{i, j, k, l, m}$ better than $h$ for $2$ has a greater spread.

\begin{example}
	$m = 13$, $\minineq[\FBP] = 6$, $\minineq[\POP] = 2$, hence $\epsilon = 1$.
	\begin{equation}
		\begin{array}{lllllllllllll}
			\bm{a}	& b	& c	& d	& e	& f	& g	& h	& i & j & k & l & m\\
			& & & & & h & \bm{a}
		\end{array}
	\end{equation}
	Put $\set{b, c, d, e, f, g}$ after $a$, with $g$ not just after $a$.
\end{example}

\begin{example}
	$m = 13$, $\minineq[\FBP] = 6$, $\minineq[\POP] = 3$, hence $\epsilon = 1$.
	\begin{equation}
		\begin{array}{lllllllllllll}
			\bm{a}	& b	& c	& d	& e	& f	& g	& h	& i & j & k & l & m\\
			& & & & h & & \bm{a}
		\end{array}
	\end{equation}
	Put $\set{b, c, d, e, f, g}$ after $a$ (some constraints omitted).
\end{example}

\begin{example}
	$m = 13$, $\minineq[\FBP] = 6$, $\minineq[\POP] = 4$, hence $\epsilon = 1$.
	\begin{equation}
		\begin{array}{lllllllllllll}
			\bm{a}	& b	& c	& d	& e	& f	& g	& h	& i & j & k & l & m\\
			& & & h & & & \bm{a}
		\end{array}
	\end{equation}
	Put $\set{b, c, d, e, f, g}$ after $a$ (some constraints omitted).
\end{example}

\begin{example}
	$m = 13$, $\minineq[\FBP] = 6$, $\minineq[\POP] = 5$, hence $\epsilon = 1$.
	\begin{equation}
		\begin{array}{lllllllllllll}
			\bm{a}	& b	& c	& d	& e	& f	& g	& h	& i & j & k & l & m\\
			& & h & & & & \bm{a}
		\end{array}
	\end{equation}
	Put $\set{b, c, d, e, f, g}$ after $a$ (some constraints omitted).
\end{example}

\begin{example}
	$m = 13$, $\minineq[\FBP] = 5$, $\minineq[\POP] = 2$, $\rho = 6$, hence $\epsilon = 1$.
	\begin{equation}
		\begin{array}{lllllllllllll}
			a	& \bm{b}	& c	& d	& e	& f	& g	& h	& i & j & k & l & m\\
			& & & & & h & \bm{b}
		\end{array}
	\end{equation}
	Put $\set{a, c, d, e, f, g}$ after $b$ (some constraints omitted).
\end{example}

\begin{example}
	$m = 13$, $\minineq[\FBP] = 5$, $\minineq[\POP] = 2$, $\rho = 5$, hence $\epsilon = 1$.
	\begin{equation}
		\begin{array}{lllllllllllll}
			\bm{a}	& b	& c	& d	& e	& f	& g	& h	& i & j & k & l & m\\
			& & & & g & \bm{a}
		\end{array}
	\end{equation}
	Put $\set{b, c, d, e, f}$ after $a$ (some constraints omitted).
\end{example}

\begin{example}
	$m = 13$, $\minineq[\FBP] = 5$, $\minineq[\POP] = 3$, $\rho = 6$, hence $\epsilon = 1$.
	\begin{equation}
		\begin{array}{lllllllllllll}
			a	& \bm{b}	& c	& d	& e	& f	& g	& h	& i & j & k & l & m\\
			& & & & h & & \bm{b}
		\end{array}
	\end{equation}
	Put $\set{a, c, d, e, f, g}$ after $b$ (some constraints omitted).
\end{example}

\begin{example}
	$m = 13$, $\minineq[\FBP] = 5$, $\minineq[\POP] = 3$, $\rho = 5$, $\epsilon = 1$.
	\begin{equation}
		\begin{array}{lllllllllllll}
			\bm{a}	& b	& c	& d	& e	& f	& g	& h	& i & j & k & l & m\\
			& & & g & & \bm{a}
		\end{array}
	\end{equation}
	Put $\set{b, c, d, e, f}$ after $a$ (some constraints omitted).
\end{example}

\begin{example}
	$m = 13$, $\minineq[\FBP] = 5$, $\minineq[\POP] = 3$, $\rho = 5$, $\epsilon = 2$.
	\begin{equation}
		\begin{array}{lllllllllllll}
			\bm{a}	& b	& c	& d	& e	& f	& g	& h	& i & j & k & l & m\\
			& & & & h & \bm{a}
		\end{array}
	\end{equation}
	Put $\set{b, c, d, e, f, g}$ after $a$ (some constraints omitted).
\end{example}

\begin{example}
	$m = 13$, $\minineq[\FBP] = 4$, $\minineq[\POP] = 2$, $\rho = 6$, $\epsilon = 1$.
	\begin{equation}
		\begin{array}{lllllllllllll}
			a	& b	& \bm{c}	& d	& e	& f	& g	& h	& i & j & k & l & m\\
			& & & & & h & \bm{c}
		\end{array}
	\end{equation}
\end{example}

\section{Examples with m = 9 and m = 11}
For $m = 9$, some of the constraints are: $\minineq[\FBP] \in \intvl{3, 4}$, $\rho \in \intvl{3, 4}$ and $\minineq[\POP] \in \intvl{2, 3}$.

The following illustrates the general construction process. 
\begin{example}
	Example for $m = 9$, $\ineq(x) = \rho = 4$, $\minineq[\POP] = 3$.
	\begin{equation}
		\begin{array}{lllllllll}
			a	& b	& c	& d	& e	& f	& g	& h	& i\\
			\scriptscriptstyle{h} & \scriptscriptstyle{i} & f & \scriptscriptstyle{g} & a & \scriptscriptstyle{b} & \scriptscriptstyle{c} & \scriptscriptstyle{d} & \scriptscriptstyle{e}
		\end{array}
	\end{equation}

	To complete the preference of the second individual so that $\FBP = \set{a}$ and the Paretian minimizers of dispersion is $f$, just put $\set{b, c, d, e}$ worst than $a$ and $\set{g, h, i}$ in the remaining places, in any order.

	Note that putting $e$ just after $a$ might actually be significantly different than other permutations of the set $\set{b, c, d, e}$.
\end{example}

\begin{example}
	$m = 9$, $\minineq[\FBP] = 3$, $\rho = 3$, $\minineq[\POP] = 2$.
	\begin{equation}
		\begin{array}{lllllllll}
			\bm{a}	& b	& c	& d	& e	& f	& g	& h	& i\\
			& & e & \bm{a}
		\end{array}
	\end{equation}
	Put $\set{b, c, d}$ after $a$.
\end{example}

\begin{example}
	$m = 9$, $\minineq[\FBP] = 3$, $\rho = 4$, $\minineq[\POP] = 2$.
	\begin{equation}
		\begin{array}{lllllllll}
			a	& \bm{b}	& c	& d	& e	& f	& g	& h	& i\\
			& & & f & \bm{b}
		\end{array}
	\end{equation}
	Put $\set{a, c, d, e}$ after $b$.
\end{example}

\begin{example}
	$m = 9$, $\minineq[\FBP] = 4$ (hence $\rho = 4$), $\minineq[\POP] = 2$.
	\begin{equation}
		\begin{array}{lllllllll}
			\bm{a}	& b	& c	& d	& e	& f	& g	& h	& i\\
			& & & f & \bm{a}
		\end{array}
	\end{equation}
	Put $\set{b, c, d, e}$ after $a$.
\end{example}

\begin{example}
	$m = 9$, $\minineq[\FBP] = 4$ (hence $\rho = 4$), $\minineq[\POP] = 3$.
	\begin{equation}
		\begin{array}{lllllllll}
			\bm{a}	& b	& c	& d	& e	& f	& g	& h	& i\\
			& & f & & \bm{a}
		\end{array}
	\end{equation}
	Put $\set{b, c, d, e}$ after $a$.
\end{example}

\begin{example}
	$m = 11$, $\minineq[\FBP] = 4$, $\rho = 4$, $\minineq[\POP] = 3$.
	\begin{equation}
		\begin{array}{lllllllllll}
			\bm{a}	& b	& c	& d	& e	& f	& g	& h	& i & j & k\\
			& & & g & \bm{a}
		\end{array}
	\end{equation}
	Put $\set{b, c, d, e, f}$ after $a$.
\end{example}

\begin{example}
	$m = 11$, $\minineq[\FBP] = 4$, $\rho = 4$, $\minineq[\POP] = 3$, alternative.
	\begin{equation}
		\begin{array}{lllllllllll}
			\bm{a}	& b	& c	& d	& e	& f	& g	& h	& i & j & k\\
			& & f & & \bm{a}
		\end{array}
	\end{equation}
	Put $\set{b, c, d, e}$ after $a$.
\end{example}

\begin{example}
	$m = 11$, $\minineq[\FBP] = 5$, $\minineq[\POP] = 2$.
	\begin{equation}
		\begin{array}{lllllllllll}
			\bm{a}	& b	& c	& d	& e	& f	& g	& h	& i & j & k\\
			& & & & g & \bm{a}
		\end{array}
	\end{equation}
	Put $\set{b, c, d, e, f}$ after $a$.
\end{example}

\section{Which \acp{SCR} are compromises?}
Let $\sigma: \alllosses → \R^+$ be an inequality measure, minimal exactly for constant loss vectors: $\forall l \in \alllosses: [\sigma(l) = 0] ⇔ [l_1 = l_2]$.
Let $\Sigma$ denote the set of functions satisfying this condition. 

Let $\argmin_{\POP} (\sigma \circ \lprof) = \set{x \in \POP \suchthat (\sigma \circ \lprof)(x) = \min_{y \in \POP} (\sigma \circ \lprof)(y)}$ denote the least unequal alternatives among the Pareto-optimal ones, according to the measure $\sigma \circ \lprof$.
Given a SCR $f$, say that $f$ is Pareto Compromise Compatible (PCC) iff $\exists \sigma \in \Sigma \suchthat \forall \prof \in \allprofs: f(\prof) \cap \argmin_{\POP} (\sigma \circ \lprof) ≠ \emptyset$.

We consider in particular the inequality measure $d \in \Sigma$ defined as the absolute difference between the losses. 

Below: copied from our other paper.

\label{sec:more2voters}
In this section we assume $n\geq 3$ and leave the analysis of $n=2$ to the
next section.

\subsection{BK-compromises}
\label{sec:BKn3}
Given any $k\in \intvl{1, m}$, we write $n_{k}(x,P)=\#\{i\in
N\mid r_{\prefi}(x)\leq k\}$ for the \emph{$k$-support} that $x$ gets at $P$, that is, the number of individuals for whom the rank of alternative $x\in A$ is lower than or equal to $k$ in the profile $P\in $ $L(A)^{N}$.
Note that $n_{k}(x,P)\in \intvl{1, n}$ is non-decreasing on $k$ and $n_{m}(x,P)=n.$ For each $q\in \intvl{1,n}$, we define $\rho_{q}(x,P)=\min \{k\in \intvl{1,m} \suchthat n_{k}(x,P)\geq q\}$ as the minimal rank $k$ at which the $k$-support that $x$ gets at $P$ is at least $q$. We
write $\rho _{q}(P) = \min_{x \in A} \set{\rho_{q}(x, P)}$ for the minimal rank $k$ at which the $k$-support that some alternative gets at $P$ is at least $q$. \textit{A Brams and Kilgour (BK) compromise with threshold }$q$ is the
\ac{SCR} $f_{q}$ defined for each $P\in \linors^N$ as $f_{q}(P)=\{x\in A | n_{\rho _{q}(P)}(x,P)\geq n_{\rho _{q}(P)}(y,P)$ $\forall y\in A\}.$

\begin{theorem}
	\label{th:FBsatsPC}
Let $n\geq 3$ and $m\geq 3.$ The BK compromise $f_{n}$ satisfies PC.
\end{theorem}

\begin{proof}
Define $\bar{\sigma } \in \Sigma$ as, $\forall l \in \intvl{0,m-1}^N$: $\bar\sigma(l) = 1$ iff $\exists i, j \in N \suchthat l_i ≠ l_j$; $\bar\sigma(l) = 0$ otherwise.
Considering any $x \in f_n(P)$, let us show that $x \in \mustar[\bar{\sigma}]$. Because $x \in f_n(P)$, $x \in \paretopt(P)$, and therefore, suffices to show that $\forall y \in \paretopt(P)$, $\bar{\sigma}(\lambda_P(y)) ≥ \bar{\sigma}(\lambda_P(x))$. Given the choice of $\bar{\sigma}$, picking any $y \in \paretopt(P)$ with $y≠x$, suffices to show that $\bar{\sigma}(\lambda_P(y)) = 1$, equivalently, that $\exists i, j \in N \suchthat r_{\prefi}(y) ≠ r_{\pref_j}(y)$. 
Because $x \in f_n(P)$, $\rho_n(P) = \rho_n(x, P) = \max_{i \in N} r_{\prefi}(x)$.
It follows from $\rho_n(P) = \min_{z \in A} \set{\rho_n(z, P)}$ that $\rho_n(y, P) ≥ \rho_n(x, P)$, thus, $\exists i \in N \suchthat r_{\prefi}(y) ≥ \rho_n(P)$. 
Also, $y \in \paretopt(P)$ implies that $\exists j \in N \suchthat r_{\pref_j}(y) < r_{\pref_j}(x)$, thus $\exists j \in N \suchthat r_{\pref_j}(y) < \rho_n(P)$. 
Therefore, $r_{\prefi}(y) ≠ r_{\pref_j}(y)$.
\end{proof}

\section{Two voters case}
Copied from our submitted paper.

In \cref{sec:more2voters} we focused on the analysis of voting rules when the number of voters involved into the decision process is greater than two. Keeping the notation introduced in \cref{sec:notation}, we consider here the case $n=2$. Two individuals express their preference over a set of alternatives $A$, and the goal is to find a common agreement on the alternative to select. This class of problems is often referred to as bargaining problems. In addition to \textit{fallback bargaining (FB)} \citep{Brams2001} (defined in \cref{sec:BKn3}) we consider three prominent solutions of the literature.

\textit{Pareto-and-Veto rules (PV)} \citep{Laslier2020} distribute a veto power of $v_1$ and $v_2$ alternatives to voters 1 and 2, respectively, with $v_1+v_2=m-1$. So, every voter $i=1,2$ (simultaneously) vetoes his worst $v_i$ alternatives. The \ac{SCR} picks all non-vetoed and Pareto optimal alternatives.

The \textit{Veto-Rank mechanism (VR)} is commonly used in the selection of arbitrators \citep{Clippel2014}. Given a list of $m$ (odd) alternatives (that are candidates to be arbitrators), each of the two voters (that are the two parties that must agree on an arbitrator) simultaneously vetoes his worst $\frac{m-1}{2}$ alternatives. The selected alternatives are the ones with the highest Borda score among the non-vetoed alternatives.

Again within the context of selecting arbitrators, \citet{Clippel2014} propose and analyze \textit{Shortlisting (SL)} where one of the two parties starts by vetoing her worst $\frac{m-1}{2}$ alternatives ($m$ being odd), and then the second party chooses her best alternative out of the remaining ones. As the outcome of the procedure depends on the party that starts, symmetry among players is ensured by defining the solution as the union of the two outcomes where one and the other party starts.

\begin{definition}
	Given any $m \geq 7$, a spread measure $\sigma \in \Sigma$ satisfies condition $D_m$ iff 
	$\sigma(\ceil{\frac{m}{2}}, \ceil{\frac{m}{2}} - 2) < \sigma(0, \ceil{\frac{m}{2}} - 1)$ and 
	$\sigma(\ceil{\frac{m}{2}} - 2, \ceil{\frac{m}{2}}) < \sigma(\ceil{\frac{m}{2}} - 1, 0)$.
\end{definition}

For $m=7$ the condition requires $\sigma(4, 2) < \sigma(0, 3)$ and $\sigma(2, 4) < \sigma(3, 0)$ which is reasonable in our context. When the value of $m$ is larger, the condition appears even more convincing. As $m$ grows, the distance between $0$ and $\ceil{\frac{m}{2}} - 1$ grows, while the distance between $\ceil{\frac{m}{2}}$ and $\ceil{\frac{m}{2}} - 2$ remains constant. Requiring, for example, the spread of $(15, 13)$ to be smaller than the spread of $(0, 14)$ is very reasonable.

We write $\Sigma^{D_{m}} \subseteq \Sigma$ for the set of spread measures that satisfy condition $D_{m}$. 

\begin{theorem} \label{th:2votPCC}
	Let $m \geq 7$. Under $\Sigma^{D_{m}}$, FB and PV fail PCC. Furthermore, when $m$ is odd, VR and SL also fail PCC.
\end{theorem}
\begin{proof}
	Take any $m \geq 7$ and any $\sigma \in \Sigma^{D_m}$. Define $\alpha = \ceil{\frac{m}{2}} - 1$ and $\beta = \ceil{\frac{m}{2}} - 2$. It follows from $\sigma \in \Sigma^{D_{m}}$, that $\sigma(\alpha + 1, \beta) < \sigma(0, \beta + 1)$ and $\sigma(\beta, \alpha + 1) < \sigma(\beta + 1, 0)$.
	For $m$ odd, note that $\alpha + \beta + 2 = m$ and consider the profile $P$ where voter $i_1$ has the preference $x \succ a_1 \succ … \succ a_\alpha \succ y \succ b_1 \succ … \succ b_\beta$ and voter $i_2$ has the preference $b_1 \succ … \succ b_\beta \succ y \succ x \succ a_1 \succ … \succ a_\alpha$. For $m$ even, note that $\alpha + \beta + 3= m$, and define the profile $P$ in the same way, except that a supplementary alternative $z$ is added at the bottom of both rankings.
	
	Note that $\sigma(\lambda_{P}(y)) = \sigma(\alpha + 1, \beta)$ and that $\sigma(\lambda_{P}(x)) = \sigma(0, \beta + 1)$. 
	Therefore, $\sigma(\lambda_{P}(y)) < \sigma(\lambda_{P}(x))$. As $y$ is not Pareto-dominated, an \ac{SCR} that uniquely picks $x$ at $P$ cannot be PCC. In a similar vein, at the profile $P'$ which is obtained by the inversion of the preferences of $i_1$ and $i_2$ at $P$, an \ac{SCR} that is PCC cannot pick $x$ uniquely.	
	
	The proof will be concluded by showing that FB, PV, and (when $m$ is odd) VR and SL all pick only $x$ at $P$ or at $P'$.
	
	We readily see that FB picks only $x$ at $P$ (and at $P'$) since $x$ is the first alternative which reaches the unanimous consent.
	For PV, let $v_{i_1} ≥ v_{i_2}$ (thus $v_{i_1} ≥ \ceil{\frac{m-1}{2}} ≥ \ceil{\frac{m-2}{2}} = \beta + 1$ and $v_{i_2} ≤ \floor{\frac{m - 1}{2}} = \ceil{\frac{m - 2}{2}} = \alpha$), and consider the profile $P$. Observe that the first voter vetoes at least $y$ and every $b_j$ ($1 ≤ j ≤ \beta$) while no voter vetoes $x$. As $x$ Pareto-dominates every $a_j$ ($1 ≤ j ≤ \alpha$), PV picks only $x$ at $P$. When $v_{i_2} ≥ v_{i_1}$, a similar reasoning yields that PV picks only $x$ at $P'$.
	
	Now let $m$ be odd.
	
	For VR, a reasoning similar to the one applied to PV yields $x$ as the unique choice at $P$: each voter vetoes her worst $\frac{m-1}{2}$ alternatives, thus $i_1$ vetoes $y$ and every $b_j$ ($1 ≤ j ≤ \beta$) and $i_2$ vetoes every $a_j$ ($1 ≤ j ≤ \alpha$). The alternative $x$ is the only non-vetoed alternative, so it is selected as the sole winner.
	
	Finally, SL also picks $x$, as it is the unique winner no matter which voter starts the veto phase. If $i_1$ starts, $y$ and every $b_j$ ($1 ≤ j ≤ \beta$) get vetoed, then $i_2$ chooses her best alternative out of the remaining ones which is $x$. If $i_2$ starts, every $a_j$ ($1 ≤ j ≤ \alpha$) get vetoed, then $i_1$ chooses her best alternative which is $x$. 
\end{proof}
\end{document}
