\RequirePackage[l2tabu, orthodox]{nag}
\documentclass[version=3.21, pagesize, twoside=off, bibliography=totoc, DIV=calc, fontsize=12pt, a4paper]{scrartcl}
\input{preamble/packages}
\input{preamble/redac}
\input{preamble/math_basics}
%Decision Theory (MCDA and SC)
\NewDocumentCommand{\allalts}{}{\mathscr{A}}
\NewDocumentCommand{\allcrits}{}{\mathscr{C}}
\NewDocumentCommand{\alts}{}{A}
\NewDocumentCommand{\dm}{}{i}
\NewDocumentCommand{\allF}{}{\mathscr{F}}
\NewDocumentCommand{\allvoters}{}{\mathscr{N}}
\NewDocumentCommand{\voters}{}{N}
\NewDocumentCommand{\allprofs}{}{\linors^{\set{1, 2}}}
\NewDocumentCommand{\fprofs}{}{\mathscr{G}}
\NewDocumentCommand{\prof}{}{P}
\NewDocumentCommand{\profB}{}{P_B}
\NewDocumentCommand{\profD}{}{P_D}
\NewDocumentCommand{\ibar}{}{\overline{i}}
\NewDocumentCommand{\lprof}{}{\lambda_P}
\NewDocumentCommand{\lprofB}{}{\lambda_{P_B}}
\NewDocumentCommand{\lprofD}{}{\lambda_{P_D}}
\NewDocumentCommand{\lprofp}{}{\lambda_{P'}}
\NewDocumentCommand{\lprofi}{O{x}}{\lambda_P(#1)_i}
\NewDocumentCommand{\lprofibar}{O{x}}{\lambda_P(#1)_{\overline{i}}}
\NewDocumentCommand{\ineq}{}{(d \circ \lambda_P)}

\NewDocumentCommand{\linors}{}{\mathcal{L}(\allalts)}
%Thanks to https://tex.stackexchange.com/q/154549
	%\makeatletter
	%\def\@myRgood@#1#2{\mathrel{R^X_{#2}}}
	%\def\myRgood{\@ifnextchar_{\@myRgood@}{\mathrel{R^X}}}
	%\makeatother
\NewDocumentCommand{\pref}{}{\succ}
\NewDocumentCommand{\prefto}{}{\succ^{\mkern-8mu -1}}
\NewDocumentCommand{\preftoeq}{}{\succeq^{\mkern-8mu -1}}
\NewDocumentCommand{\prefi}{O{i}}{\succ_{#1}}
\NewDocumentCommand{\prefiinv}{O{i}}{\prec_{#1}}
\NewDocumentCommand{\PO}{}{\mathit{PO}}
\NewDocumentCommand{\paretopt}{}{\mathit{PO}}
\NewDocumentCommand{\SPPd}{}{\Sigma^\text{PPd}}
\NewDocumentCommand{\SAll}{}{\Sigma^\text{All}}
\NewDocumentCommand{\SThreshold}{}{\Sigma_\text{threshold}}
\NewDocumentCommand{\vpr}{}{\boldsymbol{v}}

\NewDocumentCommand{\musigma}{O{\sigma}O{P}}{\min_{{#1}\circ\lambda_{{#2}}}(A)}
\NewDocumentCommand{\mustar}{O{\sigma}O{P}}{\min_{{#1} \circ \lambda_{#2}} (\paretopt({#2}))}
\NewDocumentCommand{\minineq}{O{\allalts}}{\min_{#1}(d \circ \lambda_P)}
\NewDocumentCommand{\MS}{}{\mathit{MS}}
\NewDocumentCommand{\MSP}{}{\mathit{MS(P)}}
\NewDocumentCommand{\FB}{}{\mathit{FB}}
\NewDocumentCommand{\FBP}{}{\mathit{FB}(P)}
\NewDocumentCommand{\POP}{}{\mathit{PO}(P)}
\NewDocumentCommand{\khalf}{}{\floor{\frac{m}{2}}}

\NewDocumentCommand{\alllosses}{}{\intvl{0, m-1}^{\set{1, 2}}}

\NewDocumentCommand{\Ptop}{}{\bar{P}}
\NewDocumentCommand{\sigmatop}{}{\bar{\sigma}}

\NewDocumentCommand{\fltwo}{}{\floor{\bar{l_2}}}
\NewDocumentCommand{\bltwo}{}{\bar{l_2}}

\newtheorem{conjecture}{Conjecture}


%I find these settings useful in draft mode. Should be removed for final versions.
	%Which line breaks are chosen: accept worse lines, therefore reducing risk of overfull lines. Default = 200.
		\tolerance=2000
	%Accept overfull hbox up to...
		\hfuzz=2cm
	%Reduces verbosity about the bad line breaks.
		\hbadness 5000
	%Reduces verbosity about the underful vboxes.
		\vbadness=1300

\title{Compromise XPs}
\author[1]{Olivier Cailloux}
\author[2]{Ayça Ebru Giritligil}
\author[2]{Ipek Ozkal Sanver}
\author[1]{Remzi Sanver}
\affil[1]{Université Paris-Dauphine, PSL Research University, CNRS, LAMSADE, 75016 PARIS, FRANCE}
\affil[2]{Bilgi, …}
\hypersetup{
	pdfsubject={Social choice},
	pdfkeywords={axiomatic analysis},
}

\begin{document}
\maketitle

\begin{abstract}
	Goal: investigate empirically the conditions in which people adopt a compromise notion based on minimal inequality of losses, as opposed to more classical compromise notions, in two-persons situations.
\end{abstract}

\section{Introduction}
\label{sec:introduction}
\label{sec:notation}
We have a non empty set of alternatives $A$ with $\card{A} = m$ and a set of two individuals $N = \set{1, 2}$. The possible profiles are $\linors^{\set{1, 2}}$. A \ac{SCR} is a function $f: \linors^{\set{1, 2}} → \powersetz{A}$. Given $\prof \in \allprofs$, the set of Pareto-optimal alternatives is $\POP = \set{x \in A \suchthat \forall y \in A: x \prefi[1] y \lor x \prefi[2] y}$.
A \ac{SCR} $f$ is \emph{Paretian} iff it selects only Pareto-optimal alternatives; formally, iff $\forall P \in \linors^{\set{1, 2}}: f(P) \subseteq \POP$.

Given $\prof \in \allprofs$ and $x \in A$, let $\lprof(x): \set{1, 2} → \intvl{0, m - 1}$ associate to each individual her loss at $x$, defined as the number of alternatives that are strictly preferred to $x$ in her preference $\prefi$.
Let $\sigma: \alllosses → \R^+$ be an inequality measure, with the supplementary condition that it is minimal exactly for constant loss vectors: $\forall l \in \alllosses: [\sigma(l) = 0] ⇔ [\exists k \in \intvl{0, m - 1} \suchthat l = (k, k)]$.
Let $\Sigma$ denote the set of relations satisfying this condition. 

Given the definitions so far, $\argmin_{\POP} (\sigma \circ \lprof)$ denotes the least unequal alternatives among the Pareto-optimal ones, according to the measure $\sigma \circ \lprof$.
Given a SCR $f$, say that $f$ is Pareto Compromise Compatible (PCC) iff $\exists \sigma \in \Sigma \suchthat \forall \prof \in \allprofs: f(\prof) \cap \argmin_{\POP} (\sigma \circ \lprof) ≠ \emptyset$.

Let $\min\lprof(x) = \min_{i \in N}{\lprof(x)_i}$ and $\max\lprof(x) = \max_{i \in N}{\lprof(x)_i}$ designate the minimal and maximal value of the loss vector of $x$.
Define $\rho = \min\set{l \in \intvl{0, m - 1} \suchthat \exists x \suchthat \max\lprof(x) = l}$. 
Thus, $\nexists x \suchthat \max\lprof(x) < \rho$.
Define the SCR Fallback-Bargaining \citep{Brams2001} as $\FBP = \set{x \suchthat \max\lprof(x) ≤ \rho}$, or equivalently, $\FBP = \set{x \suchthat \max\lprof(x) = \rho}$.

\section{Absolute difference of losses}
We now consider the inequality measure $d \in \Sigma$ defined as the absolute difference between the losses. Given a loss vector $l \in \alllosses$, define $d(l) = \abs{l_1 - l_2}$.

We are interested in studying $\set{\prof \in \allprofs \suchthat \minineq[\FBP] ≠ \minineq[\POP]}$, which is equivalent to: $\set{\prof \in \allprofs \suchthat \minineq[\FBP] > \minineq[\POP]}$.

\subsection{Some useful facts}
Consider any profile $\prof \in \allprofs$. Pick any $x \in \FBP$. Pick any $i \in N$ such that $\lprofi[x] = \min\lprof(x)$ and define $\ibar$ as the other individual, thus with $\lprofibar[x] = \max\lprof(x) = \rho$. Define $z$ as the alternative such that $\lprofi[z] = \rho$.

\begin{theorem}
	\label{th:fourfacts}
	The following four facts hold for any value of $m$.
	\begin{enumerate}
		\item $\rho ≤ \frac{m}{2}$.
		\item If $\rho = \frac{m}{2}$, then $\FBP = \set{x, z}$. 
		\item If $\FBP = \set{x, z}$, then $\minineq[\FBP] = \minineq[\POP]$. 
		\item If $\minineq[\FBP] ≤ 2$, then $\minineq[\FBP] = \minineq[\POP]$.
	\end{enumerate}
\end{theorem}
\begin{proof}
 	\emph{First fact}
 	
	Define $S_1 = \set{y ≠ x \suchthat 0 ≤ \lprofi[y] < \rho}$ and $S_2 = \set{y \suchthat \rho < \lprofibar[y] ≤ m - 1}$.
 	By definition of $\rho$, $\nexists y \suchthat \max\lprof(y) < \rho$,
	thus, $\forall y: [\lprofi[y] < \rho \land \lprofibar[y] ≠ \rho] ⇒ \rho < \lprofibar[y]$.
	Because $x ≠ y ⇒ \lprofibar[y] ≠ \rho$ (as $\lprofibar[x] = \rho$), we obtain that $S_1 \subseteq S_2$.
	Because $\card{\set{y \suchthat 0 ≤ \lprofi[y] < \rho}} = \rho$, $\card{S_1} ≥ \rho - 1$.
	It follows that $\card{S_2} ≥ \rho - 1$.
	Also, $\card{S_2} = m - 1 - \rho$. We obtain $m - 1 - \rho ≥ \rho - 1$, thus $\rho ≤ \frac{m}{2}$.
	
	\emph{Second fact}
	
	Pursuing with $S_1$ and $S_2$ as defined above, and assuming further that $\rho = \frac{m}{2}$, or equivalently $m - \rho = \rho$, we see that $S_1$ and $S_2$ have the same cardinalities, hence, are equal. By definition of $z$, $z \notin S_1$. Thus, $z \notin S_2$. Therefore, $\lprofibar[z] ≤ \rho$, and as $\lprofi[z] = \rho$, $z \in \FBP$. As $\FBP = \set{y \suchthat \max\lprof(y) = \rho}$, $y \in \FBP$ requires that $\lprofi[y] = \rho$ or $\lprofibar[y] = \rho$, thus, no alternative but $x$ and $z$ may be in $\FBP$.
	
	\emph{Third fact}
	
	Assuming now that $\FBP = \set{x, z}$, and picking any $y \in \POP$, let us show that $\ineq(y) ≥ \minineq[\FBP]$.
	By hypothesis, $\lprofibar[z] ≤ \rho$ and $\lprofi[x] ≤ \rho$.
	Now if $\rho < \lprofi[y]$, then $\lprofibar[y] < \lprofibar[z]$ (otherwise $z$ Pareto dominates $y$, as $\lprofi[z] = \rho$), whence $\ineq(y) > \ineq(z)$ as $\lprofibar[y] < \lprofibar[z] ≤ \lprofi[z] = \rho < \lprofi[y]$.
	Similarly, assuming that $\rho < \lprofibar[y]$ yields that $\ineq(y) > \ineq(x)$.
	The only remaining possibility is that $\lprofi[y] ≤ \rho$ and $\lprofibar[y] ≤ \rho$, in which case $y \in \FBP$.
	
	\emph{Fourth fact}
	
	Assuming that $\minineq[\FBP] ≤ 2$, and picking any $y \in \POP$, let us show that $\ineq(y) ≥ \minineq[\FBP]$.
	If $\rho < \lprofi[y]$, then $\lprofi[x] < \lprofi[y]$ and $\lprofibar[x] < \lprofi[y]$, thus $\lprofibar[y] < \lprofibar[x]$ (otherwise $x$ Pareto dominates $y$), thus $\lprofibar[y] < \lprofibar[x] < \lprofi[y]$, thus $\ineq(y) ≥ 2$.
	If $\rho < \lprofibar[y]$, an identical reasoning, exchanging $i$ and $\ibar$, concludes identically.
	Otherwise, $y \in \FBP$.
 \end{proof}

\begin{corollary}
	\label{th:morethan6}
 	For $m ≤ 6$, FB satisfies PCC for the inequality relation $d$.
\end{corollary}
\begin{proof}
	Using the first fact of \cref{th:fourfacts}, $\rho ≤ 3$, and suffices then to use either, if $\rho = 3$, the second and the third facts, or otherwise, the fourth fact (because $\minineq[\FBP] ≤ \rho$) to prove as required that some alternative in $\FBP$ reaches the required minimal inequality over the Pareto optimal alternatives.
\end{proof}

\subsection{The profiles}
\begin{corollary}
	\label{th:conds}
	The profiles such that $\minineq[\FBP] ≠ \minineq[\POP]$ must satisfy all the following conditions. Let $x \in \FBP$.
	\begin{itemize}
		\item $m ≥ 7$.
		\item $\card{\FBP} = 1$.
		\item $3 ≤ \ineq(x) ≤ \rho < \frac{m}{2}$.
		\item $2 ≤ \minineq[\POP] < \ineq(x)$.
	\end{itemize}
\end{corollary}
\begin{proof}
	We see that these conditions are required using \cref{th:fourfacts} and \cref{th:morethan6}: $\FBP$ must be a singleton because of the third fact; $3 ≤ \ineq(x)$ because of the fourth fact; and $\rho < \frac{m}{2}$ because of the first and second facts.
\end{proof}

\begin{conjecture}
	Any choice of values for $m$, $\ineq(x)$, $\rho$ and $\minineq[\POP]$ satisfying the conditions of \cref{th:conds} allow to construct a profile with corresponding values such that $\FBP = \set{x}$. 
\end{conjecture}
(I have no proof of that but I think that they can always be constructed.)

The following illustrates the general construction process. 
\begin{example}
	Example for $m = 9$, $\ineq(x) = \rho = 4$, $\minineq[\POP] = 3$.
	\begin{equation}
		\begin{array}{lllllllll}
			a	& b	& c	& d	& e	& f	& g	& h	& i\\
			\scriptscriptstyle{h} & \scriptscriptstyle{i} & f & \scriptscriptstyle{g} & a & \scriptscriptstyle{b} & \scriptscriptstyle{c} & \scriptscriptstyle{d} & \scriptscriptstyle{e}
		\end{array}
	\end{equation}

	To complete the preference of the second individual so that $\FBP = \set{a}$ and the Paretian minimizers of dispersion is $f$, just put $\set{b, c, d, e}$ worst than $a$ and $\set{g, h, i}$ in the remaining places, in any order.

	Note that putting $e$ just after $a$ might actually be significantly different than other permutations of the set $\set{b, c, d, e}$.
\end{example}

\subsection{Classes of profiles}
The constraints on the profiles we want to use in this experiment are the following. Given a profile $\prof$.
\begin{enumerate}
	\item \label{it:minineq} $\minineq[\FBP] > \minineq[\POP]$.
	\item \label{it:paretopt} No pareto-dominated alternative with a smaller rank difference than $\minineq[\POP]$.
	\item $\card{\argmin_{\POP} (d \circ \lprof)} = 1$
\end{enumerate} 
From \cref{it:minineq}, $\rho < \frac{m}{2}$.
The constraint \cref{it:paretopt} stems from a desire to focus on the subject of our study, and not perturb the subject by letting alternatives that do not distinguish our two rules appear possibly attractive. (Studying the attractivity of Pareto-dominated alternatives is an interesting goal but requires its own design, which we leave for future work.)

We are particularly interested in the following features. Given a profile $\prof$ satisfying the above constraints, and writing $\set{x} = \FBP$ and $\set{y} = \argmin_{\POP} (d \circ \lprof)$, as permitted by the constraints, which ensure single winners.
\begin{itemize}
	\item $\minineq[\FBP]$.
	\item The wost loss for $x$: $\rho = \max\lprof(x)$.
	\item The distance between the spreads of ranks for the winners selected by each rule: $\delta = \abs{\minineq[\FBP] - \minineq[\POP]} = \abs{(\rho - \min\lprof(x)) - \minineq[\POP]}$.
	\item The existence of a pareto-dominated alternative with a smaller rank difference than $\minineq[\POP]$.
	\item The easiness of the profile, captured by how many alternatives are together at the bottom of the ranking.
\end{itemize} 

During a run with a given subject, we want to vary $\delta$ in a wide enough interval. We also want $m$ to be small enough. \Cref{th:conds} indicates that $\delta < \frac{m}{2} - 2$. As a compromise, we opt for $m = 13$, which permits that distance of spreads to reach $(6 - 0) - 2) = 4$.

Define $\epsilon = \max \lprof(y) - \max \lprof(x)$. We have $\epsilon ≤ \minineq[\POP] - 1$.

\subsection{Draft}
\begin{theorem}
	$[d - a ≤ c - b \land b > a ⇔ d > c]$ is equivalent to: $[d ≤ c \land b ≤ a]$.
\end{theorem}
\begin{proof}
	From the left hand side, $d > c$ implies $b > a$ hence $d - a > c - b$, excluded; thus, $d ≤ c$, hence, $b ≤ a$. From the right hand side, $d + b ≤ c + a$.
\end{proof}

\begin{theorem}
	$[y_2 - y_1 ≤ x_2 - x_1 \land x_1 ≤ y_1 ⇔ y_2 ≤ x_2]$ is equivalent to $[y_2 - y_1 ≤ x_2 - x_1 \land x_1 ≤ y_1 \land y_2 ≤ x_2]$ and equivalent to $[x_1 ≤ y_1 \land y_2 ≤ x_2]$.
\end{theorem}
\begin{proof}
	From the left hand side, $y_2 > x_2$ implies $x_1 > y_1$ hence $y_2 - y_1 > x_2 - x_1$, excluded; thus, $y_2 ≤ x_2$, hence, $x_1 ≤ y_1$. From the right hand side, $y_2 + x_1 ≤ x_2 + y_1$.
\end{proof}

\begin{theorem}
	$[y_2 - y_1 ≤ x_2 - x_1 \land x_1 < y_1 ⇒ y_2 < x_2]$ implies that $x_2 ≥ y_2$.
\end{theorem}
\begin{proof}
	If $x_1 < y_1$, done; otherwise, $x_1 ≥ y_1$, equivalently, $x_1 - y_1 ≥ 0$, and we know $x_2 - y_2 ≥ x_1 - y_1$ thus $x_2 - y_2 ≥ 0$ whence $x_2 ≥ y_2$.
\end{proof}

\begin{theorem}
	$[\max \set{y_1, y_2} - \min \set{y_1, y_2} ≤ x_2 - x_1]$ implies $[y_2 - y_1 ≤ x_2 - x_1]$.
\end{theorem}
\begin{proof}
	If $\max \set{y_1, y_2} = y_2$, this is immediate, and otherwise, $y_2 - y_1 ≤ y_1 - y_2$ and $y_1 - y_2 ≤ x_2 - x_1$ hence $y_2 - y_1 ≤ x_2 - x_1$.
\end{proof}

\begin{theorem}
	$[\max \set{y_1, y_2} - \min \set{y_1, y_2} ≤ x_2 - x_1 \land x_1 ≤ y_1 ⇔ y_2 ≤ x_2]$ implies the same equivalent statements as above; but $[x_1 ≤ y_1 \land y_2 ≤ x_2 \land x_2 ≥ x_1 \land \max \set{y_1, y_2} = y_1]$ does not imply $y_1 - y_2 ≤ x_2 - x_1$.
\end{theorem}
\begin{proof}
	For the implication, see above. For the lack of implication, consider $0 = y_2 < x_1 = 1 < x_2 = 2 < y_1 = 3$.
\end{proof}

When $\rho = 6$, to ensure $\FBP = \set{x}$, must put $\set{a, b, c, d, e, f, g} \setminus \set{x}$ after $x$, hence, every alternative among $\set{h, i, j, k, l, m}$ must be better than $x$ for $2$, thus, $\argmin_{\POP} (d \circ \lprof) = h$, whatever $\minineq[\POP]$: any alternative among $\set{i, j, k, l, m}$ in between $h$ and $x$ for $2$ is Pareto-dominated by $h$ and any alternative among $\set{i, j, k, l, m}$ better than $h$ for $2$ has a greater spread.

%$\max \lprof(y)
Alts in between loss for $1$ equal to $\rho + 1$ up to winner of $PO$ must be after $x$ for $2$.

Let $y \in \argmin_{\POP} (\sigma \circ \lprof)$, let $\ibar$ denote a voter such that $\lprof(x)_{\ibar} = \rho$ for some $x \in \FBP$. 
If $\minineq[\FBP] > \minineq[\POP]$, $y$ is the first alt not after $x$ for $\ibar$. $y \in \argmin_{\set{w \suchthat \lprof(w)_{\ibar} < \rho}} \lprof(w)_i$. 

\begin{proof}
$x, y \in \POP$ thus $\lprof(y)_i < \lprof(x)_i ⇔ \lprof(y)_{\ibar} > \lprof(x)_{\ibar}$. 
Also, $y$ has smallest spread; formally, $\max \lprof(y) - \min \lprof(y) ≤ \lprof(x)_{\ibar} - \lprof(x)_i$.

These two claims imply that $\lprof(y)_{\ibar} ≤ \lprof(x)_{\ibar} = \rho \land \lprof(x)_i ≤ \lprof(y)_i$. (Using Theorems 5 then 3.)

By definition of $\rho$, $\max \lprof(y) ≥ \rho$. We obtain $\lprof(y)_{\ibar} ≤ \lprof(x)_{\ibar} = \rho ≤ \max \lprof(y) \land \lprof(x)_i ≤ \lprof(y)_i$.

Assume $\rho < \max \lprof(y)$. Thus, $\max \lprof(y) = \lprof(y)_i$.
Pick any $w$ such that $\lprof(w)_{\ibar} < \rho$.
Because $y$ is not Pareto-dominated by $w$, $\lprof(w)_{\ibar} < \lprof(y)_{\ibar} ⇒ \lprof(w)_i > \lprof(y)_i$. 
By definition of $\rho$, $\max \lprof(w) ≥ \rho$. Thus, $\lprof(w)_i ≥ \rho > \lprof(w)_{\ibar}$.
Also, $y$ has smallest spread; formally, $\max \lprof(y) - \min \lprof(y) ≤ \lprof(w)_i - \lprof(w)_{\ibar}$.
Theorems 5 then 4 yield that $\lprof(w)_i ≥ \lprof(y)_i$.

Assume $\rho = \max \lprof(y)$. Then, $y \in \FBP$, by definition of $\FB$.
\end{proof}

\begin{conjecture}
	If $y$ is a minimizer of spread such that $\lprof(y)_i ≤ \lprof(y)_{\ibar}$, then $y \in \FBP$.
	And conversely.
\end{conjecture}
In other words, FB selects the minimizers of spread that are in the right direction?

\subsection{Examples}
\begin{example}
	$m = 13$, $\minineq[\FBP] = 6$, $\minineq[\POP] = 2$, hence $\epsilon = 1$.
	\begin{equation}
		\begin{array}{lllllllllllll}
			\bm{a}	& b	& c	& d	& e	& f	& g	& h	& i & j & k & l & m\\
			& & & & & h & \bm{a}
		\end{array}
	\end{equation}
	Put $\set{b, c, d, e, f, g}$ after $a$.
\end{example}

\begin{example}
	$m = 13$, $\minineq[\FBP] = 6$, $\minineq[\POP] = 3$, hence $\epsilon = 1$.
	\begin{equation}
		\begin{array}{lllllllllllll}
			\bm{a}	& b	& c	& d	& e	& f	& g	& h	& i & j & k & l & m\\
			& & & & h & & \bm{a}
		\end{array}
	\end{equation}
	Put $\set{b, c, d, e, f, g}$ after $a$.
\end{example}

\begin{example}
	$m = 13$, $\minineq[\FBP] = 6$, $\minineq[\POP] = 4$, hence $\epsilon = 1$.
	\begin{equation}
		\begin{array}{lllllllllllll}
			\bm{a}	& b	& c	& d	& e	& f	& g	& h	& i & j & k & l & m\\
			& & & h & & & \bm{a}
		\end{array}
	\end{equation}
	Put $\set{b, c, d, e, f, g}$ after $a$.
\end{example}

\begin{example}
	$m = 13$, $\minineq[\FBP] = 6$, $\minineq[\POP] = 5$, hence $\epsilon = 1$.
	\begin{equation}
		\begin{array}{lllllllllllll}
			\bm{a}	& b	& c	& d	& e	& f	& g	& h	& i & j & k & l & m\\
			& & h & & & & \bm{a}
		\end{array}
	\end{equation}
	Put $\set{b, c, d, e, f, g}$ after $a$.
\end{example}

\begin{example}
	$m = 13$, $\minineq[\FBP] = 5$, $\minineq[\POP] = 2$, $\rho = 6$, hence $\epsilon = 1$.
	\begin{equation}
		\begin{array}{lllllllllllll}
			a	& \bm{b}	& c	& d	& e	& f	& g	& h	& i & j & k & l & m\\
			& & & & & h & \bm{b}
		\end{array}
	\end{equation}
	Put $\set{a, c, d, e, f, g}$ after $b$.
\end{example}

\begin{example}
	$m = 13$, $\minineq[\FBP] = 5$, $\minineq[\POP] = 2$, $\rho = 5$, hence $\epsilon = 1$.
	\begin{equation}
		\begin{array}{lllllllllllll}
			\bm{a}	& b	& c	& d	& e	& f	& g	& h	& i & j & k & l & m\\
			& & & & g & \bm{a}
		\end{array}
	\end{equation}
	Put $\set{b, c, d, e, f}$ after $a$.
\end{example}

\begin{example}
	$m = 13$, $\minineq[\FBP] = 5$, $\minineq[\POP] = 3$, $\rho = 6$, hence $\epsilon = 1$.
	\begin{equation}
		\begin{array}{lllllllllllll}
			a	& \bm{b}	& c	& d	& e	& f	& g	& h	& i & j & k & l & m\\
			& & & & h & & \bm{b}
		\end{array}
	\end{equation}
	Put $\set{a, c, d, e, f, g}$ after $b$.
\end{example}

\begin{example}
	$m = 13$, $\minineq[\FBP] = 5$, $\minineq[\POP] = 3$, $\rho = 5$, $\epsilon = 1$.
	\begin{equation}
		\begin{array}{lllllllllllll}
			\bm{a}	& b	& c	& d	& e	& f	& g	& h	& i & j & k & l & m\\
			& & & g & & \bm{a}
		\end{array}
	\end{equation}
	Put $\set{b, c, d, e, f}$ after $a$.
\end{example}

\begin{example}
	$m = 13$, $\minineq[\FBP] = 5$, $\minineq[\POP] = 3$, $\rho = 5$, $\epsilon = 2$.
	\begin{equation}
		\begin{array}{lllllllllllll}
			\bm{a}	& b	& c	& d	& e	& f	& g	& h	& i & j & k & l & m\\
			& & & & h & \bm{a}
		\end{array}
	\end{equation}
	Put $\set{b, c, d, e, f, g}$ after $a$.
\end{example}

\bibliography{biblio}

\appendix
\section{Examples}
For $m = 9$, some of the constraints are: $\minineq[\FBP] \in \intvl{3, 4}$, $\rho \in \intvl{3, 4}$ and $\minineq[\POP] \in \intvl{2, 3}$.

\begin{example}
	$m = 9$, $\minineq[\FBP] = 3$, $\rho = 3$, $\minineq[\POP] = 2$.
	\begin{equation}
		\begin{array}{lllllllll}
			\bm{a}	& b	& c	& d	& e	& f	& g	& h	& i\\
			& & e & \bm{a}
		\end{array}
	\end{equation}
	Put $\set{b, c, d}$ after $a$.
\end{example}

\begin{example}
	$m = 9$, $\minineq[\FBP] = 3$, $\rho = 4$, $\minineq[\POP] = 2$.
	\begin{equation}
		\begin{array}{lllllllll}
			a	& \bm{b}	& c	& d	& e	& f	& g	& h	& i\\
			& & & f & \bm{b}
		\end{array}
	\end{equation}
	Put $\set{a, c, d, e}$ after $b$.
\end{example}

\begin{example}
	$m = 9$, $\minineq[\FBP] = 4$ (hence $\rho = 4$), $\minineq[\POP] = 2$.
	\begin{equation}
		\begin{array}{lllllllll}
			\bm{a}	& b	& c	& d	& e	& f	& g	& h	& i\\
			& & & f & \bm{a}
		\end{array}
	\end{equation}
	Put $\set{b, c, d, e}$ after $a$.
\end{example}

\begin{example}
	$m = 9$, $\minineq[\FBP] = 4$ (hence $\rho = 4$), $\minineq[\POP] = 3$.
	\begin{equation}
		\begin{array}{lllllllll}
			\bm{a}	& b	& c	& d	& e	& f	& g	& h	& i\\
			& & f & & \bm{a}
		\end{array}
	\end{equation}
	Put $\set{b, c, d, e}$ after $a$.
\end{example}

\begin{example}
	$m = 11$, $\minineq[\FBP] = 4$, $\rho = 4$, $\minineq[\POP] = 3$.
	\begin{equation}
		\begin{array}{lllllllllll}
			\bm{a}	& b	& c	& d	& e	& f	& g	& h	& i & j & k\\
			& & & g & \bm{a}
		\end{array}
	\end{equation}
	Put $\set{b, c, d, e, f}$ after $a$.
\end{example}

\begin{example}
	$m = 11$, $\minineq[\FBP] = 4$, $\rho = 4$, $\minineq[\POP] = 3$, alternative.
	\begin{equation}
		\begin{array}{lllllllllll}
			\bm{a}	& b	& c	& d	& e	& f	& g	& h	& i & j & k\\
			& & f & & \bm{a}
		\end{array}
	\end{equation}
	Put $\set{b, c, d, e}$ after $a$.
\end{example}

\begin{example}
	$m = 11$, $\minineq[\FBP] = 5$, $\minineq[\POP] = 2$.
	\begin{equation}
		\begin{array}{lllllllllll}
			\bm{a}	& b	& c	& d	& e	& f	& g	& h	& i & j & k\\
			& & & & g & \bm{a}
		\end{array}
	\end{equation}
	Put $\set{b, c, d, e, f}$ after $a$.
\end{example}

\section{Which \acp{SCR} are compromises?}
Copied from our submitted paper.

\label{sec:more2voters}
In this section we assume $n\geq 3$ and leave the analysis of $n=2$ to the
next section.

\subsection{BK-compromises}
\label{sec:BKn3}
Given any $k\in \intvl{1, m}$, we write $n_{k}(x,P)=\#\{i\in
N\mid r_{\prefi}(x)\leq k\}$ for the \emph{$k$-support} that $x$ gets at $P$, that is, the number of individuals for whom the rank of alternative $x\in A$ is lower than or equal to $k$ in the profile $P\in $ $L(A)^{N}$.
Note that $n_{k}(x,P)\in \intvl{1, n}$ is non-decreasing on $k$ and $n_{m}(x,P)=n.$ For each $q\in \intvl{1,n}$, we define $\rho_{q}(x,P)=\min \{k\in \intvl{1,m} \suchthat n_{k}(x,P)\geq q\}$ as the minimal rank $k$ at which the $k$-support that $x$ gets at $P$ is at least $q$. We
write $\rho _{q}(P) = \min_{x \in A} \set{\rho_{q}(x, P)}$ for the minimal rank $k$ at which the $k$-support that some alternative gets at $P$ is at least $q$. \textit{A Brams and Kilgour (BK) compromise with threshold }$q$ is the
\ac{SCR} $f_{q}$ defined for each $P\in \linors^N$ as $f_{q}(P)=\{x\in A | n_{\rho _{q}(P)}(x,P)\geq n_{\rho _{q}(P)}(y,P)$ $\forall y\in A\}.$

\begin{theorem}
	\label{th:FBsatsPC}
Let $n\geq 3$ and $m\geq 3.$ The BK compromise $f_{n}$ satisfies PC.
\end{theorem}

\begin{proof}
Define $\bar{\sigma } \in \Sigma$ as, $\forall l \in \intvl{0,m-1}^N$: $\bar\sigma(l) = 1$ iff $\exists i, j \in N \suchthat l_i ≠ l_j$; $\bar\sigma(l) = 0$ otherwise.
Considering any $x \in f_n(P)$, let us show that $x \in \mustar[\bar{\sigma}]$. Because $x \in f_n(P)$, $x \in \paretopt(P)$, and therefore, suffices to show that $\forall y \in \paretopt(P)$, $\bar{\sigma}(\lambda_P(y)) ≥ \bar{\sigma}(\lambda_P(x))$. Given the choice of $\bar{\sigma}$, picking any $y \in \paretopt(P)$ with $y≠x$, suffices to show that $\bar{\sigma}(\lambda_P(y)) = 1$, equivalently, that $\exists i, j \in N \suchthat r_{\prefi}(y) ≠ r_{\pref_j}(y)$. 
Because $x \in f_n(P)$, $\rho_n(P) = \rho_n(x, P) = \max_{i \in N} r_{\prefi}(x)$.
It follows from $\rho_n(P) = \min_{z \in A} \set{\rho_n(z, P)}$ that $\rho_n(y, P) ≥ \rho_n(x, P)$, thus, $\exists i \in N \suchthat r_{\prefi}(y) ≥ \rho_n(P)$. 
Also, $y \in \paretopt(P)$ implies that $\exists j \in N \suchthat r_{\pref_j}(y) < r_{\pref_j}(x)$, thus $\exists j \in N \suchthat r_{\pref_j}(y) < \rho_n(P)$. 
Therefore, $r_{\prefi}(y) ≠ r_{\pref_j}(y)$.
\end{proof}

\section{Two voters case}
Copied from our submitted paper.

In \cref{sec:more2voters} we focused on the analysis of voting rules when the number of voters involved into the decision process is greater than two. Keeping the notation introduced in \cref{sec:notation}, we consider here the case $n=2$. Two individuals express their preference over a set of alternatives $A$, and the goal is to find a common agreement on the alternative to select. This class of problems is often referred to as bargaining problems. In addition to \textit{fallback bargaining (FB)} \citep{Brams2001} (defined in \cref{sec:BKn3}) we consider three prominent solutions of the literature.

\textit{Pareto-and-Veto rules (PV)} \citep{Laslier2020} distribute a veto power of $v_1$ and $v_2$ alternatives to voters 1 and 2, respectively, with $v_1+v_2=m-1$. So, every voter $i=1,2$ (simultaneously) vetoes his worst $v_i$ alternatives. The \ac{SCR} picks all non-vetoed and Pareto optimal alternatives.

The \textit{Veto-Rank mechanism (VR)} is commonly used in the selection of arbitrators \citep{Clippel2014}. Given a list of $m$ (odd) alternatives (that are candidates to be arbitrators), each of the two voters (that are the two parties that must agree on an arbitrator) simultaneously vetoes his worst $\frac{m-1}{2}$ alternatives. The selected alternatives are the ones with the highest Borda score among the non-vetoed alternatives.

Again within the context of selecting arbitrators, \citet{Clippel2014} propose and analyze \textit{Shortlisting (SL)} where one of the two parties starts by vetoing her worst $\frac{m-1}{2}$ alternatives ($m$ being odd), and then the second party chooses her best alternative out of the remaining ones. As the outcome of the procedure depends on the party that starts, symmetry among players is ensured by defining the solution as the union of the two outcomes where one and the other party starts.

\begin{definition}
	Given any $m \geq 7$, a spread measure $\sigma \in \Sigma$ satisfies condition $D_m$ iff 
	$\sigma(\ceil{\frac{m}{2}}, \ceil{\frac{m}{2}} - 2) < \sigma(0, \ceil{\frac{m}{2}} - 1)$ and 
	$\sigma(\ceil{\frac{m}{2}} - 2, \ceil{\frac{m}{2}}) < \sigma(\ceil{\frac{m}{2}} - 1, 0)$.
\end{definition}

For $m=7$ the condition requires $\sigma(4, 2) < \sigma(0, 3)$ and $\sigma(2, 4) < \sigma(3, 0)$ which is reasonable in our context. When the value of $m$ is larger, the condition appears even more convincing. As $m$ grows, the distance between $0$ and $\ceil{\frac{m}{2}} - 1$ grows, while the distance between $\ceil{\frac{m}{2}}$ and $\ceil{\frac{m}{2}} - 2$ remains constant. Requiring, for example, the spread of $(15, 13)$ to be smaller than the spread of $(0, 14)$ is very reasonable.

We write $\Sigma^{D_{m}} \subseteq \Sigma$ for the set of spread measures that satisfy condition $D_{m}$. 

\begin{theorem} \label{th:2votPCC}
	Let $m \geq 7$. Under $\Sigma^{D_{m}}$, FB and PV fail PCC. Furthermore, when $m$ is odd, VR and SL also fail PCC.
\end{theorem}
\begin{proof}
	Take any $m \geq 7$ and any $\sigma \in \Sigma^{D_m}$. Define $\alpha = \ceil{\frac{m}{2}} - 1$ and $\beta = \ceil{\frac{m}{2}} - 2$. It follows from $\sigma \in \Sigma^{D_{m}}$, that $\sigma(\alpha + 1, \beta) < \sigma(0, \beta + 1)$ and $\sigma(\beta, \alpha + 1) < \sigma(\beta + 1, 0)$.
	For $m$ odd, note that $\alpha + \beta + 2 = m$ and consider the profile $P$ where voter $i_1$ has the preference $x \succ a_1 \succ … \succ a_\alpha \succ y \succ b_1 \succ … \succ b_\beta$ and voter $i_2$ has the preference $b_1 \succ … \succ b_\beta \succ y \succ x \succ a_1 \succ … \succ a_\alpha$. For $m$ even, note that $\alpha + \beta + 3= m$, and define the profile $P$ in the same way, except that a supplementary alternative $z$ is added at the bottom of both rankings.
	
	Note that $\sigma(\lambda_{P}(y)) = \sigma(\alpha + 1, \beta)$ and that $\sigma(\lambda_{P}(x)) = \sigma(0, \beta + 1)$. 
	Therefore, $\sigma(\lambda_{P}(y)) < \sigma(\lambda_{P}(x))$. As $y$ is not Pareto-dominated, an \ac{SCR} that uniquely picks $x$ at $P$ cannot be PCC. In a similar vein, at the profile $P'$ which is obtained by the inversion of the preferences of $i_1$ and $i_2$ at $P$, an \ac{SCR} that is PCC cannot pick $x$ uniquely.	
	
	The proof will be concluded by showing that FB, PV, and (when $m$ is odd) VR and SL all pick only $x$ at $P$ or at $P'$.
	
	We readily see that FB picks only $x$ at $P$ (and at $P'$) since $x$ is the first alternative which reaches the unanimous consent.
	For PV, let $v_{i_1} ≥ v_{i_2}$ (thus $v_{i_1} ≥ \ceil{\frac{m-1}{2}} ≥ \ceil{\frac{m-2}{2}} = \beta + 1$ and $v_{i_2} ≤ \floor{\frac{m - 1}{2}} = \ceil{\frac{m - 2}{2}} = \alpha$), and consider the profile $P$. Observe that the first voter vetoes at least $y$ and every $b_j$ ($1 ≤ j ≤ \beta$) while no voter vetoes $x$. As $x$ Pareto-dominates every $a_j$ ($1 ≤ j ≤ \alpha$), PV picks only $x$ at $P$. When $v_{i_2} ≥ v_{i_1}$, a similar reasoning yields that PV picks only $x$ at $P'$.
	
	Now let $m$ be odd.
	
	For VR, a reasoning similar to the one applied to PV yields $x$ as the unique choice at $P$: each voter vetoes her worst $\frac{m-1}{2}$ alternatives, thus $i_1$ vetoes $y$ and every $b_j$ ($1 ≤ j ≤ \beta$) and $i_2$ vetoes every $a_j$ ($1 ≤ j ≤ \alpha$). The alternative $x$ is the only non-vetoed alternative, so it is selected as the sole winner.
	
	Finally, SL also picks $x$, as it is the unique winner no matter which voter starts the veto phase. If $i_1$ starts, $y$ and every $b_j$ ($1 ≤ j ≤ \beta$) get vetoed, then $i_2$ chooses her best alternative out of the remaining ones which is $x$. If $i_2$ starts, every $a_j$ ($1 ≤ j ≤ \alpha$) get vetoed, then $i_1$ chooses her best alternative which is $x$. 
\end{proof}
\end{document}
