\RequirePackage[l2tabu, orthodox]{nag}
\documentclass[pagesize, twoside=off, bibliography=totoc, DIV=calc, fontsize=12pt, a4paper]{scrartcl}
%Permits to copy eg x ⪰ y ⇔ v(x) ≥ v(y) from PDF to unicode data, and to search. From pdfTeX users manual. See https://tex.stackexchange.com/posts/comments/1203887.
	\input glyphtounicode
	\pdfgentounicode=1
%Latin Modern has more glyphs than Computer Modern, such as diacritical characters. fntguide commands to load the font before fontenc, to prevent default loading of cmr.
	\usepackage{lmodern}
%Encode resulting accented characters correctly in resulting PDF, permits copy from PDF.
	\usepackage[T1]{fontenc}
%UTF8 seems to be the default in recent TeX installations, but not all, see https://tex.stackexchange.com/a/370280.
	\usepackage[utf8]{inputenc}
%Provides \newunicodechar for easy definition of supplementary UTF8 characters such as → or ≤ for use in source code.
	\usepackage{newunicodechar}
%Text Companion fonts, much used together with CM-like fonts. Provides \texteuro and commands for text mode characters such as \textminus, \textrightarrow, \textlbrackdbl.
	\usepackage{textcomp}
%St Mary’s Road symbol font, used for ⟦ = \llbracket. The \SetSymbolFont command avoids spurious warnings, but also some valid ones, see https://tex.stackexchange.com/a/106719/.
	\usepackage{stmaryrd}\SetSymbolFont{stmry}{bold}{U}{stmry}{m}{n}
%Solves bug in lmodern, https://tex.stackexchange.com/a/261188; probably useful only for unusually big font sizes; and probably better to use exscale instead. Note that the authors of exscale write against this trick.
	%\DeclareFontShape{OMX}{cmex}{m}{n}{
		%<-7.5> cmex7
		%<7.5-8.5> cmex8
		%<8.5-9.5> cmex9
		%<9.5-> cmex10
	%}{}
	%\SetSymbolFont{largesymbols}{normal}{OMX}{cmex}{m}{n}
%More symbols (such as \sum) available in bold version, see https://github.com/latex3/latex2e/issues/71. In article mode (but not in presentation mode), also hides some potentially useful warnings such as when using $\bm{\llbracket}$, see stmaryrd in this document (not sure why).
	\DeclareFontShape{OMX}{cmex}{bx}{n}{%
	   <->sfixed*cmexb10%
	   }{}
	\SetSymbolFont{largesymbols}{bold}{OMX}{cmex}{bx}{n}
%https://english.stackexchange.com/questions/93008
	\usepackage[super]{nth}
%For small caps also in italics, see https://tex.stackexchange.com/questions/32942/italic-shape-needed-in-small-caps-fonts, https://tex.stackexchange.com/questions/284338/italic-small-caps-not-working.
	\usepackage{slantsc}
	\AtBeginDocument{%
		%“Since nearly no font family will contain real italic small caps variants, the best approach is to substitute them by slanted variants.” -- slantsc doc
		%\DeclareFontShape{T1}{lmr}{m}{scit}{<->ssub*lmr/m/scsl}{}%
		%There’s no bold small caps in Latin Modern, we switch to Computer Modern for bold small caps, see https://tex.stackexchange.com/a/22241
		%\DeclareFontShape{T1}{lmr}{bx}{sc}{<->ssub*cmr/bx/sc}{}%
		%\DeclareFontShape{T1}{lmr}{bx}{scit}{<->ssub*cmr/bx/scsl}{}%
	}
%Warn about missing characters.
	\tracinglostchars=2
%Nicer tables: provides \toprule, \midrule, \bottomrule.
	%\usepackage{booktabs}
%For new column type X which stretches; can be used together with booktabs, see https://tex.stackexchange.com/a/97137. “tabularx modifies the widths of the columns, whereas tabular* modifies the widths of the inter-column spaces.” Loads array.
	%\usepackage{tabularx}
%math-mode version of "l" column type. Requires \usepackage{array}.
	%\usepackage{array}
	%\newcolumntype{L}{>{$}l<{$}}
%Provides \xpretocmd and loads etoolbox which provides \apptocmd, \patchcmd, \newtoggle… Also loads xparse, which provides \NewDocumentCommand and similar commands intended as replacement of \newcommand in LaTeX3 for defining commands (see https://tex.stackexchange.com/q/98152 and https://github.com/latex3/latex2e/issues/89).
	\usepackage{xpatch}
%ntheorem doc says: “empheq provides an enhanced vertical placement of the endmarks”; must be loaded before ntheorem. Loads the mathtools package, which loads and fixes some bugs in amsmath and provides \DeclarePairedDelimiter. amsmath is considered a basic, mandatory package nowadays (Grätzer, More Math Into LaTeX).
	\usepackage[ntheorem]{empheq}
%Package frenchb asks to load natbib before babel-french. Package hyperref asks to load natbib before hyperref.
	\usepackage{natbib}

\newtoggle{LCpres}
	\newtoggle{LCart}
	\newtoggle{LCposter}
	\makeatletter
	\@ifclassloaded{beamer}{
		\toggletrue{LCpres}
		\togglefalse{LCart}
		\togglefalse{LCposter}
		\wlog{Presentation mode}
	}{
		\@ifclassloaded{tikzposter}{
			\toggletrue{LCposter}
			\togglefalse{LCpres}
			\togglefalse{LCart}
			\wlog{Poster mode}
		}{
			\toggletrue{LCart}
			\togglefalse{LCpres}
			\togglefalse{LCposter}
			\wlog{Article mode}
		}
	}
	\makeatother%

%Language options ([french, english]) should be on the document level (last is main); except with tikzposter: put [french, english] options next to \usepackage{babel} to avoid warning. beamer uses the \translate command for the appendix: omitting babel results in a warning, see https://github.com/josephwright/beamer/issues/449. Babel also seems required for \refname.
	\iftoggle{LCpres}{
		\usepackage{babel}
	}{
	}
	%\frenchbsetup{AutoSpacePunctuation=false}
%listings (1.7) does not allow multi-byte encodings. listingsutf8 works around this only for characters that can be represented in a known one-byte encoding and only for \lstinputlisting. Other workarounds: use literate mechanism; or escape to LaTeX (but breaks alignment).
	%\usepackage{listings}
	%\lstset{tabsize=2, basicstyle=\ttfamily, escapechar=§, literate={é}{{\'e}}1}
%I favor acro over acronym because the former is more recently updated (2018 VS 2015 at time of writing); has a longer user manual (about 40 pages VS 6 pages if not counting the example and implementation parts); has a command for capitalization; and acronym suffers a nasty bug when ac used in section, see https://tex.stackexchange.com/q/103483 (though this might be the fault of the silence package and might be solved in more recent versions, I do not know) and from a bug when used with cleveref, see https://tex.stackexchange.com/q/71364. However, loading it makes compilation time (one pass on this template) go from 0.6 to 1.4 seconds, see https://bitbucket.org/cgnieder/acro/issues/115.
	\usepackage[single]{acro}
	\DeclareAcronym{AMCD}{short=AMCD, long={Aide Multicritère à la Décision}}
\DeclareAcronym{AHP}{short=AHP, long={Analytic Hierarchy Process}}
\DeclareAcronym{AR}{short=AR, long={Argumentative Recommender}}
\DeclareAcronym{DA}{short=DA, long={Decision Analysis}}
\DeclareAcronym{DJ}{short=DJ, long={Deliberated Judgment}}
\DeclareAcronym{DM}{short=DM, long={Decision Maker}}
\DeclareAcronym{DP}{short=DP, long={Deliberated Preference}}
\DeclareAcronym{MAVT}{short=MAVT, long={Multiple Attribute Value Theory}}
\DeclareAcronym{MCDA}{short=MCDA, long={Multicriteria Decision Aid}}
\DeclareAcronym{MIP}{short=MIP, long={Mixed Integer Program}}
\DeclareAcronym{SEU}{short=SEU, long={Subjective Expected Utility}}


\iftoggle{LCpres}{
	%I favor fmtcount over nth because it is loaded by datetime anyway; and fmtcount warns about possible conflicts when loaded after nth.
	\usepackage{fmtcount}
	%For nice input of date of presentation. Must be loaded after the babel package. Has possible problems with srcletter: https://golatex.de/verwendung-von-babel-und-datetime-in-scrlttr2-schlaegt-fehlt-t14779.html.
	\usepackage[nodayofweek]{datetime}
}{
}
%For presentations, Beamer implicitely uses the pdfusetitle option. ntheorem doc says to load hyperref “before the first use of \newtheorem”. autonum doc mandates option hypertexnames=false. I want to highlight links only if necessary for the reader to recognize it as a link, to reduce distraction. In presentations, this is already taken care of by beamer (https://tex.stackexchange.com/a/262014). If using colorlinks=true in a presentation, see https://tex.stackexchange.com/q/203056. Crashes the first compilation with tikzposter, just compile again and the problem disappears, see https://tex.stackexchange.com/q/254257.
\makeatletter
\iftoggle{LCpres}{
	\usepackage{hyperref}
}{
	\usepackage[hypertexnames=false, pdfusetitle, linkbordercolor={1 1 1}, citebordercolor={1 1 1}, urlbordercolor={1 1 1}]{hyperref}
	%https://tex.stackexchange.com/a/466235
	\pdfstringdefDisableCommands{%
		\let\thanks\@gobble
	}
}
\makeatother
%urlbordercolor is used both for \url and \doi, which I think shouldn’t be colored, and for \href, thus might want to color manually when required. Requires xcolor.
	\NewDocumentCommand{\hrefblue}{mm}{\textcolor{blue}{\href{#1}{#2}}}
%hyperref doc says: “Package bookmark replaces hyperref’s bookmark organization by a new algorithm (...) Therefore I recommend using this package”.
	\usepackage{bookmark}
%Need to invoke hyperref explicitly to link to line numbers: \hyperlink{lintarget:mylinelabel}{\ref*{lin:mylinelabel}}, with \ref* to disable automatic link. Also see https://tex.stackexchange.com/q/428656 for referencing lines from another document.
	%\usepackage{lineno}
	%\NewDocumentCommand{\llabel}{m}{\hypertarget{lintarget:#1}{}\linelabel{lin:#1}}
	%\setlength\linenumbersep{9mm}
%For complex authors blocks. Seems like authblk wants to be later than hyperref, but sooner than silence. See https://tex.stackexchange.com/q/475513 for the patch to hyperref pdfauthor.
	\ExplSyntaxOn
	\seq_new:N \g_oc_hrauthor_seq
	\NewDocumentCommand{\addhrauthor}{m}{
		\seq_gput_right:Nn \g_oc_hrauthor_seq { #1 }
	}
	%Should be \NewExpandableDocumentCommand, but this is not yet provided by my version of xparse
	\DeclareExpandableDocumentCommand{\hrauthor}{}{
		\seq_use:Nn \g_oc_hrauthor_seq {,~}
	}
	\ExplSyntaxOff
	{
		\catcode`#=11\relax
		\gdef\fixauthor{\xpretocmd{\author}{\addhrauthor{#2}}{}{}}%
	}
	\iftoggle{LCart}{
		\usepackage{authblk}
		\renewcommand\Affilfont{\small}
		\fixauthor
		\AtBeginDocument{
		    \hypersetup{pdfauthor={\hrauthor}}
		}
	}{
	}
%I do not use floatrow, because it requires an ugly hack for proper functioning with KOMA script (see scrhack doc). Instead, the following command centers all floats (using \centering, as the center environment adds space, http://texblog.net/latex-archive/layout/center-centering/), and I manually place my table captions above and figure captions below their contents (https://tex.stackexchange.com/a/3253).
	\makeatletter
	\g@addto@macro\@floatboxreset\centering
	\makeatother
%Permits to customize enumeration display and references
	%\nottoggle{LCpres}{
		%\usepackage{enumitem} %follow list environments by a string to customize enumeration, example: \begin{description}[itemindent=8em, labelwidth=!] or \begin{enumerate}[label=({\roman*}), ref={\roman*}].
	%}{
	%}
%Provides \Centering, \RaggedLeft, and \RaggedRight and environments Center, FlushLeft, and FlushRight, which allow hyphenation. With tikzposter, seems to cause 1=1 to be printed in the middle of the poster.
	%\usepackage{ragged2e}
%To typeset units by closely following the “official” rules.
	%\usepackage[strict]{siunitx}
%Turns the doi provided by some bibliography styles into URLs. However, uses old-style dx.doi url (see 3.8 DOI system Proxy Server technical details, “Users may resolve DOI names that are structured to use the DOI system Proxy Server (https://doi.org (current, preferred) or earlier syntax http://dx.doi.org).”, https://www.doi.org/doi_handbook/3_Resolution.html). The patch solves this.
	\usepackage{doi}
	\makeatletter
	\patchcmd{\@doi}{http://dx.doi.org}{https://doi.org}{}{}
	\makeatother
%Makes sure upper case greek letters are italic as well.
	\usepackage{fixmath}
%Provides \mathbb; obsoletes latexsym (see http://tug.ctan.org/macros/latex/base/latexsym.dtx). Relatedly, \usepackage{eucal} to change the mathcal font and \usepackage[mathscr]{eucal} (apparently equivalent to \usepackage[mathscr]{euscript}) to supplement \mathcal with \mathscr. This last option is not very useful as both fonts are similar, and the intent of the authors of eucal was to provide a replacement to mathcal (see doc euscript). Also provides \mathfrak for supplementary letters.
	\usepackage{amsfonts}
%Provides a beautiful (IMHO) \mathscr and really different than \mathcal, for supplementary uppercase letters. But there is no bold version. Alternative: mathrsfs (more slanted), but when used with tikzposter, it warns about size substitution, see https://tex.stackexchange.com/q/495167.
	\usepackage[scr]{rsfso}
%Multiple means to produce bold math: \mathbf, \boldmath (defined to be \mathversion{bold}, see fntguide), \pmb, \boldsymbol (all legacy, from LaTeX base and AMS), \bm (the most recommended one), \mathbold from package fixmath (I don’t see its advantage over \boldsymbol).
%“The \boldsymbol command is obtained preferably by using the bm package, which provides a newer, more powerful version than the one provided by the amsmath package. Generally speaking, it is ill-advised to apply \boldsymbol to more than one symbol at a time.” — AMS Short math guide. “If no bold font appears to be available for a particular symbol, \bm will use ‘poor man’s bold’” — bm. It is “best to load the package after any packages that define new symbol fonts” – bm. bm defines \boldsymbol as synonym to \bm. \boldmath accesses the correct font if it exists; it is used by \bm when appropriate. See https://tex.stackexchange.com/a/10643 and https://github.com/latex3/latex2e/issues/71 for some difficulties with \bm.
	\usepackage{bm}
	\nottoggle{LCpres}{
	%https://ctan.org/pkg/amsmath recommends ntheorem, which supersedes amsthm, which corrects the spacing of proclamations and allows for theoremstyle. Option standard loads amssymb and latexsym. Must be loaded after amsmath (from ntheorem doc). From cleveref doc, “ntheorem is fully supported and even recommended”; says to load cleveref after ntheorem. When used with tikzposter, warns about size substitution for the lasy (latexsym) font when using \url, because ntheorem loads latexsym; relatedly (but not directly related to ntheorem), size substitution warning with the cmex font happens when loading amsmath and using \url. According to https://tex.stackexchange.com/q/535950, ntheorem “seems essentially unmaintaned and has severe problems”, but I use it anyway because it is very handy. Yields “! LaTeX Error: Something's wrong--perhaps a missing \item.” if some theorem follows thebibliography.
		\usepackage[thmmarks, amsmath, standard, hyperref]{ntheorem}
		%empheq doc says to do this after loading ntheorem
		\usetagform{default}
	%Provides \cref. Unfortunately, cref fails when the language is French and referring to a label whose name contains a colon (https://tex.stackexchange.com/q/83798). Use \cref{sec\string:intro} to work around this. cleveref should go “laster” than hyperref.
		\usepackage{cleveref}
	}{
	}
	\nottoggle{LCposter}{
	%Equations get numbers iff they are referenced. Loading order should be “amsmath → hyperref → cleveref → autonum”, according to autonum doc. Use this in preference to the showonlyrefs option from mathtools, see https://tex.stackexchange.com/q/459918 and autonum doc. See https://tex.stackexchange.com/a/285953 for the etex line. Incompatible with my version of tikzposter (produces “! Improper \prevdepth”).
		\expandafter\def\csname ver@etex.sty\endcsname{3000/12/31}\let\globcount\newcount
		\usepackage{autonum}
	}{
	}
%Also loaded by tikz.
	\usepackage{xcolor}
\iftoggle{LCpres}{
	\usepackage{tikz}
	%\usetikzlibrary{babel, matrix, fit, plotmarks, calc, trees, shapes.geometric, positioning, plothandlers, arrows, shapes.multipart}
}{
}
%Vizualization, on top of TikZ
	%\usepackage{pgfplots}
	%\pgfplotsset{compat=1.14}
\usepackage{graphicx}
	\graphicspath{{graphics/}}

%Provides \printlength{length}, useful for debugging.
	%\usepackage{printlen}
	%\uselengthunit{mm}

\iftoggle{LCpres}{
	\usepackage{appendixnumberbeamer}
	%I have yet to see anyone actually use these navigation symbols; let’s disable them
	\setbeamertemplate{navigation symbols}{} 
	\usepackage{preamble/beamerthemeParisFrance}
	\setcounter{tocdepth}{10}
}{
}

%Do not use the displaymath environment: use equation. Do not use the eqnarray or eqnarray* environments: use align(*). This improves spacing. (See l2tabu or amsldoc.)


%Requires package xcolor.
%\definecolor{ao(english)}{rgb}{0.0, 0.5, 0.0}
\NewDocumentCommand{\commentOC}{m}{\textcolor{blue}{\small$\big[$OC: #1$\big]$}}
\NewDocumentCommand{\commentRS}{m}{\textcolor{red}{\small$\big[$YM: #1$\big]$}}

\bibliographystyle{abbrvnat}
\NewDocumentCommand{\possessivecite}{mO{}}{\citeauthor{#1}’s \citeyearpar[#2]{#1}}
\NewDocumentCommand{\Possessivecite}{mO{}}{\Citeauthor{#1}’s \citeyearpar[#2]{#1}}%TODO test

%https://tex.stackexchange.com/a/467188, https://tex.stackexchange.com/a/36088 - uncomment if one of those symbols is used.
%\DeclareFontFamily{U} {MnSymbolD}{}
%\DeclareFontShape{U}{MnSymbolD}{m}{n}{
%  <-6> MnSymbolD5
%  <6-7> MnSymbolD6
%  <7-8> MnSymbolD7
%  <8-9> MnSymbolD8
%  <9-10> MnSymbolD9
%  <10-12> MnSymbolD10
%  <12-> MnSymbolD12}{}
%\DeclareFontShape{U}{MnSymbolD}{b}{n}{
%  <-6> MnSymbolD-Bold5
%  <6-7> MnSymbolD-Bold6
%  <7-8> MnSymbolD-Bold7
%  <8-9> MnSymbolD-Bold8
%  <9-10> MnSymbolD-Bold9
%  <10-12> MnSymbolD-Bold10
%  <12-> MnSymbolD-Bold12}{}
%\DeclareSymbolFont{MnSyD} {U} {MnSymbolD}{m}{n}
%\DeclareMathSymbol{\ntriplesim}{\mathrel}{MnSyD}{126}
%\DeclareMathSymbol{\nlessgtr}{\mathrel}{MnSyD}{192}
%\DeclareMathSymbol{\ngtrless}{\mathrel}{MnSyD}{193}
%\DeclareMathSymbol{\nlesseqgtr}{\mathrel}{MnSyD}{194}
%\DeclareMathSymbol{\ngtreqless}{\mathrel}{MnSyD}{195}
%\DeclareMathSymbol{\nlesseqgtrslant}{\mathrel}{MnSyD}{198}
%\DeclareMathSymbol{\ngtreqlessslant}{\mathrel}{MnSyD}{199}
%\DeclareMathSymbol{\npreccurlyeq}{\mathrel}{MnSyD}{228}
%\DeclareMathSymbol{\nsucccurlyeq}{\mathrel}{MnSyD}{229}
%\DeclareFontFamily{U} {MnSymbolA}{}
%\DeclareFontShape{U}{MnSymbolA}{m}{n}{
%  <-6> MnSymbolA5
%  <6-7> MnSymbolA6
%  <7-8> MnSymbolA7
%  <8-9> MnSymbolA8
%  <9-10> MnSymbolA9
%  <10-12> MnSymbolA10
%  <12-> MnSymbolA12}{}
%\DeclareFontShape{U}{MnSymbolA}{b}{n}{
%  <-6> MnSymbolA-Bold5
%  <6-7> MnSymbolA-Bold6
%  <7-8> MnSymbolA-Bold7
%  <8-9> MnSymbolA-Bold8
%  <9-10> MnSymbolA-Bold9
%  <10-12> MnSymbolA-Bold10
%  <12-> MnSymbolA-Bold12}{}
%\DeclareSymbolFont{MnSyA} {U} {MnSymbolA}{m}{n}
%%Rightwards wave arrow: ↝. Alternative: \rightsquigarrow from amssymb, but it’s uglier
%\DeclareMathSymbol{\rightlsquigarrow}{\mathrel}{MnSyA}{160}

%0394 
\newunicodechar{Δ}{\Delta}
%03B3 Greek Small Letter Gamma
\newunicodechar{γ}{\gamma}
%03B4 Greek Small Letter Delta
\newunicodechar{δ}{\delta}
%2115 Double-Struck Capital N
\newunicodechar{ℕ}{\mathbb{N}}
%211D Double-Struck Capital R
\newunicodechar{ℝ}{\mathbb{R}}
%21CF Rightwards Double Arrow with Stroke
\newunicodechar{⇏}{\nRightarrow}
%21D0 Leftwards Double Arrow
\newunicodechar{⇐}{\ensuremath{\Leftarrow}}
%21D2 Rightwards Double Arrow
\newunicodechar{⇒}{\ensuremath{\Rightarrow}}
%21D4 Left Right Double Arrow
\newunicodechar{⇔}{\Leftrightarrow}
%21DD Rightwards Squiggle Arrow
\newunicodechar{⇝}{\rightsquigarrow}
%2205 Empty Set
\newunicodechar{∅}{\emptyset}
%2212 Minus Sign
\newunicodechar{−}{\ifmmode{-}\else\textminus\fi}
%2227 Logical And
\newunicodechar{∧}{\land}
%2228 Logical Or
\newunicodechar{∨}{\lor}
%2229 Intersection
\newunicodechar{∩}{\cap}
%222A Union
\newunicodechar{∪}{\cup}
%2260 Not Equal To (handy also as text in informal writing)
\newunicodechar{≠}{\ensuremath{\neq}}
%2264 Less-Than or Equal To
\newunicodechar{≤}{\leq}
%2265 Greater-Than or Equal To
\newunicodechar{≥}{\geq}
%2270 Neither Less-Than nor Equal To
\newunicodechar{≰}{\nleq}
%2271 Neither Greater-Than nor Equal To
\newunicodechar{≱}{\ngeq}
%2272 Less-Than or Equivalent To
\newunicodechar{≲}{\lesssim}
%2273 Greater-Than or Equivalent To
\newunicodechar{≳}{\gtrsim}
%2274 Neither Less-Than nor Equivalent To – also, from MnSymbol: \nprecsim, a more exact match to the Unicode symbol; and \npreccurlyeq, too small
\newunicodechar{≴}{\not\preccurlyeq}
%2275 Neither Greater-Than nor Equivalent To
\newunicodechar{≵}{\not\succcurlyeq}
%2279 Neither Greater-Than nor Less-Than – requires MnSymbol; also \nlessgtr from txfonts/pxfonts, \ngtreqless from MnSymbol (but much higher), \ngtrless from MnSymbol (a more exact match to the Unicode symbol); for incomparability (not matching this Unicode symbol), may also consider \ntriplesim from MnSymbol,\nparallelslant from fourier, \between from mathabx, or ⋈
\newunicodechar{≹}{\ngtreqlessslant}
%227A Precedes
\newunicodechar{≺}{\prec}
%227B Succeeds
\newunicodechar{≻}{\succ}
%227C Precedes or Equal To
\newunicodechar{≼}{\preccurlyeq}
%227D Succeeds or Equal To
\newunicodechar{≽}{\succcurlyeq}
%227E Precedes or Equivalent To
\newunicodechar{≾}{\precsim}
%227F Succeeds or Equivalent To
\newunicodechar{≿}{\succsim}
%2280 Does Not Precede
\newunicodechar{⊀}{\nprec}
%2281 Does Not Succeed
\newunicodechar{⊁}{\nsucc}
%2286
\newunicodechar{⊆}{\subseteq}
%22B2 Normal Subgroup Of – using \vartriangleleft from amsfonts, which goes well with \trianglelefteq, \ntriangleright, and so on, also from amsfonts; another possibility is \lhd from latexsym, which seems visually equivalent to \vartriangleleft from amsfonts; latexsym also has ⊴=\unlhd, but doesn’t have a symbol for ⊴. Other related symbols: \triangleleft from latesym package is too small; fdsymbol provides \triangleleft=\medtriangleleft and \vartriangleleft=\smalltriangleleft; MnSymbol provides \medtriangleleft and \vartriangleleft=\lessclosed=\lhd which are smaller than \vartriangleleft from amsfont; \vartriangleleft from mathabx (p. 67), looks different (wider); also \vartriangleleft from boisik (p. 69) looks still different; \vartriangleleft=\lhd from stix are smaller. Oddly enough, \triangleright appears as the LMMathItalic12-Regular font whereas \rhd appears as LASY10 and \vartriangleright appears as MSAM10.
\newunicodechar{⊲}{\vartriangleleft}
%22B3 Contains as Normal Subgroup (also: 25B7 White right-pointing triangle or 25B9 White right-pointing small triangle)
\newunicodechar{⊳}{\vartriangleright}
%22B4 Normal Subgroup of or Equal To
\newunicodechar{⊴}{\trianglelefteq}
%22B5 Contains as Normal Subgroup or Equal To
\newunicodechar{⊵}{\trianglerighteq}
%22C8 Bowtie
\newunicodechar{⋈}{\bowtie}
%22EA Not Normal Subgroup Of
\newunicodechar{⋪}{\ntriangleleft}
%22EB Does Not Contain As Normal Subgroup
\newunicodechar{⋫}{\ntriangleright}
%22EC Not Normal Subgroup of or Equal To
\newunicodechar{⋬}{\ntrianglelefteq}
%22ED Does Not Contain as Normal Subgroup or Equal
\newunicodechar{⋭}{\ntrianglerighteq}
%25A1 White Square
\newunicodechar{□}{\Box}
%27E6 Mathematical Left White Square Bracket – requires stmaryrd (alternative: \text{\textlbrackdbl}, but ugly if used in an italicized text such as a theorem)
\newunicodechar{⟦}{\llbracket}
%27E7 Mathematical Right White Square Bracket
\newunicodechar{⟧}{\rrbracket}
%27FC Long Rightwards Arrow from Bar
\newunicodechar{⟼}{\longmapsto}
%2AB0 Succeeds Above Single-Line Equals Sign
\newunicodechar{⪰}{\succeq}
%301A Left White Square Bracket
\newunicodechar{〚}{\textlbrackdbl}
%301B Right White Square Bracket
\newunicodechar{〛}{\textrbrackdbl}
%→ is defined by default as \textrightarrow, which is invalid in math mode. Same thing for the three other commands. Using \DeclareUnicodeCharacter instead of \newunicodechar because the latter warns about the previous definition.
%← Leftwards Arrow
\DeclareUnicodeCharacter{2190}{\ifmmode\leftarrow\else\textleftarrow\fi}
%→ Rightwards Arrow
\DeclareUnicodeCharacter{2192}{\ifmmode\rightarrow\else\textrightarrow\fi}
%¬ Not Sign
\DeclareUnicodeCharacter{00AC}{\ifmmode\lnot\else\textlnot\fi}
%… Horizontal Ellipsis
\DeclareUnicodeCharacter{2026}{\ifmmode\dots\else\textellipsis\fi}
%× Multiplication Sign
\DeclareUnicodeCharacter{00D7}{\ifmmode\times\else\texttimes\fi}
%Permits to really obtain a straight quote when typing a straight quote; potentially dangerous, see https://tex.stackexchange.com/a/521999
\catcode`\'=\active
\DeclareUnicodeCharacter{0027}{\ifmmode^\prime\else\textquotesingle\fi}


\NewDocumentCommand{\R}{}{ℝ}
\NewDocumentCommand{\N}{}{ℕ}
%\mathscr is rounder than \mathcal.
\NewDocumentCommand{\powerset}{m}{\mathscr{P}(#1)}
%Powerset without zero.
\NewDocumentCommand{\powersetz}{m}{\mathscr{P}^*(#1)}
%https://tex.stackexchange.com/a/45732, works within both \set and \set*, same spacing than \mid (https://tex.stackexchange.com/a/52905).
\NewDocumentCommand{\suchthat}{}{\;\ifnum\currentgrouptype=16 \middle\fi|\;}
%Integer interval.
\NewDocumentCommand{\intvl}{m}{⟦#1⟧}
%Allows for \abs and \abs*, which resizes the delimiters.
\DeclarePairedDelimiter\abs{\lvert}{\rvert}
\DeclarePairedDelimiter\card{\lvert}{\rvert}
\DeclarePairedDelimiter\floor{\lfloor}{\rfloor}
\DeclarePairedDelimiter\ceil{\lceil}{\rceil}
%Perhaps should use U+2016 ‖ DOUBLE VERTICAL LINE here?
\DeclarePairedDelimiter\norm{\lVert}{\rVert}
%From mathtools. Better than using the package braket because braket introduces possibly undesirable space. Then: \begin{equation}\set*{x \in \R^2 \suchthat \norm{x}<5}\end{equation}.
\DeclarePairedDelimiter\set{\{}{\}}
\DeclareMathOperator*{\argmax}{arg\,max}
\DeclareMathOperator*{\argmin}{arg\,min}

%UTR #25: Unicode support for mathematics recommend to use the straight form of phi (by default, given by \phi) rather than the curly one (by default, given by \varphi), and thus use \phi for the mathematical symbol and not \varphi. I however prefer the curly form because the straight form is too easy to mix up with the symbol for empty set.
\let\phi\varphi

%The amssymb solution.
%\NewDocumentCommand{\restr}{mm}{{#1}_{\restriction #2}}
%Another acceptable solution.
%\NewDocumentCommand{\restr}{mm}{{#1|}_{#2}}
%https://tex.stackexchange.com/a/278631; drawback being that sometimes the text collides with the line below.
\NewDocumentCommand\restr{mm}{#1\raisebox{-.5ex}{$|$}_{#2}}


%Decision Theory (MCDA and SC)
\NewDocumentCommand{\allalts}{}{\mathscr{A}}
\NewDocumentCommand{\allcrits}{}{\mathscr{C}}
\NewDocumentCommand{\alts}{}{A}
\NewDocumentCommand{\dm}{}{i}
\NewDocumentCommand{\allF}{}{\mathscr{F}}
\NewDocumentCommand{\allvoters}{}{\mathscr{N}}
\NewDocumentCommand{\voters}{}{N}
\NewDocumentCommand{\allprofs}{}{\linors^{\set{1, 2}}}
\NewDocumentCommand{\fprofs}{}{\mathscr{G}}
\NewDocumentCommand{\prof}{}{P}
\NewDocumentCommand{\ibar}{}{\overline{i}}
\NewDocumentCommand{\lprof}{}{\lambda_P}
\NewDocumentCommand{\lprofi}{O{x}}{\lambda_P(#1)_i}
\NewDocumentCommand{\lprofibar}{O{x}}{\lambda_P(#1)_{\overline{i}}}
\NewDocumentCommand{\ineq}{}{(d \circ \lambda_P)}

\NewDocumentCommand{\linors}{}{\mathcal{L}(\allalts)}
%Thanks to https://tex.stackexchange.com/q/154549
	%\makeatletter
	%\def\@myRgood@#1#2{\mathrel{R^X_{#2}}}
	%\def\myRgood{\@ifnextchar_{\@myRgood@}{\mathrel{R^X}}}
	%\makeatother
\NewDocumentCommand{\pref}{}{\succ}
\NewDocumentCommand{\prefto}{}{\succ^{\mkern-8mu -1}}
\NewDocumentCommand{\preftoeq}{}{\succeq^{\mkern-8mu -1}}
\NewDocumentCommand{\prefi}{O{i}}{\succ_{#1}}
\NewDocumentCommand{\prefiinv}{O{i}}{\prec_{#1}}
\NewDocumentCommand{\PO}{}{\mathit{PO}}
\NewDocumentCommand{\paretopt}{}{\mathit{PO}}
\NewDocumentCommand{\SPPd}{}{\Sigma^\text{PPd}}
\NewDocumentCommand{\SAll}{}{\Sigma^\text{All}}
\NewDocumentCommand{\SThreshold}{}{\Sigma_\text{threshold}}
\NewDocumentCommand{\vpr}{}{\boldsymbol{v}}

\NewDocumentCommand{\musigma}{O{\sigma}O{P}}{\min_{{#1}\circ\lambda_{{#2}}}(A)}
\NewDocumentCommand{\mustar}{O{\sigma}O{P}}{\min_{{#1} \circ \lambda_{#2}} (\paretopt({#2}))}
\NewDocumentCommand{\minineq}{O{\allalts}}{\min_{#1}(d \circ \lambda_P)}
\NewDocumentCommand{\MS}{}{\mathit{MS}}
\NewDocumentCommand{\MSP}{}{\mathit{MS(P)}}
\NewDocumentCommand{\FB}{}{\mathit{FB}}
\NewDocumentCommand{\FBP}{}{\mathit{FB}(P)}
\NewDocumentCommand{\POP}{}{\mathit{PO}(P)}

\NewDocumentCommand{\alllosses}{}{\intvl{0, m-1}^{\set{1, 2}}}

\NewDocumentCommand{\Ptop}{}{\bar{P}}
\NewDocumentCommand{\sigmatop}{}{\bar{\sigma}}

\NewDocumentCommand{\fltwo}{}{\floor{\bar{l_2}}}
\NewDocumentCommand{\bltwo}{}{\bar{l_2}}

\newtheorem{conjecture}{Conjecture}


%I find these settings useful in draft mode. Should be removed for final versions.
	%Which line breaks are chosen: accept worse lines, therefore reducing risk of overfull lines. Default = 200.
		\tolerance=2000
	%Accept overfull hbox up to...
		\hfuzz=2cm
	%Reduces verbosity about the bad line breaks.
		\hbadness 5000
	%Reduces verbosity about the underful vboxes.
		\vbadness=1300

\title{Compromise XPs}
\author[1]{Olivier Cailloux}
\author[2]{Ayça Ebru Giritligil}
\author[2]{Ipek Ozkal Sanver}
\author[1]{Remzi Sanver}
\affil[1]{Université Paris-Dauphine, PSL Research University, CNRS, LAMSADE, 75016 PARIS, FRANCE}
\affil[2]{Bilgi, …}
\hypersetup{
	pdfsubject={Social choice},
	pdfkeywords={axiomatic analysis},
}

\begin{document}
\maketitle

%\begin{abstract}
%	
%\end{abstract}

\section{Introduction}
\label{sec:introduction}
Goal: investigate empirically the conditions in which people adopt a compromise notion based on minimal inequality of losses, as opposed to more classical compromise notions, in two-persons situations.

\section{Basic definitions and notation}
\label{sec:notation}
We have a non empty set of alternatives $\allalts$ with $\card{\allalts} = m$ and a set $N = \set{1, 2}$ of two individuals. 
Given $i \in N$, let $\ibar$ denote the other individual (and conversely, given $\ibar \in N$, let $i$ denote the other individual).
The possible profiles are $\linors^{\set{1, 2}}$. A \ac{SCR} is a function $f: \linors^{\set{1, 2}} → \powersetz{\allalts}$. 

Given $\prof = \set{(i, \prefi), (\ibar, \prefi[\ibar])} \in \allprofs$ and $x \in \allalts$, let $\lprof(x): N → \intvl{0, m - 1}$ associate to each individual $i \in N$ her loss at $x$, defined as the number of alternatives that are strictly preferred to $x$ in her preference $\prefi$, namely, $\lprof(x)_i = \card{\set{y \in \allalts \suchthat y \prefi x}}$, where $\set{y \in \allalts \suchthat y \prefi x}$ designate the upper contour set of $x$.
Let $\min\lprof(x) = \min_{i \in N}{\lprof(x)_i}$, $\max\lprof(x) = \max_{i \in N}{\lprof(x)_i}$ and $\sum \lprof(x) = \sum_{i \in N} \lprof(x)_i$ designate the minimal value, maximal value and sum of the losses of $x$.
Given a loss vector $l \in \alllosses$, define $d(l) = \abs{l_1 - l_2} = \max l - \min l$.

Given $\prof \in \allprofs$, the set of Pareto-optimal alternatives is 
$\POP
% = \set{x \in \allalts \suchthat \forall y \in \allalts \setminus \set{x}: x \prefi[1] y \lor x \prefi[2] y}
 = \set{x \in \allalts \suchthat \forall y \in \allalts, \exists i \in N \suchthat \lprof(x) ≤ \lprof(y)}$.
% = \set{x \in \allalts \suchthat \forall y \in \allalts, i \in N: \lprof(y)_i < \lprof(x)_i ⇒ \lprof(x)_{\ibar} < \lprof(y)_{\ibar}}$.

Note that $\argmin_{\POP} (d \circ \lprof) = \set{x \in \POP \suchthat (d \circ \lprof)(x) = \min_{y \in \POP} (d \circ \lprof)(y)}$ is the least unequal alternatives among the Pareto-optimal ones, according to the measure $d \circ \lprof$.
Also note that $\min_\allalts \max \lprof$ is the threshold that is reached when going from the first (best) rank downwards until some alternative is unanimously considered worth that rank or better. 

We study the “min spread” SCR $\MSP = \argmin_{\POP} (d \circ \lprof)$, which picks the least unequal alternatives among the Pareto-optimal ones, according to the differences of losses; the Fallback-Bargaining SCR \citep{Brams2001} $\FBP = \argmin_\allalts \max \lprof$; and the Borda SCR $B(P) = \argmin \sum \lprof$ which selects the alternatives that minimize the sum of losses.

\commentOC{We should have a look at \doi{10.1287/mnsc.2021.4025}.}

\section{Filtering and classifying profiles}
The results that this section refers to are proven in \cref{sec:proofs}.

\subsection{Filtering profiles}
We are interested in selecting profiles on which $\FB$ and $\MS$ disagree. 
As a minimal requirement, we thus request $\FBP \cap \MSP = \emptyset$. 
This implies $\card{\FBP} = \card{\MSP} = 1$ (\cref{th:equiv}).

We also require $\prof$ to contain no Pareto-dominated alternative with a smaller spread than $\minineq[\POP]$.
This constraint stems from a desire to focus on the subject of our study, and not perturb the subject by letting alternatives that do not distinguish our rules of interest appear possibly attractive. Studying the attractivity of Pareto-dominated alternatives is an interesting goal but requires its own design.

Let $\fprofs = \set{\prof \in \allprofs \suchthat \FBP \cap \MSP = \emptyset \land \forall a \in \allalts: (d \circ \lprof)(a) ≥ \minineq[\POP]}$ designate the satisfactory profiles.
Given $\prof \in \fprofs$ (thus with $\card{\FBP} = \card{\MSP} = 1$), we write $\set{x} = \FBP$ and $\set{y} = \MSP$.

During a run with a given subject, we want the spread of the Fallback-Bargaining winner, $d(\lprof(x))$, to reach high enough values compared to $d(\lprof(y))$, in order to ensure a sufficient contrast with min spread. This requires a big enough value for $m$ as $\delta = d(\lprof(x)) - d(\lprof(y)) ≤ \frac{m - 5}{2}$ (\cref{th:delta}).
We also want $m$ to be small enough for cognitive simplicity. 
As a compromise, we opt for $m = 13$, which permits to reach $\delta = 4$.

\subsection{Classifying profiles}
\label{sec:classifying}
To each profile $\prof \in \fprofs$ can be associated these properties;:
\begin{enumerate}
	\item the losses of $x$ and $y$, $\lprof(x)$ and $\lprof(y)$;
	\item whether $\FBP = \BP = \set{x}$, $\FBP \subset \BP$, or $\FBP \cap \BP = \emptyset$.
\end{enumerate}
Accordingly, we divide $\fprofs$ into classes of profiles and use a representative profile for each such class.

Let $T = \set{t \in \intvl{0, m - 1}^4 \suchthat t_1 < t_2 < t_3 < t_4 \land t_1 + t_4 + 1 ≤ t_2 + t_3 \land t_3 + t_4 ≤ m}$ denote the set of tuples of loss values that satisfy the specified constraints.
Given $t \in T$ and $i \in N$, define the class $C_{t, i} = \set{\prof \in \fprofs \suchthat \lprof(x) = \set{(i, t_1), (\ibar, t_3)} \land \lprof(y) = \set{(i, t_4), (\ibar, t_2)}}$ as the set of profiles where $x$ and $y$ have the specified losses. 
Define the class $C_t = \bigcup_{i \in N} C_{t, i}$ as the set of profiles where $x$ has the loss values $(t_1, t_3)$ and $y$ has the loss values $(t_2, t_4)$ associated both with individuals $(i, \ibar)$ or both with $(\ibar, i)$. 
We consider the classes $\set{C_t, t \in T}$. 
This set of classes forms a complete cover of $\fprofs$ (\cref{th:cover}).

Within $\fprofs$, the Borda winner always differs from the Min spread winner (\cref{th:BMS}); and
here are the possible relationships between $\BP$ and $\FBP = \set{x}$ (\cref{rk:BFB}):
\begin{itemize}
	\item if $t_1 = \min \lprof(x) = 0$, no profile from $\fprofs$ distinguish the Borda winner and the FB winner, formally, $\forall t \in T, \prof \in C_t: t_1 = 0 ⇒ \BP = \FBP$;
	\item if $t_1 = \min \lprof(x) = 1$, $x \in \BP$, with both $\FBP = \BP$ and $\FBP \subset \BP$ possible;
	\item if $t_1 = \min \lprof(x) ≥ 2$, both $\FBP = \BP$ and $\FBP \cap \BP = \emptyset$ are possible.
\end{itemize}

\subsection{Enumerating the classes of profiles}
Let us now enumerate exhaustively the classes of profiles in $\fprofs$.
When $\max \lprof(y) = 8$, only the class with $\min \lprof(x) = 0$, $\min \lprof(y) = 4$, $\max \lprof(x) = 5$ exists in $\fprofs$.
When $\max \lprof(y) ≠ 8$, the following constraints exhaust the possibilities (\cref{th:bounds138,th:bounds137}).
\begin{itemize}
	\item $4 ≤ \max \lprof(y) ≤ 7$;
	\item $2 ≤ \min \lprof(y) ≤ \max \lprof(y) - 2$;
	\item $\max \set{\min \lprof(y) + 1, \max \lprof(y) - \min \lprof(y) + 1} ≤ \max \lprof(x) ≤ \max \lprof(y) - 1$;
	\item $0 ≤ \min \lprof(x) ≤ \max \lprof(x) - (\max \lprof(y) - \min \lprof(y) + 1)$.
\end{itemize}

\begin{table}
	\begin{tabular}{*{4}C}
		\toprule
		\max \lprof(y) & \min \lprof(y) & \max \lprof(x) \text{ range} & \min \lprof(x) \text{ range} \\
		\midrule 
		8	& 4	& \set{5}	& \set{0}\\
		7	& 5	& \set{6}	& [0, 3]\\
			& 4	& [5, 6]	& [0, \max \lprof(x) - 4]\\
			& 3	& [5, 6]	& [0, \max \lprof(x) - 5]\\
			& 2	& \set{6}	& \set{0}\\
		6	& 4	& \set{5}	& [0, 2]\\
			& 3	& [4, 5]	& [0, \max \lprof(x) - 4]\\
			& 2	& \set{5}	& \set{0}\\
		5	& 3	& \set{4}	& [0, 1]\\
			& 2	& \set{4}	& \set{0}\\
		4	& 2	& \set{3}	& \set{0}\\
		\bottomrule
	\end{tabular}
	\caption{Possible ranges for $m = 13$.}
	\label{fig:m13ranges}
\end{table}

\begin{table}
	\small
	\begin{tabular}{CCCCCCCCC}
		\toprule
		\text{id} & \min \lprof(x) & \min \lprof(y) & \max \lprof(x) & \max \lprof(y) & d(\lprof(x)) & d(\lprof(y)) & \delta \\
		\midrule 
		\csvreader[late after line = \\]{Profiles.csv}%
		{dlPx = \dlpx, dlPy = \dlpy, max lPx = \maxlpx, max lPy = \maxlpy, min lPx = \minlpx, avg lPx = \avglPx, min lPy = \minlpy, avg lPy = \avglpy, e = \cole, top = \coltop, dmin = \dmin, davg = \davg, delta = \coldelta, cl coarse = \clcoarse, subclass = \subclass}{%
			\thecsvrow & \minlpx & \minlpy & \maxlpx & \maxlpy & \dlpx & \dlpy & \coldelta
		}% 
		\bottomrule
	\end{tabular}
	\caption{Possible values for $m = 13$.}
	\label{fig:m13}
\end{table}

\Cref{fig:m13ranges} lists the possible ranges for these variables, and \cref{fig:m13} lists all possible values.

%Generated – please do not edit.
\newcounter{ct:profclassid}
\refstepcounter{ct:profclassid} \label{profclassid:0657}
\refstepcounter{ct:profclassid} \label{profclassid:1657}
\refstepcounter{ct:profclassid} \label{profclassid:0546}
\refstepcounter{ct:profclassid} \label{profclassid:0647}
\refstepcounter{ct:profclassid} \label{profclassid:2657}
\refstepcounter{ct:profclassid} \label{profclassid:1546}
\refstepcounter{ct:profclassid} \label{profclassid:0435}
\refstepcounter{ct:profclassid} \label{profclassid:1647}
\refstepcounter{ct:profclassid} \label{profclassid:0536}
\refstepcounter{ct:profclassid} \label{profclassid:0637}
\refstepcounter{ct:profclassid} \label{profclassid:0547}
\refstepcounter{ct:profclassid} \label{profclassid:3657}
\refstepcounter{ct:profclassid} \label{profclassid:2546}
\refstepcounter{ct:profclassid} \label{profclassid:1435}
\refstepcounter{ct:profclassid} \label{profclassid:0324}
\refstepcounter{ct:profclassid} \label{profclassid:2647}
\refstepcounter{ct:profclassid} \label{profclassid:1536}
\refstepcounter{ct:profclassid} \label{profclassid:0425}
\refstepcounter{ct:profclassid} \label{profclassid:1637}
\refstepcounter{ct:profclassid} \label{profclassid:0526}
\refstepcounter{ct:profclassid} \label{profclassid:0627}
\refstepcounter{ct:profclassid} \label{profclassid:1547}
\refstepcounter{ct:profclassid} \label{profclassid:0436}
\refstepcounter{ct:profclassid} \label{profclassid:0537}
\refstepcounter{ct:profclassid} \label{profclassid:0548}
\crefname{ct:profclassid}{project}{projects}

\section{Protocol}
\subsection{Questions}
Here are some questions we want to investigate.
Ceteris Paribus…
\begin{enumerate}
	\item Does high $\delta$ favor MS?
	\begin{itemize}
		\item 0324 (class \ref{profclassid:0324}, \cref{ex:0324B}) VS 0657 (class \ref{profclassid:0657}, \cref{ex:0657B})
	\end{itemize}
	\item Does it matter for the FB winner to be top-ranked by a voter? 
	\begin{itemize}
		\item 0324 (class \ref{profclassid:0324}, \cref{ex:0324B}) VS 1435 with $\set{x} \subset \BP$ (class \ref{profclassid:1435}, \cref{ex:1435D})
	\end{itemize}
	\item Does the best position of the FB winner matter?
	\begin{itemize}
		\item 0324 (class \ref{profclassid:0324}, \cref{ex:0324B}) VS 3657 with $\BP = \set{x}$ (class \ref{profclassid:3657}, \cref{ex:3657B})
	\end{itemize}
	\item Does $FB(P) = B(P)$ favor $FB$?
	\begin{itemize}
		\item 3657 with $\BP = \set{x}$ (class \ref{profclassid:3657}, \cref{ex:3657B}) VS 3657 with $\BP \cap \FBP = \emptyset$ (class \ref{profclassid:3657}, \cref{ex:3657D})
	\end{itemize}
	\item Does $d(\lprof(y))$ matter, fixing $\delta$?
	\begin{itemize}
		\item 0627 (class \ref{profclassid:0627}, \cref{ex:0627B}) VS 0324 (class \ref{profclassid:0324}, \cref{ex:0324B})
	\end{itemize}
	\item Do people select Borda rather than FB or MS?
	\begin{itemize}
		\item 3657 with $\BP \cap \FBP = \emptyset$ (class \ref{profclassid:3657}, \cref{ex:3657D}), 1435 with $\set{x} \subset \BP$ (class \ref{profclassid:1435}, \cref{ex:1435D})
	\end{itemize}
	\item Does showing profiles in favor of MS first favor MS choices in other profiles? This will be tested ex-post, the order of presentation being equiprobable across every permutation.
	How much a profile is MS-favoring could be determined using these criteria, lexicographically.
	\begin{itemize}
		\item high $\delta$
		\item small $\delta^B = \sum \lprof(y) - \sum \lprof(x)$ (the difference between the Borda score of $x$ and $y$, or equivalently, the difference between the avg loss of $x$ and $y$)
		\item small $d(\lprof(y))$
		\item high $\min \lprof(x)$
	\end{itemize}
	\item Does using the word “compromise” in an abstract scenario trigger a different behavior than using a concrete scenario with no such wording?
\end{enumerate}
As a consequence, we pick the following set of profiles:
\begin{enumerate}
	\item 0324 (class \ref{profclassid:0324}, \cref{ex:0324B}), 
	\item 0657 (class \ref{profclassid:0657}, \cref{ex:0657B}), 
	\item 1435 with $\set{x} \subset \BP$ (class \ref{profclassid:1435}, \cref{ex:1435D}), 
	\item 3657 with $\BP = \set{x}$ (class \ref{profclassid:3657}, \cref{ex:3657B}), 
	\item 3657 with $\BP \cap \FBP = \emptyset$ (class \ref{profclassid:3657}, \cref{ex:3657D}), 
	\item 0627 (class \ref{profclassid:0627}, \cref{ex:0627B}).
\end{enumerate}

\subsection{A proposal for a protocol}
We run two settings (each subject is exposed to exactly one of these settings).
One is an abstract setting and uses the word uzlaşı for compromise. At end of the experiment we could ask the subject how she defines uzlaşı (free form text).
Another setting is a concrete setting that does not use any word related to compromise (the understanding that we need a compromise comes from the description of the situation). Choosing which DVD should be watched (by two friends), or which movie they should go to, or pizza to eat, seems suitable.
\commentOC{We also mentioned: choosing a restaurant for your twins (but this involves children so highlights paternalism), or choosing a honey moon package (but this may introduce gender specific consideration).}

We do not pinpoint $x$ or $y$: subjects can choose any alternative. (This is still under discussion; the advantage is that it permits to check that they agree with our notion of compromise and are paying close attention, and it lets them pick Borda winners if they want to, so I view it as a more ambitious version which allows to claim that we capture the way people conceive of a compromise; the problem is that we can only do it with sufficient sample as it will increase the noise.)

Pick the classes of profiles indicated above and repeat the fourth one at the end of the sequence, to check for consistency. Pick a random permutation to order the profiles, and pick a random permutation to rename alternatives, for each choice we present.

We ask for how they choose, their rationale, for each choice we present. Subjects answer using free text and must provide some text before continuing.
 
%\subsection{An example run}
%To be modified, depending on our choice of profiles…
%%Generated – please do not edit.

\begin{equation}
  \begin{array}{*{13}c}
    b&d&j&f&k&g&h&i&m&e&a&c&l\\
    g&h&k&b&i&m&e&a&c&l&d&j&f
  \end{array}
\end{equation}
\begin{equation}
  \begin{array}{*{13}c}
    e&f&k&m&h&g&d&i&j&c&l&a&b\\
    i&j&d&c&l&e&a&b&f&k&m&h&g
  \end{array}
\end{equation}
\begin{equation}
  \begin{array}{*{13}c}
    f&b&i&k&c&g&a&d&h&l&m&e&j\\
    h&l&m&e&j&d&f&b&i&k&c&g&a
  \end{array}
\end{equation}
\begin{equation}
  \begin{array}{*{13}c}
    m&j&h&a&d&l&e&k&b&g&c&i&f\\
    e&k&b&l&j&g&c&i&f&m&h&a&d
  \end{array}
\end{equation}
\begin{equation}
  \begin{array}{*{13}c}
    i&f&l&j&e&a&c&k&g&h&b&m&d\\
    g&h&b&m&d&k&j&i&f&l&e&a&c
  \end{array}
\end{equation}
\begin{equation}
  \begin{array}{*{13}c}
    b&e&m&k&i&d&l&j&c&g&f&a&h\\
    j&c&g&f&l&b&a&h&e&m&k&i&d
  \end{array}
\end{equation}
\begin{equation}
  \begin{array}{*{13}c}
    a&i&e&f&d&h&l&j&g&b&m&k&c\\
    g&b&m&k&j&c&a&i&e&f&d&h&l
  \end{array}
\end{equation}
\begin{equation}
  \begin{array}{*{13}c}
    h&d&g&f&m&b&l&j&c&a&i&e&k\\
    j&c&l&a&i&h&e&k&d&g&f&m&b
  \end{array}
\end{equation}

\bibliography{biblio}

\appendix
\section{Formal statements}
\label{sec:proofs}

\subsection{General results}
\begin{theorem}
	\label{th:maxFB}
	$\forall \prof \in \allprofs: \min \max \lprof ≤ \frac{m}{2}$.
\end{theorem}
\begin{proof}
	Define $A = \set{a \in \allalts \suchthat 0 ≤ \lprof(a)_1 ≤ \khalf}$.
	Define $B = \set{a \in \allalts \suchthat \khalf < \lprof(a)_2 ≤ m - 1}$.
	Observe that $\card{A} = \khalf + 1$ and $\card{B} = (m - 1) - \khalf = (m - \khalf) - 1 = \ceil{\frac{m}{2}} - 1 ≤ \khalf$, thus, $\card{B} < \card{A}$.
	Thus, $\exists x \in A \setminus B$.
	Thus, $\lprof(x)_1 ≤ \khalf$ (as $x \in A$) and $\lprof(x)_2 ≤ \khalf$ (as $x \notin B$).
	It follows that $\max \lprof(x) ≤ \khalf$.
\end{proof}

\begin{theorem}
	\label{th:FBPO}
	$\forall \prof \in \allprofs: \argmin \max \lprof \subseteq \POP$.
\end{theorem}
\begin{proof}
	Consider any $x \in \argmin \max \lprof$ and any $y \in \allalts$.
	
	Observe that $\max \lprof(x) ≤ \max \lprof(y)$, equivalently, $\forall i \in N: \lprof(x)_i ≤ \max \lprof(y)$.
	Thus, given any $i \in \argmax \lprof(y)$, $\lprof(x)_i ≤ \lprof(y)_i$.
\end{proof}

\subsection{Restraining profiles}
\begin{theorem}
	\label{th:equiv}
	The following propositions are equivalent:
	\begin{enumerate}
		\item \label{it:noInters} $\FBP \cap \MSP = \emptyset$;
		\item \label{it:order} $\forall y \in \MSP, x \in \FBP, i \in \argmax \lprof(y): \min \lprof(x) = \lprof(x)_i < \min \lprof(y) = \lprof(y)_{\ibar} < \max \lprof(x) = \lprof(x)_{\ibar} < \max \lprof(y) = \lprof(y)_i$.
	\end{enumerate}
	Furthermore, they imply:
	\begin{enumerate}[label=({\roman*}), ref={\roman*}]
		\item \label{it:singFBP} $\card{\FBP} = 1$;
		\item \label{it:singMSP} $\card{\MSP = 1}$;
		\item \label{it:ourMax} $\min \max \lprof ≤ \frac{m - 1}{2}$.
	\end{enumerate}
\end{theorem}
\begin{proof}
	Assuming \cref{it:noInters}, let us prove \cref{it:order}.
	Pick any $x \in \FBP = \argmin \max \lprof$ and any $y \in \MSP = \argmin_{\POP} (d \circ \lprof)$.
	Thanks to \cref{it:noInters}, we know that $y \notin \FBP$ thus $\max \lprof(x) < \max \lprof(y)$;
	also, $x \in \POP$ (as given by \cref{th:FBPO}), thus $d(\lprof(y)) ≤ d(\lprof(x))$.
	
	We see that $\min \lprof(x) < \min \lprof(y)$ as otherwise $\min \lprof(y) ≤ \min \lprof(x) ≤ \max \lprof(x) < \max \lprof(y)$ which yields $d(\lprof(x)) < d(\lprof(y))$.
	
	Pick any $i \in \argmax \lprof(y)$.
	We can see that $\min \lprof(y) = \lprof(y)_{\ibar} < \lprof(x)_{\ibar}$. That is because $\max \lprof(x) < \max \lprof(y) = \lprof(y)_i$ thus $\lprof(x)_i ≤ \max \lprof(x) < \lprof(y)_i$ and as $y \in \POP$, we need to have $\lprof(y)_{\ibar} < \lprof(x)_{\ibar}$.

	So far we know that $\min \lprof(x) < \lprof(y)_{\ibar} < \lprof(x)_{\ibar}$ and $\max \lprof(x) < \lprof(y)_i$, which also permit to deduce that $\lprof(x)_i = \min \lprof(x)$ and $\lprof(x)_{\ibar} = \max \lprof(x)$.
	This establishes \cref{it:order}.
	
	In turn, \cref{it:order} implies \cref{it:noInters} as, picking any $y \in \MSP$, we see that $y \in \FBP$ is excluded by picking any $x \in \FBP$ and using $\max \lprof(x) < \max \lprof(y)$.
	
	We now turn to the consequences of these claims.
	Pick any $x \in \FBP$ and $y \in \MSP$ and $i \in \argmax \lprof(y)$. Thus, from \cref{it:order}, $\max \lprof(x) = \lprof(x)_{\ibar}$.
	
	Considering any $x' \in \FBP$, from \cref{it:order} again, $\max \lprof(x') = \lprof(x')_{\ibar}$. As $x, x' \in \FBP$, we also have $\max \lprof(x) = \max \lprof(x')$. Thus, $\lprof(x')_{\ibar} = \lprof(x)_{\ibar}$ and $x = x'$.
	
	A similar argument establishes \cref{it:singMSP}.
	
	Define $A = \set{a \in \allalts \suchthat \max \lprof(x) < \lprof(a)_i ≤ m - 1}$,
	$B = \set{a \in \allalts \suchthat 0 ≤ \lprof(a)_{\ibar} < \max \lprof(x)}$.
	As $\card{\FBP} = 1$ and $x \notin B$, $a \in B ⇒ a \notin \FBP$, whence $a \in B ⇒ \max \lprof(x) < \max \lprof(a) ⇒ \max \lprof(x) < \lprof(a)_i$ (because $a \in B ⇒ \lprof(a)_{\ibar} < \max \lprof(x)$), thus $B \subseteq A$.
	It follows that $\max \lprof(x) = \card{B} ≤ \card{A} = m - 1 - \max \lprof(x)$, whence $\max \lprof(x) ≤ \frac{m - 1}{2}$.
\end{proof}
Comparing \cref{th:equiv} \cref{it:ourMax} and \cref{th:FBPO}, we see that the hypothesis $\FBP \cap \MSP = \emptyset$ lead to a slightly lower bound for $\min \max \lprof$.

From now on, each time we deal with profiles such that \cref{th:equiv} \cref{it:noInters} or \ref{it:order} holds, we will therefore consider, without explicitly invoking \cref{th:equiv}, that $\card{\FBP} = \card{\MSP} = 1$, thus, we will consider these sets as singletons, and similarly for the sets $\argmin \lprof(\FBP)$, $\argmax \lprof(\FBP)$, $\argmin \lprof(\MSP)$ and $\argmax \lprof(\MSP)$.

\subsection{Bounds on losses}
\begin{theorem}
	\label{th:sufficientBounds}
	$\forall \prof \in \allprofs \suchthat \FBP \cap \MSP = \emptyset$, letting $\set{x} = \FBP$, $\set{y} = \MSP$ and $\set{i} = \argmax \lprof(y)$, and defining $t_1 = \min \lprof(x)$, $t_2 = \min \lprof(y)$, $t_3 = \max \lprof(x)$ and $t_4 = \max \lprof(y)$:
	\begin{enumerate}
		\item \label{it:suffOrder} $t_1 = \lprof(x)_i < t_2 = \lprof(y)_{\ibar} < t_3 = \lprof(x)_{\ibar} < t_4 = \lprof(y)_i$;
		\item \label{it:suff1423} $t_1 + t_4 + 1 ≤ t_2 + t_3$;
		\item \label{it:suff34m} $t_3 + t_4 ≤ m$.
	\end{enumerate}
\end{theorem}
\begin{proof}
	\Cref{it:suffOrder} comes directly from \cref{th:equiv} \cref{it:order}.
	
	\Cref{it:suff1423} holds as $d(\lprof(y)) < d(\lprof(x))$, equivalently, $\max \lprof(y) - \min \lprof(y) < \max \lprof(x) - \min \lprof(x)$, thus, $t_4 - t_2 < t_3 - t_1$.
	
	To prove \cref{it:suff34m}, define $A = \set{a \in \allalts \suchthat 0 ≤ \lprof(a)_i < t_4}$ and $B = \set{a \in \allalts \suchthat t_3 ≤ \lprof(a)_{\ibar} ≤ m - 1}$.
	Pick $a \in A$. 
	As $\lprof(a)_i < t_4 = \lprof(y)_i$, $a ≠ y$. 
	As $y \in \POP$, $\lprof(y)_i < \lprof(a)_i \lor \lprof(y)_{\ibar} < \lprof(a)_{\ibar}$, thus $t_2 = \lprof(y)_{\ibar} < \lprof(a)_{\ibar}$.

	Now assume that $a \in A \cup \POP$. 
	As $t_3 = \max \lprof(x) = \min \max \lprof$, $t_3 ≤ \max \lprof(a)$. 
	Thus, if $\lprof(a)_i < t_3$ then $t_3 ≤ \lprof(a)_{\ibar}$.
	Otherwise, $t_3 ≤ \lprof(a)_i$, which also leads to $t_3 ≤ \lprof(a)_{\ibar}$ because otherwise $t_2 < t_3 ≤ \lprof(a)_i < t_4$ (as $a \in A$) and $t_2 < \lprof(a)_{\ibar} < t_3 < t_4$, implying $d(\lprof(a)) < t_4 - t_2 = d(\lprof(y))$, contradicting $y \in \MSP$.
	We conclude that $a \in A \cup \POP ⇒ t_3 ≤ \lprof(a)_{\ibar}$.
	
	Consider now $a \in A$, $a \notin \POP$.
	Thus $\exists a' \in \POP \suchthat \lprof(a')_i < \lprof(a)_i \land \lprof(a')_{\ibar} < \lprof(a)_{\ibar}$. Thus, $\lprof(a')_i < \lprof(a)_i < t_4$, whence $a' \in A$, which implies $t_3 ≤ \lprof(a')_{\ibar}$ and thus, again, $t_3 ≤ \lprof(a)_{\ibar}$.

	We obtain $A \subseteq B$.
	But $\card{A} = t_4$ and $\card{B} = m - t_3$, thus establishing \cref{it:suff34m}.
\end{proof}

Let $T = \set{t \in \intvl{0, m - 1}^4 \suchthat t_1 < t_2 < t_3 < t_4 \land t_1 + t_4 + 1 ≤ t_2 + t_3 \land t_3 + t_4 ≤ m}$ be defined as in \cref{sec:classifying}.
We now state a canonical rewriting of these inequalities, and several consequences, that will reveal useful.
\begin{theorem}
	\label{th:consequentBounds}
	$\forall t \in \intvl{0, m - 1}^4$, $t \in T$ iff all the following hold:
	\begin{enumerate}
		\item \label{it:torder} $0 ≤ t_1 < t_2 < t_3 < t_4$ (thus $t_1 ≤ t_2 - 1$, $t_2 ≤ t_3 - 1$, $t_3 ≤ t_4 - 1$);
		\item \label{it:t1423} $t_1 + t_4 + 1 ≤ t_2 + t_3$;
		\item \label{it:t34m} $t_3 + t_4 ≤ m$.
	\end{enumerate}
	Equivalently:		
	\begin{enumerate}[start=4]
		\item \label{it:0t1} $0 ≤ t_1$;
		\item \label{it:t32} $0 ≤ t_3 - t_2 - 1$;
		\item \label{it:t43} $0 ≤ t_4 - t_3 - 1$;
		\item \label{it:t2143} $0 ≤ t_2 - t_1 - t_4 + t_3 - 1$;
		\item \label{it:tm34} $0 ≤ m - t_3 - t_4$.
	\end{enumerate}
	In turn, these imply:		
	\begin{enumerate}[start=9]
		\item \label{it:t4231} $t_4 - t_2 ≤ t_3 - t_1 - 1$;
		\item \label{it:t243} $0 ≤ t_2 - t_4 + t_3 - 1$;
		\item \label{it:t12} $0 ≤ t_2 - t_1 - 2$;
		\item \label{it:t22} $0 ≤ t_2 - 2$;
		\item \label{it:t42} $0 ≤ t_4 - t_2 - 2$;
		\item \label{it:t142} $0 ≤ t_1 + t_4 - t_2 - 2$;
		\item \label{it:t414} $0 ≤ t_4 - t_1 - 4$;
		\item \label{it:t44} $0 ≤ t_4 - 4$;
		\item \label{it:t1m42} $0 ≤ m - 1 - 2 t_4 + t_2 - t_1$;
		\item \label{it:tm242} $0 ≤ m - 1 - 2 t_4 + t_2$;
		\item \label{it:t2m} $0 ≤ 2 m - 3 t_4 - 2$;
		\item \label{it:tm3} $0 ≤ m - 2 t_3 - 1$ (that is, $0 ≤ \frac{m - 1}{2} - t_3$);
		\item \label{it:tm2} $0 ≤ \frac{m - 3}{2} - t_2$;
		\item \label{it:t24m} $0 ≤ t_2 - t_4 + \frac{m - 3}{2}$;
		\item \label{it:tm42} $0 ≤ m - 1 - t_4 - t_2$;
		\item \label{it:tm3142} $0 ≤ \frac{m - 5}{2} - t_3 + t_1 + t_4 - t_2$;
		\item \label{it:t13m} $t_1 + t_3 ≤ m - 4$.
	\end{enumerate}
\end{theorem}
\begin{proof}
	\Cref{it:torder,it:t1423,it:t34m} are together equivalent to $t \in T$ by definition of $T$. 
	To see that they imply \cref{it:0t1,it:t32,it:t43,it:t2143,it:tm34}, observe that \cref{it:0t1,it:t32,it:t43} come from \cref{it:torder}; \cref{it:t2143} is equivalent to \cref{it:t1423}; and \cref{it:tm34} is equivalent to \cref{it:t34m}.
	We now show that \crefrange{it:0t1}{it:tm34} together imply \crefrange{it:t4231}{it:t13m}.
	\begin{itemize}
		\item \Cref{it:t4231} rearranges \cref{it:t2143}.
		\item Adding \cref{it:0t1,it:t2143} yields \cref{it:t243}.
		\item Adding \cref{it:t43,it:t2143} yields \cref{it:t12}.
		\item Adding \cref{it:0t1,it:t12} yields \cref{it:t22}.
		\item Adding \cref{it:t32,it:t43} yields \cref{it:t42}.
		\item Adding \cref{it:0t1,it:t42} yields \cref{it:t142}.
		\item Adding \cref{it:t42,it:t12} yields \cref{it:t414}.
		\item Adding \cref{it:0t1,it:t414} yields \cref{it:t44}.
		\item Adding \cref{it:t2143,it:tm34} yields \cref{it:t1m42}.
		\item Adding \cref{it:0t1,it:t1m42} yields \cref{it:tm242}.
		\item Adding \crefns{it:t1m42,it:0t1,it:t32,it:tm34} yields \cref{it:t2m}.
		\item Adding \cref{it:t43,it:tm34} yields \cref{it:tm3}.
		\item Adding \cref{it:t32,it:tm3} yields \cref{it:tm2}.
		\item Adding \cref{it:0t1,it:t2143,it:tm3} yields \cref{it:t24m}.
		\item Adding \cref{it:t32,it:tm34} yields \cref{it:tm42}.
		\item Adding \cref{it:t142,it:tm3} yields \cref{it:tm3142}.
		\item Finally, adding \cref{it:tm34,it:t414} yields \cref{it:t13m}.
	\end{itemize}
	To end the proof, observe that \crefrange{it:0t1}{it:tm34} (which imply \crefrange{it:t4231}{it:t13m}) imply \cref{it:torder,it:t1423,it:t34m}: \crefns{it:0t1,it:t12,it:t32,it:t43} together yield \cref{it:torder}; and \cref{it:t2143,it:tm34} are respectively equivalent to \cref{it:t1423,it:t34m}.
\end{proof}

The following corollary states how to list exhaustively every tuple belonging to $t$, starting from picking an appropriate value for $t_4$, then picking an appropriate value for $t_2$ (whose bounds thus depend on the choice of $t_4$), then proceeding similarly for $t_3$ and finally $t_1$.
\begin{corollary}
	\label{th:boundsEquiv}
	$\forall t \in \intvl{0, m - 1}^4$, $t \in T$ iff all the following hold:
	\begin{enumerate}[label=({\roman*}), ref={\roman*}]
		\item \label{it:t4s} $4 ≤ t_4 ≤ \frac{2 m - 2}{3}$;
		\item \label{it:t2s} $\max \set{2, t_4 - \frac{m - 3}{2}, 2 t_4 - m + 1} ≤ t_2 ≤ \min \set{m - 1 - t_4, t_4 - 2, \frac{m - 3}{2}}$;
		\item \label{it:t3s} $\max \set{t_2 + 1, t_4 - t_2 + 1} ≤ t_3 ≤ \min \set{\frac{m - 1}{2}, t_4 - 1, m - t_4}$;
		\item \label{it:t1s} $0 ≤ t_1 ≤ t_3 - (t_4 - t_2 + 1)$.
	\end{enumerate}
\end{corollary}
\begin{proof}
	Given $t \in T$, 
	\cref{it:t4s} follows from \cref{th:consequentBounds} \cref{it:t44,it:t2m};
	\cref{it:t2s} from \crefns{it:t22,it:t24m,it:tm242,it:tm42,it:t42,it:tm2};
	\cref{it:t3s} from \crefns{it:t32,it:t243,it:tm3,it:t43,it:tm34}
	and \cref{it:t1s} from \cref{it:0t1,it:t2143}.
	
	For the backwards direction, \cref{it:t1s} implies \cref{th:consequentBounds} \cref{it:0t1,it:t2143}
	and \cref{it:t3s} implies \crefns{it:t32,it:t43,it:tm34}, thus (still using \cref{th:consequentBounds}) $t \in T$.
\end{proof}

\begin{corollary}
	\label{th:delta}
	$\forall \prof \in \allprofs \suchthat \FBP \cap \MSP = \emptyset$, letting $\set{x} = \FBP$ and $\set{y} = \MSP$:
	\begin{equation}
		1 ≤ d(\lprof(x)) - d(\lprof(y)) ≤ \frac{m - 5}{2}.
	\end{equation}
\end{corollary}
\begin{proof}
	Define $t_1 = \min \lprof(x)$, $t_2 = \min \lprof(y)$, $t_3 = \max \lprof(x)$ and $t_4 = \max \lprof(y)$ and apply \cref{th:sufficientBounds} and \cref{th:consequentBounds} \cref{it:t2143,it:tm3142} to get $1 ≤ t_3 - t_1 - t_4 + t_2 ≤ \frac{m - 5}{2}$.
\end{proof}

Let us write the values of the bounds for $m = 13$, as adopted for our experiment.
\begin{corollary}
	\label{th:bounds13}
	Given $m = 13$, $\forall t \in \intvl{0, m - 1}^4$, $t \in T$ iff all the following hold:
	\begin{enumerate}[label=({\roman*}), ref={\roman*}]
		\item $4 ≤ t_4 ≤ 8$;
		\item $\max \set{2, t_4 - 5, 2 t_4 - 12} ≤ t_2 ≤ \min \set{12 - t_4, t_4 - 2, 5}$;
		\item $\max \set{t_2 + 1, t_4 - t_2 + 1} ≤ t_3 ≤ \min \set{6, t_4 - 1, 13 - t_4}$;
		\item $0 ≤ t_1 ≤ t_3 - (t_4 - t_2 + 1)$.
	\end{enumerate}
\end{corollary}

These bounds can be simplified by considering separately the cases $t_4 = 8$ and $t_4 ≤ 7$.

\begin{corollary}
	\label{th:bounds138}
	Given $m = 13$, $\forall t \in \intvl{0, m - 1}^4 \suchthat t_4 = 8$, $t \in T$ iff all the following hold:
	\begin{enumerate}[label=({\roman*}), ref={\roman*}]
		\item $t_2 = 4$;
		\item $t_3 = 5$;
		\item $t_1 = 0$.
	\end{enumerate}
\end{corollary}

\begin{corollary}
	\label{th:bounds137}
	Given $m = 13$, $\forall t \in \intvl{0, m - 1}^4 \suchthat t_4 ≤ 7$, $t \in T$ iff all the following hold:
	\begin{enumerate}[label=({\roman*}), ref={\roman*}]
		\item $4 ≤ t_4 ≤ 7$;
		\item $2 ≤ t_2 ≤ t_4 - 2$;
		\item $\max \set{t_2 + 1, t_4 - t_2 + 1} ≤ t_3 ≤ t_4 - 1$;
		\item $0 ≤ t_1 ≤ t_3 - (t_4 - t_2 + 1)$.
	\end{enumerate}
\end{corollary}

\subsection{Classifying profiles}
Let $C_{t, i} = \set{\prof \in \fprofs \suchthat \lprof(\FBP) = \set{(i, t_1), (\ibar, t_3)} \land \lprof(\MSP) = \set{(i, t_4), (\ibar, t_2)}}$ and $C_t = \bigcup_{i \in N} C_{t, i}$ be defined as in \cref{sec:classifying}.
Let $\mathscr{C} = \set{C_t \suchthat t \in T}$ denote the set of classes.

\begin{theorem}
	\label{th:B}
	$\forall t \in T, i \in N$:
	\begin{itemize}
		\item $\exists \profB \in C_{t, i} \suchthat \FB(\profB) = B(\profB)$;
		\item $t_1 = 1 ⇒ \exists \profD \in C_{t, i} \suchthat \FB(\profD) \subset B(\profD)$;
		\item $t_1 ≥ 2 ⇒ \exists \profD \in C_{t, i} \suchthat \FB(\profD) \cap B(\profD) = \emptyset$;
	\end{itemize}
\end{theorem}
\begin{proof}
%	Let $t \in T$ and $i \in N$. 
	Throughout this proof, item numbers refer to \cref{th:consequentBounds}.

	Name the alternatives $a_1, …, a_m$.
	Define $x = a_{t_1 + 1}$ and $y = a_{t_4 + 1}$.
	Define sequences of alternatives 
	\begin{description}
		\item[$A_1$] $(a_{t_4 + 2}, …, a_{t_2 + t_4 + 1})$;
		\item[$A_2$] $(a_{t_2 + t_4 + 2}, …, a_{t_3 + t_4})$ (with $A_2 = \emptyset$ iff $t_3 + t_4 < t_2 + t_4 + 2$);
		\item[$A_3$] $(a_{t_3 + t_4 + 1}, …, a_m)$;
		\item[$A_4$] $(a_1, …, a_{t_1})$;
		\item[$A_5$] $(a_{t_1 + 2}, …, a_{t_4})$.
	\end{description}
	Given a sequence $A$, define $A^{-1}$ as the inverse of $A$.
	Define $\profB$ as
	\begin{description}
		\item[$i$] $(a_1, …, a_m) = (A_4, x, A_5, y, A_1, A_2, A_3)$,
		\item[$\ibar$] $(A_1^{-1}, y, A_2, x, A_3, A_5, A_4^{-1})$
	\end{description}
	and define $\profD$ as
	\begin{description}
		\item[$i$] $(a_1, …, a_m) = (A_4, x, A_5, y, A_1, A_2, A_3)$,
		\item[$\ibar$] $(A_1^{-1}, y, A_2, x, A_4, A_3, A_5)$.
	\end{description}
	Note that $\profB$ and $\profD$ differ only by the position and ordering of $A_4$ in $\ibar$. Therefore, most of the facts we will establish for $\profB$ also hold, and can be established in a similar manner, for $\profD$.
	We will therefore reason in parallel for these two profiles in what follows.
	
	Note that $\profB(\ibar)$ and $\profD(\ibar)$ are indeed linear orders on $\allalts$, equivalently, list each alternative of $\allalts$ exactly once, as they are permutations of the sequence $(A_4, x, A_5, y, A_1, A_2, A_3)$ which lists each alternative of $\allalts$ exactly once. The latter fact holds because $t_2 + t_4 + 1 ≤ t_3 + t_4$, equivalently, $t_2 + 1 ≤ t_3$, from \cref{it:torder}.
	
	Observe that $t_1 = \lprofB(x)_i = \lprofD(x)_i$ and $t_3 = \card{(A_1 \cup \set{y} \cup A_2)} = \lprofB(x)_{\ibar} = \lprofD(x)_{\ibar}$. It follows that $\min \lprofB(x) = \min \lprofD(x) = t_1$ and $\max \lprofB(x) = \max \lprofD(x) = t_3$.
	Similarly, $t_4 = \lprofB(y)_i = \lprofD(y)_i$ and $t_2 = \card{A_1} = \lprofB(y)_{\ibar} = \lprofD(y)_{\ibar}$. It follows that $\min \lprofB(y) = \min \lprofD(y) = t_2$ and $\max \lprofB(y) = \max \lprofD(y) = t_4$.

	To see that $\FB(\profB) = \FB(\profD) = \argmin \max \lprofB = \set{x}$, observe that $\max \lprofB(x) = \max \lprofD(x) = t_3$ and let us show that $\forall z \in \allalts \setminus \set{x}: \max \lprofB(z) > t_3 \land \max \lprofD(z) > t_3$. 
	Indeed:
	\begin{itemize}
		\item $\forall z \in \set{y} \cup A_1 \cup A_2 \cup A_3, t_3 < t_4 = \lprofB(y)_i < \lprofB(z)_i ≤ \max \lprofB(z)$;
		\item $\forall z \in \set{y} \cup A_1 \cup A_2 \cup A_3, t_3 < t_4 = \lprofD(y)_i < \lprofD(z)_i ≤ \max \lprofD(z)$;
		\item $\forall z \in \set{A_4 \cup A_5}, t_3 = \lprofB(x)_{\ibar} < \lprofB(z)_{\ibar} ≤ \max \lprofB(z)$;
		\item $\forall z \in \set{A_4 \cup A_5}, t_3 = \lprofD(x)_{\ibar} < \lprofD(z)_{\ibar} ≤ \max \lprofD(z)$.
	\end{itemize}
	
	To show that $\MSP = \MS(\profD) = \argmin_{\POP} (d \circ \lprof) = \set{y}$, observe that $y \in \POP$, $y \in \PO(\profD)$, $(d \circ \lprof)(y) = (d \circ \lprofD)(y) = t_4 - t_2$, and let us show that $\forall z \in \allalts \setminus \set{y}: (d \circ \lprof)(z) > t_4 - t_2 \land (d \circ \lprofD)(z) > t_4 - t_2$.
	\begin{itemize}
		\item $\forall z \in A_1$, $\lprofB(z)_{\ibar} = \lprofD(z)_{\ibar} < \lprofB(y)_{\ibar} < \lprofB(y)_i < \lprofB(z)_i = \lprofD(z)_i$ thus $d(\lprofB(z)) = d(\lprofD(z)) > d(\lprofB(y)) = t_4 - t_2$.
		\item $\forall z \in A_3$, $d(\lprofD(z)) = \max \lprofD(z) - \min \lprofD(z) = \card{(A_4 \cup \set{x} \cup A_5 \cup \set{y} \cup A_1 \cup A_2)} - \card{(A_1 \cup \set{y} \cup A_2 \cup \set{x} \cup A_4)} = \card{A_5} = t_4 - t_1 - 1 > t_4 - t_2$ 
			(using $t_1 + 1 < t_1 + 2 ≤ t_2$ from \cref{it:t12}).
		\item $\forall z \in A_3$, $d(\lprofB(z)) = \card{(A_4 \cup A_5)} ≥ \card{A_5} > t_4 - t_2$ (from the previous line).
		\item $\forall z \in A_2$, $d(\lprofB(z)) = d(\lprofD(z))> \card{A_5} > t_4 - t_2$.
		\item $\forall z \in A_5$, $d(\lprofB(z)) = \max \lprofB(z) - \min \lprofB(z) = \abs{\card{(A_1 \cup \set{y} \cup A_2 \cup \set{x} \cup A_3)} - \card{(A_4 \cup \set{x})}} = \abs{\card{(A_1 \cup \set{y} \cup A_2 \cup A_3)} - \card{A_4}} = \abs{m - t_4 - t_1}$.
		And $m - t_4 - t_1 > m - t_4 - t_1 - 1 ≥ t_4 - t_2 > 0$, from \crefns{it:t1m42,it:torder}.
		\item $\forall z \in A_5$, $d(\lprofD(z)) = d(\lprofB(z)) + \card{A_4} > t_4 - t_2$.
		\item $d(\lprofB(x)) = d(\lprofD(x)) = t_3 - t_1 > t_3 - t_1 - 1 ≥ t_4 - t_2$ (using \cref{it:t4231}).
		\item $\forall z \in A_4$, $d(\lprofB(z)) = \card{(A_1 \cup \set{y} \cup A_2 \cup \set{x} \cup A_3 \cup A_5)} - \card{A_4} > m - t_4 - t_1 > t_4 - t_2$ (from a previous line).
		\item $\forall z \in A_4$, $d(\lprofD(z)) = \card{(A_1 \cup \set{y} \cup A_2 \cup \set{x})} = t_3 + 1 > t_3 ≥ t_3 - t_1 > t_4 - t_2$ (from a previous line).
	\end{itemize}

	We see that $\profB \in \fprofs$ and $\profD \in \fprofs$ as it has already been established that $\forall z \in \allalts \setminus \set{y}: (d \circ \lprofB)(z) > (d \circ \lprofB)(y) \land (d \circ \lprofD)(z) > (d \circ \lprofD)(y)$.

	Finally, we turn to the Borda winners. 
	Observe that $\sum \lprofB(x) = t_1 + t_3$. 
	Let us show first that $\forall z \in \allalts \setminus (A_4 \cup \set{x}): \sum \lprofB(z) > t_1 + t_3 \land \sum \lprofD(z) > t_1 + t_3$. 
	\begin{itemize}
		\item $\forall z \in A_1$, $\sum \lprofB(z) = \sum \lprofD(z) = \card{(A_4 \cup \set{x} \cup A_5 \cup \set{y} \cup A_1 \setminus \set{z})} = t_4 + 1 + t_2 - 1 = t_2 + t_4 > t_1 + t_3$.
		\item $\forall z \in \set{y} \cup A_2 \cup A_3$, $\sum \lprofB(z) ≥ \sum \lprofB(y) = t_2 + t_4 > t_1 + t_3$.
		\item $\forall z \in \set{y} \cup A_2 \cup A_3$, $\sum \lprofD(z) ≥ \sum \lprofD(y) = t_2 + t_4 > t_1 + t_3$.
	\end{itemize}

	To see that $B(\profB) = \argmin \sum \lprof = \set{x}$, observe furthermore that
	$\forall z \in A_4$, $\sum \lprofB(z) ≥ \card{(A_1 \cup \set{y} \cup A_2 \cup \set{x} \cup A_3 \cup A_5 \cup A_4 \setminus \set{z})} = \card{\allalts \setminus \set{z}} = m - 1 > m - 4 ≥ t_1 + t_3$ (using \cref{it:t13m}).
	
	Only remains to consider $B(\profD)$.
	We know from above that $A_4$ and $x$ are the only candidates. Thus, $B(\profD) \subseteq \set{a_1, x}$. (Note that $a_1 ≠ x ⇔ t_1 ≥ 1 ⇔ A_4 ≠ \emptyset ⇔ \profB ≠ \profD$.)
	We see that $\sum \lprofD(a_1) - \sum \lprofD(x) = (\lprofD(a_1)_i - \lprofD(x)_i) + (\lprofD(a_1)_{\ibar} - \lprofD(x)_{\ibar}) = - \card{A_4} + 1 = 1 - t_1$.
	Thus, if $t_1 = 1$, $B(\profD) = \set{a_1, x}$ and if $t_1 ≥ 2$, $B(\profD) = \set{a_1}$.
\end{proof}

\begin{example}
	\label{ex:profs}
Here is an example construction of profiles $\profB, \profD$ as in the proof of \cref{th:B}, with $m = 13$ and $t_1 = \lprof(x)_i = 1$; $t_2 = \lprof(y)_{\ibar} = 4$; $t_3 = \lprof(x)_{\ibar} = 5$; $t_4 = \lprof(y)_i = 6$; thus $\sum \lprof(y) = 10$, $x = \bm{b}$ and $y = \boxed{g}$ (for $\prof \in \set{\profB, \profD}$).
	\begin{description}
		\item[$A_1$] $(a_8, …, a_{11}) = (h, i, j, k)$;
		\item[$A_2$] $(a_{12}, …, a_{11}) = \emptyset$;
		\item[$A_3$] $(a_{12}, …, a_{13}) = (l, m)$;
		\item[$A_4$] $(a_1, …, a_1) = (a)$;
		\item[$A_5$] $(a_3, …, a_6) = (c, d, e, f)$.
	\end{description}

	The profile $\profB$ is:
  \begin{equation}
    \begin{array}{*{13}c}
      a&\bm{b}&c&d&e&f&\boxed{g}&h&i&j&k&l&m\\
      k&j&i&h&\boxed{g}&\bm{b}&l&m&c&d&e&f&a.
    \end{array}
  \end{equation}
	Observe that $B(\profB) = \set{\bm{b}}$.

	The profile $\profD$ is:
  \begin{equation}
    \begin{array}{*{13}c}
      a&\bm{b}&c&d&e&f&\boxed{g}&h&i&j&k&l&m\\
      k&j&i&h&\boxed{g}&\bm{b}&a&l&m&c&d&e&f.
    \end{array}
  \end{equation}
	Observe that $B(\profD) = \set{a, \bm{b}}$.
\end{example}

\begin{theorem}
	\label{th:cover}
	The set of classes $\set{C_t \suchthat t \in T}$ is a complete and disjoint cover of $\fprofs$. 
	Furthermore, given any $t \in T$, no class $C_t$ is empty.
\end{theorem}
\begin{proof}
	First, it is a complete cover, formally, $\fprofs = \bigcup \set{C_t \suchthat t \in T}$: from \cref{th:sufficientBounds}, $\forall \prof \in \fprofs, \exists t \in T \suchthat \prof \in C_t$; and conversely, $\forall t \in T, \prof \in C_t$, it follows from the definitions that $\prof \in \fprofs$.
	
	Second, that the classes are disjoint, formally, $C_t \cap C_{t'} ≠ \emptyset ⇒ t = t'$, can be seen as follows. 
	Consider any $\prof \in C_t$, let $\set{x} = \FBP$ and $\set{y} = \MSP$ and pick any $i \suchthat \prof \in C_{t, i}$. 
	By definition of $C_{t, i}$, $\lprof(x)_i = t_1 < \lprof(y)_{\ibar} = t_2 < \lprof(x)_{\ibar} = t_3 < \lprof(y)_i = t_4$. 
	It follows that $\prof \notin C_{t', \ibar}$ (as the latter requires $\lprof(x)_{\ibar} = t'_1 < \lprof(x)_i = t'_3$).
	Therefore, if $\prof \in C_{t'}$, then $\prof \in C_{t', i}$, thus $\lprof(x)_i = t'_1, \lprof(y)_{\ibar} = t'_2, \lprof(x)_{\ibar} = t'_3$ and $\lprof(y)_i = t'_4$, whence $t = t'$.
	
	Finally, \cref{th:B} proves that no $C_t$ is empty.
\end{proof}

\subsection{Borda}
\begin{theorem}
	\label{th:min0B}
	$\forall \prof \in \allprofs: [\exists x \in FB(P) \suchthat \min \lprof(x) = 0] ⇒ B(P) \subseteq FB(P)$.
\end{theorem}
\begin{proof}
	Consider any $z \in B(P)$. 
	Observe that $\sum \lprof(z) ≤ \sum \lprof(x)$, thus, $\max \lprof(z) ≤ \max \lprof(z) + \min \lprof(z) = \sum \lprof(z) ≤ \sum \lprof(x) = \max \lprof(x)$.
	It follows that $z \in FB(P)$.
\end{proof}
\begin{theorem}
	\label{th:BMS}
	$\forall \prof \in \allprofs: \MSP \nsubseteq \FBP ⇒ \MSP \cap B(P) = \emptyset$.
\end{theorem}
\begin{proof}
	Consider any $y \in \MSP$ and let us show that $y \notin \argmin \sum \lprof = \BP$.
	Pick any $x \in \FBP$ (thus $\max \lprof(x) = \min \max \lprof$).
	
	As $\MSP \nsubseteq \FBP$, $y \notin \FBP$, thus $\max \lprof(x) = \min \max \lprof < \max \lprof(y)$.
	
	We furthermore deduce that $\min \lprof(y) ≥ \min \lprof(x)$, otherwise, $\min \lprof(y) < \min \lprof(x) ≤ \max \lprof(x) = \min \max \lprof < \max \lprof(y)$ hence $y \notin \MSP = \argmin_{\POP}(d \circ \lprof)$ (using $x \in \FBP ⇒ x \in \POP$).
	
	It follows that $\sum \lprof(x) < \sum \lprof(y)$.
\end{proof}
\begin{theorem}
	\label{th:min1B}
	$\forall \prof \in \allprofs: [\exists x \in \FBP \suchthat \min \lprof(x) = 1] ⇒ \FBP \cap B(P) ≠ \emptyset$.
\end{theorem}
\begin{proof}
	Considering any $z \notin \FBP$, let us show that $\sum \lprof(x) ≤ \sum \lprof(z)$ (which excludes $\argmin \sum \lprof \cap \FBP = \emptyset$).
	As $x \in \FBP$, $\max \lprof(x) = \min \max \lprof$.
	If $\max \lprof(z) = \min \max \lprof$ then $z \in \FBP$, thus, $\max \lprof(z) > \max \lprof(x)$, equivalently, $\max \lprof(x) + 1 ≤ \max \lprof(z)$.
	Then $\sum \lprof(x) = \min \lprof(x) + \max \lprof(x) = 1 + \max \lprof(x) ≤ \max \lprof(z) ≤ \sum \lprof(z)$.
\end{proof}

\begin{remark}
	\label{rk:BFB}
	Here are the possible relationships between $\BP$ and $\FBP$, considering $\prof \in \fprofs$ (thus $\MSP \nsubseteq \FBP$, with $\FBP = \set{x}$):
	\begin{itemize}
		\item if $\min \lprof(x) = 0$, $\BP = \FBP$ (\cref{th:min0B});
		\item if $\min \lprof(x) = 1$, $x \in \BP$ (\cref{th:min1B}), with both $\FBP = \BP$ and $\FBP \subset \BP$ possible (\cref{th:B});
		\item if $\min \lprof(x) ≥ 2$, both $\FBP = \BP$ and $\FBP \cap \BP = \emptyset$ are possible (\cref{th:B}).
	\end{itemize}
\end{remark}

\subsection{Tightness results}
The following two theorems illustrate why we cannot obtain the results that were obtained above with the hypothesis $\FBP \cap \MSP = \emptyset$ by using merely an hypothesis of non inclusion.

\begin{theorem}
	\label{th:FBMSposs}
	For any $m ≥ 6$, $\FBP \cap \MSP ≠ \emptyset \not⇒ \MSP \subseteq \FBP$.
	In other words, $\exists \prof \suchthat \FBP \cap \MSP ≠ \emptyset \land \MSP \nsubseteq \FBP$.
	In supplement, $\exists \prof \suchthat \FBP \cap \argmin_{\allalts}(d \circ \lprof)  ≠ \emptyset \land \argmin_{\allalts}(d \circ \lprof) \nsubseteq \FBP$.
\end{theorem}
\begin{proof}
	In the following profile $\prof$,
	\begin{equation}
		\begin{array}{*{13}l}
			a	& b	& c	& d	& e	& f	& \ldots\\
			e& d& a &f &b &c &…
		\end{array},
	\end{equation}
	$\FBP = \argmin \max \lprof = \set{a}$ and $\MSP = \argmin_{\POP}(d \circ \lprof) = \argmin_{\allalts}(d \circ \lprof) = \set{a, d}$.
\end{proof}

\begin{theorem}
	\label{th:MSFBposs}
	For any $m ≥ 4$, $\FBP \cap \MSP ≠ \emptyset \not⇒ \FBP \subseteq \MSP$.
\end{theorem}
\begin{proof}
	In the following profile $\prof$,
	\begin{equation}
		\begin{array}{*{13}l}
			a	& c	& b	& d	& \ldots\\
			d& b& a &c &…
		\end{array},
	\end{equation}
	$\FBP = \argmin \max \lprof = \set{a, b}$ and $\MSP = \argmin_{\POP}(d \circ \lprof) = \argmin_{\allalts}(d \circ \lprof) = \set{b}$.
\end{proof}

\subsection{Instances B}
The profiles in \crefrange{ex:0657B}{ex:0548B} all satisfy $\BP = \FBP$ and are generated using the process described in \cref{th:B} and illustrated in \cref{ex:profs}.
They are listed in the same order as \cref{fig:m13}.

%Generated – please do not edit.

\begin{example}[$\lprof(x) = \{0, 6\}$; $\lprof(y) = \{5, 7\}$; B]
  \label{ex:0657B}
  \begin{equation}
    \begin{array}{*{13}c}
      \bm{a}&b&c&d&e&f&g&\boxed{h}&i&j&k&l&m\\
      m&l&k&j&i&\boxed{h}&\bm{a}&b&c&d&e&f&g
    \end{array}
  \end{equation}
\end{example}

\begin{example}[$\lprof(x) = \{1, 6\}$; $\lprof(y) = \{5, 7\}$; B]
  \label{ex:1657B}
  \begin{equation}
    \begin{array}{*{13}c}
      a&\bm{b}&c&d&e&f&g&\boxed{h}&i&j&k&l&m\\
      m&l&k&j&i&\boxed{h}&\bm{b}&c&d&e&f&g&a
    \end{array}
  \end{equation}
\end{example}

\begin{example}[$\lprof(x) = \{0, 5\}$; $\lprof(y) = \{4, 6\}$; B]
  \label{ex:0546B}
  \begin{equation}
    \begin{array}{*{13}c}
      \bm{a}&b&c&d&e&f&\boxed{g}&h&i&j&k&l&m\\
      k&j&i&h&\boxed{g}&\bm{a}&l&m&b&c&d&e&f
    \end{array}
  \end{equation}
\end{example}

\begin{example}[$\lprof(x) = \{0, 6\}$; $\lprof(y) = \{4, 7\}$; B]
  \label{ex:0647B}
  \begin{equation}
    \begin{array}{*{13}c}
      \bm{a}&b&c&d&e&f&g&\boxed{h}&i&j&k&l&m\\
      l&k&j&i&\boxed{h}&m&\bm{a}&b&c&d&e&f&g
    \end{array}
  \end{equation}
\end{example}

\begin{example}[$\lprof(x) = \{2, 6\}$; $\lprof(y) = \{5, 7\}$; B]
  \label{ex:2657B}
  \begin{equation}
    \begin{array}{*{13}c}
      a&b&\bm{c}&d&e&f&g&\boxed{h}&i&j&k&l&m\\
      m&l&k&j&i&\boxed{h}&\bm{c}&d&e&f&g&b&a
    \end{array}
  \end{equation}
\end{example}

\begin{example}[$\lprof(x) = \{1, 5\}$; $\lprof(y) = \{4, 6\}$; B]
  \label{ex:1546B}
  \begin{equation}
    \begin{array}{*{13}c}
      a&\bm{b}&c&d&e&f&\boxed{g}&h&i&j&k&l&m\\
      k&j&i&h&\boxed{g}&\bm{b}&l&m&c&d&e&f&a
    \end{array}
  \end{equation}
\end{example}

\begin{example}[$\lprof(x) = \{0, 4\}$; $\lprof(y) = \{3, 5\}$; B]
  \label{ex:0435B}
  \begin{equation}
    \begin{array}{*{13}c}
      \bm{a}&b&c&d&e&\boxed{f}&g&h&i&j&k&l&m\\
      i&h&g&\boxed{f}&\bm{a}&j&k&l&m&b&c&d&e
    \end{array}
  \end{equation}
\end{example}

\begin{example}[$\lprof(x) = \{1, 6\}$; $\lprof(y) = \{4, 7\}$; B]
  \label{ex:1647B}
  \begin{equation}
    \begin{array}{*{13}c}
      a&\bm{b}&c&d&e&f&g&\boxed{h}&i&j&k&l&m\\
      l&k&j&i&\boxed{h}&m&\bm{b}&c&d&e&f&g&a
    \end{array}
  \end{equation}
\end{example}

\begin{example}[$\lprof(x) = \{0, 5\}$; $\lprof(y) = \{3, 6\}$; B]
  \label{ex:0536B}
  \begin{equation}
    \begin{array}{*{13}c}
      \bm{a}&b&c&d&e&f&\boxed{g}&h&i&j&k&l&m\\
      j&i&h&\boxed{g}&k&\bm{a}&l&m&b&c&d&e&f
    \end{array}
  \end{equation}
\end{example}

\begin{example}[$\lprof(x) = \{0, 6\}$; $\lprof(y) = \{3, 7\}$; B]
  \label{ex:0637B}
  \begin{equation}
    \begin{array}{*{13}c}
      \bm{a}&b&c&d&e&f&g&\boxed{h}&i&j&k&l&m\\
      k&j&i&\boxed{h}&l&m&\bm{a}&b&c&d&e&f&g
    \end{array}
  \end{equation}
\end{example}

\begin{example}[$\lprof(x) = \{0, 5\}$; $\lprof(y) = \{4, 7\}$; B]
  \label{ex:0547B}
  \begin{equation}
    \begin{array}{*{13}c}
      \bm{a}&b&c&d&e&f&g&\boxed{h}&i&j&k&l&m\\
      l&k&j&i&\boxed{h}&\bm{a}&m&b&c&d&e&f&g
    \end{array}
  \end{equation}
\end{example}

\begin{example}[$\lprof(x) = \{3, 6\}$; $\lprof(y) = \{5, 7\}$; B]
  \label{ex:3657B}
  \begin{equation}
    \begin{array}{*{13}c}
      a&b&c&\bm{d}&e&f&g&\boxed{h}&i&j&k&l&m\\
      m&l&k&j&i&\boxed{h}&\bm{d}&e&f&g&c&b&a
    \end{array}
  \end{equation}
\end{example}

\begin{example}[$\lprof(x) = \{2, 5\}$; $\lprof(y) = \{4, 6\}$; B]
  \label{ex:2546B}
  \begin{equation}
    \begin{array}{*{13}c}
      a&b&\bm{c}&d&e&f&\boxed{g}&h&i&j&k&l&m\\
      k&j&i&h&\boxed{g}&\bm{c}&l&m&d&e&f&b&a
    \end{array}
  \end{equation}
\end{example}

\begin{example}[$\lprof(x) = \{1, 4\}$; $\lprof(y) = \{3, 5\}$; B]
  \label{ex:1435B}
  \begin{equation}
    \begin{array}{*{13}c}
      a&\bm{b}&c&d&e&\boxed{f}&g&h&i&j&k&l&m\\
      i&h&g&\boxed{f}&\bm{b}&j&k&l&m&c&d&e&a
    \end{array}
  \end{equation}
\end{example}

\begin{example}[$\lprof(x) = \{0, 3\}$; $\lprof(y) = \{2, 4\}$; B]
  \label{ex:0324B}
  \begin{equation}
    \begin{array}{*{13}c}
      \bm{a}&b&c&d&\boxed{e}&f&g&h&i&j&k&l&m\\
      g&f&\boxed{e}&\bm{a}&h&i&j&k&l&m&b&c&d
    \end{array}
  \end{equation}
\end{example}

\begin{example}[$\lprof(x) = \{2, 6\}$; $\lprof(y) = \{4, 7\}$; B]
  \label{ex:2647B}
  \begin{equation}
    \begin{array}{*{13}c}
      a&b&\bm{c}&d&e&f&g&\boxed{h}&i&j&k&l&m\\
      l&k&j&i&\boxed{h}&m&\bm{c}&d&e&f&g&b&a
    \end{array}
  \end{equation}
\end{example}

\begin{example}[$\lprof(x) = \{1, 5\}$; $\lprof(y) = \{3, 6\}$; B]
  \label{ex:1536B}
  \begin{equation}
    \begin{array}{*{13}c}
      a&\bm{b}&c&d&e&f&\boxed{g}&h&i&j&k&l&m\\
      j&i&h&\boxed{g}&k&\bm{b}&l&m&c&d&e&f&a
    \end{array}
  \end{equation}
\end{example}

\begin{example}[$\lprof(x) = \{0, 4\}$; $\lprof(y) = \{2, 5\}$; B]
  \label{ex:0425B}
  \begin{equation}
    \begin{array}{*{13}c}
      \bm{a}&b&c&d&e&\boxed{f}&g&h&i&j&k&l&m\\
      h&g&\boxed{f}&i&\bm{a}&j&k&l&m&b&c&d&e
    \end{array}
  \end{equation}
\end{example}

\begin{example}[$\lprof(x) = \{1, 6\}$; $\lprof(y) = \{3, 7\}$; B]
  \label{ex:1637B}
  \begin{equation}
    \begin{array}{*{13}c}
      a&\bm{b}&c&d&e&f&g&\boxed{h}&i&j&k&l&m\\
      k&j&i&\boxed{h}&l&m&\bm{b}&c&d&e&f&g&a
    \end{array}
  \end{equation}
\end{example}

\begin{example}[$\lprof(x) = \{0, 5\}$; $\lprof(y) = \{2, 6\}$; B]
  \label{ex:0526B}
  \begin{equation}
    \begin{array}{*{13}c}
      \bm{a}&b&c&d&e&f&\boxed{g}&h&i&j&k&l&m\\
      i&h&\boxed{g}&j&k&\bm{a}&l&m&b&c&d&e&f
    \end{array}
  \end{equation}
\end{example}

\begin{example}[$\lprof(x) = \{0, 6\}$; $\lprof(y) = \{2, 7\}$; B]
  \label{ex:0627B}
  \begin{equation}
    \begin{array}{*{13}c}
      \bm{a}&b&c&d&e&f&g&\boxed{h}&i&j&k&l&m\\
      j&i&\boxed{h}&k&l&m&\bm{a}&b&c&d&e&f&g
    \end{array}
  \end{equation}
\end{example}

\begin{example}[$\lprof(x) = \{1, 5\}$; $\lprof(y) = \{4, 7\}$; B]
  \label{ex:1547B}
  \begin{equation}
    \begin{array}{*{13}c}
      a&\bm{b}&c&d&e&f&g&\boxed{h}&i&j&k&l&m\\
      l&k&j&i&\boxed{h}&\bm{b}&m&c&d&e&f&g&a
    \end{array}
  \end{equation}
\end{example}

\begin{example}[$\lprof(x) = \{0, 4\}$; $\lprof(y) = \{3, 6\}$; B]
  \label{ex:0436B}
  \begin{equation}
    \begin{array}{*{13}c}
      \bm{a}&b&c&d&e&f&\boxed{g}&h&i&j&k&l&m\\
      j&i&h&\boxed{g}&\bm{a}&k&l&m&b&c&d&e&f
    \end{array}
  \end{equation}
\end{example}

\begin{example}[$\lprof(x) = \{0, 5\}$; $\lprof(y) = \{3, 7\}$; B]
  \label{ex:0537B}
  \begin{equation}
    \begin{array}{*{13}c}
      \bm{a}&b&c&d&e&f&g&\boxed{h}&i&j&k&l&m\\
      k&j&i&\boxed{h}&l&\bm{a}&m&b&c&d&e&f&g
    \end{array}
  \end{equation}
\end{example}

\begin{example}[$\lprof(x) = \{0, 5\}$; $\lprof(y) = \{4, 8\}$; B]
  \label{ex:0548B}
  \begin{equation}
    \begin{array}{*{13}c}
      \bm{a}&b&c&d&e&f&g&h&\boxed{i}&j&k&l&m\\
      m&l&k&j&\boxed{i}&\bm{a}&b&c&d&e&f&g&h
    \end{array}
  \end{equation}
\end{example}


\subsection{Instances D}
The profiles in \crefrange{ex:0657D}{ex:0548D} all satisfy $\BP ≠ \FBP$ if possible (that is, iff $\min \lprof(\FBP) ≥ 1$) and are generated using the process described in \cref{th:B} and illustrated in \cref{ex:profs}.
They are listed in the same order as \cref{fig:m13}.

%Generated – please do not edit.

\begin{example}[$\lprof(x) = \{0, 6\}$; $\lprof(y) = \{5, 7\}$; D]
  \label{ex:0657D}
  \begin{equation}
    \begin{array}{*{13}c}
      \bm{a}&b&c&d&e&f&g&\boxed{h}&i&j&k&l&m\\
      m&l&k&j&i&\boxed{h}&\bm{a}&b&c&d&e&f&g
    \end{array}
  \end{equation}
\end{example}

\begin{example}[$\lprof(x) = \{1, 6\}$; $\lprof(y) = \{5, 7\}$; D]
  \label{ex:1657D}
  \begin{equation}
    \begin{array}{*{13}c}
      a&\bm{b}&c&d&e&f&g&\boxed{h}&i&j&k&l&m\\
      m&l&k&j&i&\boxed{h}&\bm{b}&a&c&d&e&f&g
    \end{array}
  \end{equation}
\end{example}

\begin{example}[$\lprof(x) = \{0, 5\}$; $\lprof(y) = \{4, 6\}$; D]
  \label{ex:0546D}
  \begin{equation}
    \begin{array}{*{13}c}
      \bm{a}&b&c&d&e&f&\boxed{g}&h&i&j&k&l&m\\
      k&j&i&h&\boxed{g}&\bm{a}&l&m&b&c&d&e&f
    \end{array}
  \end{equation}
\end{example}

\begin{example}[$\lprof(x) = \{0, 6\}$; $\lprof(y) = \{4, 7\}$; D]
  \label{ex:0647D}
  \begin{equation}
    \begin{array}{*{13}c}
      \bm{a}&b&c&d&e&f&g&\boxed{h}&i&j&k&l&m\\
      l&k&j&i&\boxed{h}&m&\bm{a}&b&c&d&e&f&g
    \end{array}
  \end{equation}
\end{example}

\begin{example}[$\lprof(x) = \{2, 6\}$; $\lprof(y) = \{5, 7\}$; D]
  \label{ex:2657D}
  \begin{equation}
    \begin{array}{*{13}c}
      a&b&\bm{c}&d&e&f&g&\boxed{h}&i&j&k&l&m\\
      m&l&k&j&i&\boxed{h}&\bm{c}&a&b&d&e&f&g
    \end{array}
  \end{equation}
\end{example}

\begin{example}[$\lprof(x) = \{1, 5\}$; $\lprof(y) = \{4, 6\}$; D]
  \label{ex:1546D}
  \begin{equation}
    \begin{array}{*{13}c}
      a&\bm{b}&c&d&e&f&\boxed{g}&h&i&j&k&l&m\\
      k&j&i&h&\boxed{g}&\bm{b}&a&l&m&c&d&e&f
    \end{array}
  \end{equation}
\end{example}

\begin{example}[$\lprof(x) = \{0, 4\}$; $\lprof(y) = \{3, 5\}$; D]
  \label{ex:0435D}
  \begin{equation}
    \begin{array}{*{13}c}
      \bm{a}&b&c&d&e&\boxed{f}&g&h&i&j&k&l&m\\
      i&h&g&\boxed{f}&\bm{a}&j&k&l&m&b&c&d&e
    \end{array}
  \end{equation}
\end{example}

\begin{example}[$\lprof(x) = \{1, 6\}$; $\lprof(y) = \{4, 7\}$; D]
  \label{ex:1647D}
  \begin{equation}
    \begin{array}{*{13}c}
      a&\bm{b}&c&d&e&f&g&\boxed{h}&i&j&k&l&m\\
      l&k&j&i&\boxed{h}&m&\bm{b}&a&c&d&e&f&g
    \end{array}
  \end{equation}
\end{example}

\begin{example}[$\lprof(x) = \{0, 5\}$; $\lprof(y) = \{3, 6\}$; D]
  \label{ex:0536D}
  \begin{equation}
    \begin{array}{*{13}c}
      \bm{a}&b&c&d&e&f&\boxed{g}&h&i&j&k&l&m\\
      j&i&h&\boxed{g}&k&\bm{a}&l&m&b&c&d&e&f
    \end{array}
  \end{equation}
\end{example}

\begin{example}[$\lprof(x) = \{0, 6\}$; $\lprof(y) = \{3, 7\}$; D]
  \label{ex:0637D}
  \begin{equation}
    \begin{array}{*{13}c}
      \bm{a}&b&c&d&e&f&g&\boxed{h}&i&j&k&l&m\\
      k&j&i&\boxed{h}&l&m&\bm{a}&b&c&d&e&f&g
    \end{array}
  \end{equation}
\end{example}

\begin{example}[$\lprof(x) = \{0, 5\}$; $\lprof(y) = \{4, 7\}$; D]
  \label{ex:0547D}
  \begin{equation}
    \begin{array}{*{13}c}
      \bm{a}&b&c&d&e&f&g&\boxed{h}&i&j&k&l&m\\
      l&k&j&i&\boxed{h}&\bm{a}&m&b&c&d&e&f&g
    \end{array}
  \end{equation}
\end{example}

\begin{example}[$\lprof(x) = \{3, 6\}$; $\lprof(y) = \{5, 7\}$; D]
  \label{ex:3657D}
  \begin{equation}
    \begin{array}{*{13}c}
      a&b&c&\bm{d}&e&f&g&\boxed{h}&i&j&k&l&m\\
      m&l&k&j&i&\boxed{h}&\bm{d}&a&b&c&e&f&g
    \end{array}
  \end{equation}
\end{example}

\begin{example}[$\lprof(x) = \{2, 5\}$; $\lprof(y) = \{4, 6\}$; D]
  \label{ex:2546D}
  \begin{equation}
    \begin{array}{*{13}c}
      a&b&\bm{c}&d&e&f&\boxed{g}&h&i&j&k&l&m\\
      k&j&i&h&\boxed{g}&\bm{c}&a&b&l&m&d&e&f
    \end{array}
  \end{equation}
\end{example}

\begin{example}[$\lprof(x) = \{1, 4\}$; $\lprof(y) = \{3, 5\}$; D]
  \label{ex:1435D}
  \begin{equation}
    \begin{array}{*{13}c}
      a&\bm{b}&c&d&e&\boxed{f}&g&h&i&j&k&l&m\\
      i&h&g&\boxed{f}&\bm{b}&a&j&k&l&m&c&d&e
    \end{array}
  \end{equation}
\end{example}

\begin{example}[$\lprof(x) = \{0, 3\}$; $\lprof(y) = \{2, 4\}$; D]
  \label{ex:0324D}
  \begin{equation}
    \begin{array}{*{13}c}
      \bm{a}&b&c&d&\boxed{e}&f&g&h&i&j&k&l&m\\
      g&f&\boxed{e}&\bm{a}&h&i&j&k&l&m&b&c&d
    \end{array}
  \end{equation}
\end{example}

\begin{example}[$\lprof(x) = \{2, 6\}$; $\lprof(y) = \{4, 7\}$; D]
  \label{ex:2647D}
  \begin{equation}
    \begin{array}{*{13}c}
      a&b&\bm{c}&d&e&f&g&\boxed{h}&i&j&k&l&m\\
      l&k&j&i&\boxed{h}&m&\bm{c}&a&b&d&e&f&g
    \end{array}
  \end{equation}
\end{example}

\begin{example}[$\lprof(x) = \{1, 5\}$; $\lprof(y) = \{3, 6\}$; D]
  \label{ex:1536D}
  \begin{equation}
    \begin{array}{*{13}c}
      a&\bm{b}&c&d&e&f&\boxed{g}&h&i&j&k&l&m\\
      j&i&h&\boxed{g}&k&\bm{b}&a&l&m&c&d&e&f
    \end{array}
  \end{equation}
\end{example}

\begin{example}[$\lprof(x) = \{0, 4\}$; $\lprof(y) = \{2, 5\}$; D]
  \label{ex:0425D}
  \begin{equation}
    \begin{array}{*{13}c}
      \bm{a}&b&c&d&e&\boxed{f}&g&h&i&j&k&l&m\\
      h&g&\boxed{f}&i&\bm{a}&j&k&l&m&b&c&d&e
    \end{array}
  \end{equation}
\end{example}

\begin{example}[$\lprof(x) = \{1, 6\}$; $\lprof(y) = \{3, 7\}$; D]
  \label{ex:1637D}
  \begin{equation}
    \begin{array}{*{13}c}
      a&\bm{b}&c&d&e&f&g&\boxed{h}&i&j&k&l&m\\
      k&j&i&\boxed{h}&l&m&\bm{b}&a&c&d&e&f&g
    \end{array}
  \end{equation}
\end{example}

\begin{example}[$\lprof(x) = \{0, 5\}$; $\lprof(y) = \{2, 6\}$; D]
  \label{ex:0526D}
  \begin{equation}
    \begin{array}{*{13}c}
      \bm{a}&b&c&d&e&f&\boxed{g}&h&i&j&k&l&m\\
      i&h&\boxed{g}&j&k&\bm{a}&l&m&b&c&d&e&f
    \end{array}
  \end{equation}
\end{example}

\begin{example}[$\lprof(x) = \{0, 6\}$; $\lprof(y) = \{2, 7\}$; D]
  \label{ex:0627D}
  \begin{equation}
    \begin{array}{*{13}c}
      \bm{a}&b&c&d&e&f&g&\boxed{h}&i&j&k&l&m\\
      j&i&\boxed{h}&k&l&m&\bm{a}&b&c&d&e&f&g
    \end{array}
  \end{equation}
\end{example}

\begin{example}[$\lprof(x) = \{1, 5\}$; $\lprof(y) = \{4, 7\}$; D]
  \label{ex:1547D}
  \begin{equation}
    \begin{array}{*{13}c}
      a&\bm{b}&c&d&e&f&g&\boxed{h}&i&j&k&l&m\\
      l&k&j&i&\boxed{h}&\bm{b}&a&m&c&d&e&f&g
    \end{array}
  \end{equation}
\end{example}

\begin{example}[$\lprof(x) = \{0, 4\}$; $\lprof(y) = \{3, 6\}$; D]
  \label{ex:0436D}
  \begin{equation}
    \begin{array}{*{13}c}
      \bm{a}&b&c&d&e&f&\boxed{g}&h&i&j&k&l&m\\
      j&i&h&\boxed{g}&\bm{a}&k&l&m&b&c&d&e&f
    \end{array}
  \end{equation}
\end{example}

\begin{example}[$\lprof(x) = \{0, 5\}$; $\lprof(y) = \{3, 7\}$; D]
  \label{ex:0537D}
  \begin{equation}
    \begin{array}{*{13}c}
      \bm{a}&b&c&d&e&f&g&\boxed{h}&i&j&k&l&m\\
      k&j&i&\boxed{h}&l&\bm{a}&m&b&c&d&e&f&g
    \end{array}
  \end{equation}
\end{example}

\begin{example}[$\lprof(x) = \{0, 5\}$; $\lprof(y) = \{4, 8\}$; D]
  \label{ex:0548D}
  \begin{equation}
    \begin{array}{*{13}c}
      \bm{a}&b&c&d&e&f&g&h&\boxed{i}&j&k&l&m\\
      m&l&k&j&\boxed{i}&\bm{a}&b&c&d&e&f&g&h
    \end{array}
  \end{equation}
\end{example}


\end{document}
